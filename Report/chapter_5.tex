\chapter{Psychologie}
\label{chapter:psychologie}

\section{psychologische Grundlagen}

"Es wurden 6 soziale Einfallstore und Mental Shortcuts identifiziert:
o Hilfsbereitschaft
o Leichtgläubigkeit
o Neugier
o (Wunsch nach) Anerkennung
o Druck
o Angst.
"\cite{10_bka}

"Es gibt keinen Abwehrzauber gegen Social-Engineering, denn dabei handelt es sich um Verhalten,
das in der Regel sozial erwünscht ist. Technische
Maßnahmen sind nicht in der Lage, derartige Vorfälle zu verhindern, da es
sich um ein soziales Problem handelt. Zur Abwehr wird die Fähigkeit benötigt,
soziale Beziehungen und Kontexte zu deuten. Es ist notwendig, in Organisationen
ein Sicherheitsbewusstsein im Rahmen einer Sicherheitskultur zu schaffen"\cite{10_bka}

"Prognostisch bleibt zu befürchten, dass SE-Fälle in Zukunft eher ansteigen als abnehmen
werden und die Aufklärung problematisch bleibt. Gründe hierfür sind insbesondere:

Die Betrügereien werden weiterhin und zunehmend aus dem Ausland oder von nicht
zu identifizierenden Rechnern oder Personen begangen. Dadurch sinkt das Entdeckungsrisiko.
Scham oder die Angst vor Reputationsverlust kann die Anzeigebereitschaft hemmen.
Die Aussicht auf immense (schwer abzuschöpfende) Gewinne erhöht den Tatanreiz.
Die Verfügbarkeit relevanter offener Informationen, die für einen SE-Angriff genutzt
werden können, wird eher ansteigen als abnehmen. Dadurch werden Manipulationen
erleichtert.
Der Druck auf einzelne Mitarbeiter in der heutigen Arbeitswelt steigt eher als dass er
sinkt und der notwendige Rückhalt/ das Vertrauen in die Organisation, sich vermeintlichen Anweisungen zunächst zu widersetzen, ist nicht immer vorhanden.
Die "Europäisierung des Betruges" wir nicht adäquat mit der Europäisierung der
Strafverfolgung beantwortet und "die internationale Rechtshilfe ist in hohem Maße defizitär"."\cite{10_bka}


\section{menschliche Beeinflussung}

\subsection{Einflussprinzipien}

Ein Angreifer kann die Entscheidungsfähigkeit zu seinem Vorteil beeinflussen, denn Menschen reagieren oftmals mit automatisiertem Sozialverhalten \bcite{10_bka}.
Der Psychologe Robert Cialdini entwickelte die sechs Prinzipien der Überzeugung (\qq{Six Principles of Persuasion}),
welche in weitreichenden Studien demonstiert wurden\footnote{Da in der Psychologie Situationsfaktoren und menschliche Emotionen immer eine Rolle spielen, ist der Erfolg bei Anwendung dieser Prinzipien nicht garantiert.}\bcite{7_mdpi}.
Die sechs Prinzipien bestehen aus:

\subsubsection{Authority}
Die meisten Menschen neigen dazu, Autoritätspersonen, beziehungsweise Personen, mit Fachwissen oder ergiebigen Erfahrungen, zu glauben, und gehorchen den Anweisungen eben jener.
Autorität (engl. \qqq{Authority}) kann Menschen selbst dazu verleiten, gegen ihren Glauben oder ihre moralische Vorstellung zu handeln.
Eine Person gilt als Autoritätssymbol, wenn sie als legitimer Experte wahrgenommen wird. Symbole der Autorität, wie etwa Titel, äu\ss eres Erscheinungsbild oder Statussymbole, wie
luxuriöse Gegenstände, erhöhen die Gefolgsamkeit bei Anderen \bcite{7_mdpi,psyprinciples,10_bka}.

\subsubsection{Commitment \& Consistency}
Konsistenz (engl. \qqq{Consistency}) sorgt dafür, dass Personen sich konsequent zu ihren Verpflichtungen (engl. \qqq{Commitments}) und Überzeugungen verhalten.
Das menschliche Verlangen nach Konsistenz gegenüber eingegangenen Verpflichtungen kann das Verhalten einer Person langzeitlich beeinflussen.
Beispielsweise lässt sich dieses Verhalten insofern feststellen, dass Personen die eine kleine Petition unterschreiben später wesentlich gewillter sind, sich auch anderweitig zu diesem Zweck zu engagieren \bcite{7_mdpi,psyprinciples,10_bka}.

\subsubsection{Reciprocity}
Das Prinzip der Gegenseitigkeit (engl \qqq{Reciprocity}) beruht auf der fundamentalen Tendenz, dass sich Menschen dazu verpflichtet fühlen, Gefallen oder Geschenke zu erwiedern.
Dieses Prinzip ist derartig stark, dass die Erwiederung vehementer ausfallen kann, als den initial erhaltenen Gefallen \bcite{7_mdpi,psyprinciples,10_bka}.

\subsubsection{Liking}
Aufgrund des grundlegenden Motivs, soziale Beziehungen aufzubauen und aufrechtzuhalten, führen Menschen die Anfragen Anderer eher aus, wenn sie diese Person kennen oder mögen (engl. \qqq{like}).
Wahrgenommene Ähnlichkeiten erhöhen die Fügsamkeit einer Person. Diese können durchaus oberflächlicher Natur sein, wie etwa ein gemeinsamer Geburtstag oder Name.
Andere Faktoren, die dazu beitragen von einer Person gemocht zu werden, sind physische Attraktivität und positive Assoziation, etwa durch Komplimente \bcite{7_mdpi,psyprinciples,10_bka}.

\subsubsection{Social Proof}
Das Prinzip des sozialen Beweises (engl. \qqq{social proof}) ist ein Phänomen, dass in der Psychologie als \qqq{informativer sozialer Einfluss} bekannt ist.
Es geht um eine mächtige Überzeugungstaktik, die ausnutzt, dass Menschen eine natürliche (bzw. primitive) Tendenz haben, dem Beispiel Anderer folgen zu wollen, um sozial akzeptiert zu werden \bcite{7_mdpi,psyprinciples,10_bka}.

\subsubsection{Scarcity}
Das letzte Prinzip der Überzeugung ist die Knappheit (engl. \qqq{scarcity}), und beschreibt die Tatsache, dass Menschen einer geringeneren Quantität eine höhere Qualität zuschreiben.
Dieses Prinzip ist nicht ausschließlich auf Materielles anzuwenden, sondern gilt beispielsweise auch bei verhaltenstechnisch weniger verfügbaren Möglichkeiten, oder bei Informationen, die nicht allgemein verfügbar sind.
So wirken Informationen, die einem im Geheimen anvertraut werden, oft spektakulärer \bcite{7_mdpi,psyprinciples,10_bka}.

\subsection{Persönlichkeitsmerkmale}

Das Fünf-Faktor-Modell ist ein Modell der Persönlichkeitspsychologie, welches fünf Kernaspekte der Persönlichkeit identifiziert und besagt, dass jeder Mensch sich auf den Skalen dieser Kernaspekte einordnen lässt.
Das Modell wird auch als OCEAN-Modell bezeichnet (nach den entsprechenden Anfangsbuchstaben Openness, Conscientiousness, Extraversion, Agreeableness, Neuroticism\footnote{zu Deutsch: Offenheit, Gewissenhaftigkeit, Extraversion, Verträglichkeit, Neurotizismus}).
Die folgende Tabelle beschreibt knapp die verschiedenen Kerneigenschaften:

\begin{table}[!h]
    \begin{center}
        \caption{Charakteristiken der OCEAN-Modell Persönlichkeitseigenschaften \bcite{psyse}}
        \begin{tabular}{ |c|m{11em}|m{11em}| }
            \hline
            \textbf{Eigenschaft} & \textbf{hoher Wert}                                                      & \textbf{niedriger Wert}                                   \\
            \hline \hline
            Offenheit                           & intellektuell, fantasievoll, kontaktfreudik. Offen für Neues             & praktisch, konventoniell, skeptisch, rational             \\
            \hline
            Gewissenhaftigkeit                  & organisiert, eigenständig, gründlich, zuverlässig, aber kontrollierend   & desorganisiert, nachlässig, kann anfällig für Sucht sein  \\
            \hline
            Extraversion                        & aufgeschlossen, enthusiastisch, aktiv; sucht Neues                       & distanziert, ruhig, unabhängig; vorsichtig, zurückgezogen \\
            \hline
            Verträglichkeit                     & vertrauensvoll, unkompliziert, empathisch, nachgiebig, umgänglich        & unkooperativ, feindselig; unempathisch                    \\
            \hline
            Neurotizismus                       & anfällig für Stress, Angst, Befangenheit, Launenhaftigkeit, Impulsivität & emotional stabil, ruhig und sicher.                       \\
            \hline
        \end{tabular}
    \end{center}
\end{table}

\subsection{Anfälligkeit für Social Engineering}





