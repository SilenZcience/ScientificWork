\chapter{Psychologie}

"Es wurden 6 soziale Einfallstore und Mental Shortcuts identifiziert:
o Hilfsbereitschaft
o Leichtgläubigkeit
o Neugier
o (Wunsch nach) Anerkennung
o Druck
o Angst.
"\cite{10_bka}

"Der Mensch reagiert auf bestimmte Auslösemerkmale mit automatisiertem Sozialverhalten. Regeln mit solch hoher gesellschaftlicher Durchschlagkraft lassen sich leicht
missbrauchen:
12
o Die Regel der Reziprozität (Wechselseitigkeit, d.h. wir müssen uns für erhaltene Gefälligkeiten, Geschenke etc. revanchieren. Auf Zugeständnisse müssen wir mit Zugeständnissen reagieren). Falls aber durch Bewusstheit/ Sensibilität erkannt wird, dass der Gefallen oder das Geschenk in Wirklichkeit nur
ein Manöver war, um Vorteile zu erlangen, verliert die Reziprozitätsregel ihre
Durchschlagskraft.
o Das Kontrastprinzip (Kontraste erscheinen durch eine geschickte Präsentation
größer als sie unter anderen Umständen erscheinen würden).
o Die Regel des Commitments und der Konsistenz (d.h. den Menschen wohnt
ein geradezu zwanghaftes Verhalten inne, in Konsistenz mit ihren früheren
Handlungen zu erscheinen - also konsequent zu sein. Wurde eine Entscheidung getroffen, treten intra- und interpsychische Vorgänge in Kraft, die uns
dazu drängen, konsistent zu bleiben. In der Sozialpsychologie und dem Marketing arbeitet man daher mit der sog. "Fuß-in-der-Tür-Taktik". Man beginnt
mit einer kleinen Bitte und arbeitet sich dann zur großen vor oder verändert
das Selbstbild des Gegenübers in die gewünscht Richtung. Hat man das
Selbstbild einer Person erst einmal in eine neue Rolle manipuliert, tut die Person nahezu alles um mit dem neuen Selbstbild konsistent zu bleiben). Zumeist spürt der Mensch, dass er betrogen oder ausgenutzt werden soll, achtet
aber nicht auf dieses "Bauch-Gefühl". Durch Awareness-Training kann hier,
als Gegenmaßnahme, eingegriffen werden.
o Das Prinzip der sozialen Bewährtheit (Das Verhalten anderer wird als richtig
angenommen und gegebenenfalls kopiert bzw. adaptiert).
o Sympathie (Uns sympathische Menschen können uns eher zu einem bestimmten Verhalten verleiten). Sympathie verstärkt alle anderen eingesetzten
Überzeugungstricks. Dazu gehören auch Attraktivität, Ähnlichkeit, gleiche
Herkunft, ähnliche Interessen, Schmeicheleien, Sympathiebekundungen, Flirts.
o Autorität (Autoritätssymbole: Titel, Uniform, Luxus).
o Knappheit (je knapper eine Ware ist, desto mehr gewinnt sie an Wert). "\cite{10_bka}

"Es gibt keinen Abwehrzauber gegen Social-Engineering, denn dabei handelt es sich um Verhalten,
das in der Regel sozial erwünscht ist. Technische
Maßnahmen sind nicht in der Lage, derartige Vorfälle zu verhindern, da es
sich um ein soziales Problem handelt. Zur Abwehr wird die Fähigkeit benötigt,
soziale Beziehungen und Kontexte zu deuten. Es ist notwendig, in Organisationen
ein Sicherheitsbewusstsein im Rahmen einer Sicherheitskultur zu schaffen"\cite{10_bka}

"Prognostisch bleibt zu befürchten, dass SE-Fälle in Zukunft eher ansteigen als abnehmen
werden und die Aufklärung problematisch bleibt. Gründe hierfür sind insbesondere:

 Die Betrügereien werden weiterhin und zunehmend aus dem Ausland oder von nicht
zu identifizierenden Rechnern oder Personen begangen. Dadurch sinkt das Entdeckungsrisiko.
 Scham oder die Angst vor Reputationsverlust kann die Anzeigebereitschaft hemmen.
 Die Aussicht auf immense (schwer abzuschöpfende) Gewinne erhöht den Tatanreiz.
 Die Verfügbarkeit relevanter offener Informationen, die für einen SE-Angriff genutzt
werden können, wird eher ansteigen als abnehmen. Dadurch werden Manipulationen
erleichtert.
 Der Druck auf einzelne Mitarbeiter in der heutigen Arbeitswelt steigt eher als dass er
sinkt und der notwendige Rückhalt/ das Vertrauen in die Organisation, sich vermeintlichen Anweisungen zunächst zu widersetzen, ist nicht immer vorhanden.
 Die "Europäisierung des Betruges" wir nicht adäquat mit der Europäisierung der
Strafverfolgung beantwortet und "die internationale Rechtshilfe ist in hohem Maße defizitär"."\cite{10_bka}

"The Dual Process Model of Persuasion [19] defines two different ways how we process information: the peripheral route or heuristic processing via intuition
(system 1) and the central route via reasoning (system 2) (cf. [18]). Attackers can target both systems."\cite{7_mdpi}

"six principles of influence":\cite{7_mdpi}

"Authority
Most people comply to authorities (cf. Milgram experiments [24]), even if they persuade them to act against their beliefs and ethics. It also works for symbols of authority, e.g. uniforms, badges, and titles or in telephone conversations where authority can easily be claimed. Two types of authority exist: one based on expertise and one relying on the relative hierarchical position in an organisation or society [19].

Commitment \& Consistency
Commitment is an act of stating what one person thinks he is and does, while consistency makes that same person behave consistently according to his or her commitments and beliefs revealing a a highly successful influence principle [19].

Reciprocity
A strong social norm that obliges us to repay others for what we have received from them. Relationships rely and societies are built on it. Reciprocity helps establishing trust with others and refers to our need for equity. The power of reciprocity can be so high that the target would return an even greater favour than what was received.

Liking
'If you make it plain you like people, it's hard for them to resist liking you back” [25]. We prefer to comply with requests from people we know and like due to the fundamental motive to create and maintain social relationships. Perceived similarity enhances compliance as it can originate from a potential friend. These can be as superficial as shared names or birthdays.

Social Proof
Besides adapting beliefs and behaviour of people around in order to become socially “accepted”, social proof also implies higher trust levels towards people who share alike opinions, especially in ambiguous situations.

Scarcity
We assign more value to less available opportunities due to a short-cut from availability to quality. Moreover, if something becomes scarce, we sense losing freedoms. Reactance Theory [26] suggests that we respond to scarcity by wanting to have what has become rare more than before. Even information with limited access persuades better."\cite{7_mdpi}

"In psychology, personality is defined as a person's relatively stable feelings, thoughts, and behavioural patterns. These are predominantly determined by inheritance, social and environmental influence, and experience, and are therefore unique for every individual [27]."\cite{7_mdpi}


"The FFM consists of five broad, empirically derived personality dimensions or traits, which split in several subtraits and are used across research areas with high validity: these traits are defined as Conscientiousness which focus on competence, self-discipline, self-control, persistence, and dutifulness as well as following standards and rules. Extraversion comprises positive emotions, sociability, dominance, ambition, and excitement seeking. Agreeableness includes compassion, cooperation, belief in the goodness of mankind, trustfulness, helpfulness, compliance, and straightforwardness. Openness to experience encompasses as a preference for creativity, flexibility’ fantasy as well as an appreciation of new experiences and different ideas and beliefs. Neuroticism describes the tendency to experience negative emotions, anxiety, pessimism, impulsiveness, vulnerability to stress, and personal insecurity."\cite{7_mdpi}

