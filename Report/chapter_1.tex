\chapter{Einleitung}
\label{chapter:einleitung}

Social Engineering ist konträr zu seiner modernen Namensgebung sehrwohl bereits seit
Menschengedenken existent. Es lassen sich Beispiele von Social Engineering in der Mythologie,
Religion und Geschichte der Menschheit finden.
Unter den prominäntesten Beispielen ist das Trojanische Pferd\footnote{Es wird erzählt, dass
die Griechen den Krieg gegen Troja gewannen,
indem sich Odysseus die Social Engingeering Taktik ausdachte, das hölzerne Pferd zu bauen,
und die Trojaner zu manipulieren, dieses in die eigene Stadt zu bringen.}\bcite{origins,origins2}.

Social Engineering Angriffe dienen also seit Langem als Grundlage für die unterschiedlichsten Betrugsmaschen,
aber nehmen im digitalen Zeitalter quantitativ kontinuierlich zu.
Sie zielen darauf ab durch Manipulation an sensible oder wertvolle Daten zu gelangen
und richten damit immensen Schaden an \bcite{seofwnep,4_mdpi,2_bsi}.
Diese Form von Angriffen richtet sich nicht nur gegen Unternehmen und Regierungsinstitutionen,
sondern auch gegen Individuen (insbesondere bezüglich Identitätsdiebstahl) \bcite{7_mdpi,verizon2012}.

Mit der Entwicklung heutiger ICT\footnote{Informationen and Communication Technology} entwickeln sich auch
Social Engineering Taktiken beständig weiter und mit neuen technologischen Möglichkeiten werden auch
konsequent neue Formen des Social Engineering ermöglicht. Heutzutage verwenden die meisten
Cyber-Angriffe eine Form des Social Engineerings \bcite{1_enisa,evolving}.

Social Engineering stellt also eine allgemeine Gefahr für jeden dar, welshalb sich jeder über dieses
Thema informieren sollte um sich entsprechend schützen zu können.

Insbesondere aufgrund dessen, dass Social Engineering ein gesellschaftliches Phänomen ist, welches bereits
schon lange existiert, analysiert dieser Report das Thema hinsichtlich der Frage wieso es keine konsequent effektiven
Methoden gibt, um Social Engineering Angriffen entgegenzuwirken.
In \autoref{chapter:se} wird ein grundlegender Überblich bezüglich Social Engineering, und seinen Angriffsmethoden, verschafft.
\autoref{chapter:konsequenzen} untermauert die Vehemenz des Themas, indem die Konsequenzen erfolgreicher Angriffe vermittelt werden.
Darauffolgend analysieren \autoref{chapter:massnahmen} und \autoref{chapter:psychologie} Social Engineering hinsichtlich der
Forschungsfrage auf sowohl technische und psychologische Weise, und zuletzt wird eine Konklusion genannt.

% Zuletzt w




% Subsection \ref{next_subsection} is not useless, it shows how to include figures.


% \subsection{Next Subsection}
% \label{next_subsection}

