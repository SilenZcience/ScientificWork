\chapter{Was ist Social Engineering}

"has significantly evolved with ICT technologies"\cite{1_enisa}
(Informationen and Communication Technology)

"most cyber attacks nowadays include some form of social engineering"\cite{1_enisa}

"Social Engineering ist an sich nichts Neues und dient seit Menschengedenken als Grundlage
für die unterschiedlichsten Betrugsmaschen. Im Zeitalter der digitalen Kommunikation ergeben
sich jedoch äußerst effektive, neue Möglichkeiten für Kriminelle, mit denen sie Millionen von
potenziellen Opfern erreichen können."\cite{2_bsi}

first we need to tlook what se is. different sources define different things ...

\section{Definition}

"Social engineering refers to all techniques aimed at talking a target into revealing specific information or performing a
specific action for illegitimate reasons."\cite{1_enisa}

"Beim Social Engineering werden menschliche Eigenschaften wie Hilfsbereitschaft, Vertrauen, Angst oder Respekt vor Autorität
ausgenutzt, um Personen geschickt zu manipulieren. Cyber-Kriminelle verleiten das Opfer auf diese Weise beispielsweise dazu,
vertrauliche Informationen preiszugeben, Sicherheitsfunktionen auszuhebeln, Überweisungen zu tätigen oder Schadsoftware auf dem
privaten Gerät oder einem Computer im Firmennetzwerk zu installieren"\cite{2_bsi}

"Der Studie liegt die Definition des Verfassungsschutzes Brandenburg zugrunde: "Social Engineering ist der Versuch unter Ausnutzung menschlicher Eigenschaften Zugang zu Knowhow zu erhalten. Der Angreifer nutzt dabei Dankbarkeit, Hilfsbereitschaft, Stolz, Karrierestreben, Geltungssucht, Bequemlichkeit oder Konfliktvermeidung aus. Dabei bieten häufig soziale Netzwerke oder auch Firmenwebseiten Möglichkeiten, um sich auf sein Opfer gründlich
vorzubereiten. Zu diesen "Vorfeldermittlungen" können auch Anrufe im Unternehmen gehören. Professionelle Angreifer versuchen dabei nicht, mit einem Anruf alle gewünschten Informationen zu erlangen, dies könnte misstrauisch stimmen. Der Angerufene wird dabei im
Gespräch nach vermeintlich nebensächlich erscheinenden Informationen gefragt." Kurzfassung Definition Studie: Social Engineering ist eine zwischenmenschliche Manipulation, bei
der ein Unbefugter unter Vortäuschung falscher Tatsachen versucht, unberechtigten Zugang
zu Informationen oder IT-Systemen zu erlangen."\cite{10_bka}

Subsection~\ref{next_subsection} is not useless, it shows how to include figures.


\subsection{Next Subsection}
\label{next_subsection}

