\chapter{Einleitung}
\label{Einleitung}

Social Engineering ist konträr zu seiner modernen Namensgebung sehrwohl bereits seit
Menschengedenken existent. Es lassen sich Beispiele von Social Engineering in der Mythologie,
Religion und Geschichte der Menschheit finden.
Unter den prominäntest\-en Beispielen ist das Trojanische Pferd\footnote{Es wird erzählt, dass
die Griechen den Krieg gegen Troja gewannen,
indem sich Odysseus die Social Engingeering Taktik ausdachte, das hölzerne Pferd zu bauen,
und die Trojaner zu manipulieren, dieses in die eigene Stadt zu bringen.}\bcite{origins,origins2}.

Social Engineering Angriffe dienen also seit Langem als Grundlage für die unterschiedlichsten Betrugsmaschen,
aber nehmen im digitalen Zeitalter quantitativ kontinuierlich zu.
Sie zielen darauf ab durch Manipulation an sensible oder wertvolle Daten zu gelangen
und richten damit immensen Schaden an \bcite{seofwnep,4_mdpi,2_bsi}.
Diese Form von Angriffen richtet sich nicht nur gegen Unternehmen und Regierungsinstitutionen,
sondern auch gegen Individuen (insbesondere bezüglich Identitätsdiebstahl) \bcite{7_mdpi,verizon2012}.

Mit der Entwicklung heutiger ICT\footnote{Informationen and Communication Technology} entwickeln sich auch
Social Engineering Taktiken beständig weiter und mit neuen technologischen Möglichkeiten werden auch
konsequent neue Formen des Social Engineering ermöglicht. Heutzutage verwenden die meisten
Cyber-Angriffe eine Form des Social Engineerings \bcite{1_enisa,evolving}.

\section{Motivation}

Das primäre Motiv von Cyber-Kriminellen ist finanziell. Im Durchschnitt richtet ein 'Data Breach'
4.45 Millionen US-Dollar an Schaden an, wobei Social Engineering als initialer Angriffsvektor noch
über diesem Wert liegt \bcite{6_ibmsecurity}.
Im Jahr 2016 wurde die Bangladesh Bank gehackt, was zu einem immensen finanziellen Verlust führte.
Der Angriff wurde langwierig geplant und begannt bereits ein Jahr zuvor.
Es gelang den Cyber-Kriminellen in das \textit{SWIFT} Bank Netzwerk einzudringen, welches für Geld-überweisungen
genutzt wird. Insgesamt sollten 1 Milliarden US-Dollar transferiert werden, wobei es den Angreifern
letztendlich nur möglich war, 81 Millionen US-Dollar zu stehlen.

Der Verlust, nach erfolgreichen Social Engineering Angriffen, ist jedoch für Unterneh\-men weitreichender.
Neben dem direkten finanziellen Verlust, durch den Diebstahl der Angreifer erleiden Unternehmen zusätzlich
Wiederherstellungskosten, da etwaige Daten verloren gegangen sind, und Sicherheitslücken gefunden und repariert
werden müssen. Des Weiteren entsteht eine Betriebsdisruption, was zu indirektem finanziellen Schaden, durch
Verlust von Produktivität, führt. Zuletzt erleiden Unternehmen einen Reputationsschaden, was in vielen Fällen
den verheerendsten Faktor ausmacht, insbesondere für kleinere Firmen \bcite{agony}.
Für Unternehmen können Cyber-Angriffe auch eine Form der Espionage darstellen, weshalb der Schaden eines
Unternehmens zusätzlich einen kompetitiven Schaden in der Marktwirtschaft darstellen kann.

Individuen erleiden ebenfalls, neben finanziellen-, auch weitere Formen von Schäden.
Nicht außer Acht zu lassen ist der emotionale Schaden, da Personen oft, in Folge einer erfolgreichen
Manipulation, als naiv dargestellt werden.


% TODO: verweis auf explizite SE angriffe im film



% \begin{figure}[H]
%     \centering
%     \includegraphics[width=5in]{IBM_Data.Breach.Report.png}
%     \caption{IBM - Measured in USD millions}
% \end{figure}
% Avg. Kosten pro Breach (2022)


% \begin{figure}[H]
%     \centering
%     \includegraphics[scale=.5]{Barracuda_Social.Engineering.Attacks.png}
%     \caption{Barracuda - Social Engineering Attacks}
% \end{figure}


% \newpage

% "Actor Motives Financial (89\%), Espionage (11\%) (breaches)"\cite{verizon2022}
% "Actor Motives Financial (95\%), Espionage (5\%) (breaches)"\cite{verizon2024}

% conversation hijacking wenig ,denn verlangt etwas erfolg bei vorherigen angriffen.
% z.b. folgt onversation hijacking oftmals auf account takeover.

% "Hackers are starting to increasingly use phishing as part of their
% ransomware attacks."\cite{3_barracuda}

% "Extortion attacks make up only 2\% of the total number of
% targeted phishing attacks we have seen in the past year. These
% attacks were mostly sextortion email threats, where hackers
% threaten to expose sensitive or embarrassing content to their
% victim’s contacts unless a ransom is paid. Demands are usually
% a few hundred or a few thousand dollars and need to be paid
% in bitcoin, which is difficult to trace. In the UK, the number of
% sextortion cases reported to National Crime Agency increased
% by 88\% between 2018 and 2020, and the number is expected to
% continue to increase"\cite{3_barracuda} stand 2021
% stand 2024: Extortion: ($\sim$ 25\%) \cite{verizon2024}
% Darunter fällt auch Ransomware

% Pretexting: 2022 (27\%), 2024 (more than 40\%)
% Phishing: 2022 ($\sim$ 70\%), 2024 (31\%)\cite{verizon2024,verizon2022}

% "Account takeover is a form of identity theft and fraud where a
% malicious third party successfully gains access to a user’s account
% credentials"\cite{3_barracuda}
% "Account takeover is one of the fastest growing threats. In 2021,
% roughly 1 in 5 organizations (20\%) had at least one of their
% Microsoft 365 accounts compromised. This means that in 2021
% hackers managed to compromise around 500,000 Microsoft 365
% accounts around the globe."\cite{3_barracuda}

% Social Engineering MTTI (Mean Time To Identify): 218 Days
% Social Engineering MTTC (Mean Time To Contain) :  80 Days\cite{6_ibmsecurity}




% Subsection \ref{next_subsection} is not useless, it shows how to include figures.


% \subsection{Next Subsection}
% \label{next_subsection}

