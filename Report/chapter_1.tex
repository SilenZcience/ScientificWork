\chapter{Einleitung}
\label{chapter:einleitung}

Social Engineering ist konträr zu seiner modernen Namensgebung sehrwohl bereits seit
Menschengedenken existent. Es lassen sich Beispiele von Social Engineering in der Mythologie,
Religion und Geschichte der Menschheit finden.
Unter den prominentesten Beispielen ist das Trojanische Pferd\footnote{Es wird erzählt, dass
die Griechen den Krieg gegen Troja gewannen,
indem sich Odysseus die Social Engingeering Taktik ausdachte, das hölzerne Pferd zu bauen,
und die Trojaner zu manipulieren, dieses in die eigene Stadt zu bringen.}\bcite{origins,origins2}.

Social Engineering Angriffe dienen also seit Langem als Grundlage für die unterschiedlichsten Betrugsmaschen,
aber nehmen im digitalen Zeitalter quantitativ kontinuierlich zu.
Sie zielen darauf ab, durch Manipulation an sensible oder wertvolle Daten zu gelangen
und richten damit immensen Schaden an \bcite{seofwnep,4_mdpi,2_bsi}.

In 2016 erklärte Cyence, ein Cybersicherheitsanalyseunternehmen, dass Deutschland nach den Vereinigten Staaten
das Land ist, mit den meisten Social Engineering Angriffen; doch dem U.S. Department of Justice zu Folge stellt
dies sogar eine der weltweit bedeutsamsten Gefahren dar.
In demselben Jahr (2016) wurde die Bangladesch Bank gehackt, was zu einem immensen finanziellen Verlust führte.
Der Angriff wurde langwierig geplant und begannt bereits ein Jahr zuvor.
Es gelang den Cyber-Kriminellen in das \textit{SWIFT} Bank Netzwerk einzudringen, welches für Geldüberweisungen
genutzt wird. Bei diesem Angriff wurden verschiedenste Social Engineering Methoden angewandt.
Insgesamt sollten 1 Milliarden US-Dollar transferiert werden, wobei es den Angreifern
letztendlich nur möglich war, 81 Millionen US-Dollar zu stehlen.

Mit der Entwicklung heutiger ICT\footnote{Informationen and Communication Technology} entwickeln sich auch
Social Engineering Taktiken beständig weiter und mit neuen technologischen Möglichkeiten werden auch
konsequent neue Formen des Social Engineering ermöglicht.
So gelang es 2019 Hackern erfolgreich einen Social Engineering Angriff auf ein unbenanntes Energieunternehmen
durchzuführen, indem mit deepfake Technologie der CEO der Firma imitiert wurde. Die Audiodaten waren ausreichend
authentisch, sodass die Angreifer einen Angestellten davon überzeugen konnten eine Überweisung in Höhe von
243.000 Dollar zu tätigen \bcite{deepfake}.

Heutzutage verwenden die meisten Cyber-Angriffe eine Form des Social Engineerings \bcite{1_enisa,evolving}.
Genauer setzen etwa 98\% aller Angriffe auf denen Menschen als Schwachstelle \bcite{cdse}.
Diese Form von Cyber-Angriffen richtet sich nicht nur gegen Unternehmen und Regierungsinstitutionen,
sondern auch gegen Individuen (insbesondere bezüglich Identitätsdiebstahl) \bcite{7_mdpi,verizon2012}.

Social Engineering stellt also eine allgemeine Gefahr für jeden dar, weshalb sich jeder über dieses
Thema informieren sollte, um sich entsprechend schützen zu können.

Insbesondere aufgrund dessen, dass Social Engineering ein gesellschaftliches Phänomen ist, welches bereits
schon lange existiert, analysiert dieser Report das Thema hinsichtlich der Frage, wieso es keine konsequent effektiven
Methoden gibt, um Social Engineering Angriffen entgegenzuwirken.
In \autoref{chapter:se} wird ein grundlegender Überblick bezüglich Social Engineering und seinen Angriffsmethoden verschafft.
\autoref{chapter:massnahmen} beschäftigt sich mit validen Gegenmaßnahmen zu Social Engineering.
Darauffolgend analysieren \autoref{chapter:auswirkungen} und \autoref{chapter:psychologie} Social Engineering hinsichtlich der
Forschungsfrage auf sowohl technische und psychologische Weise. Zuletzt wird eine Konklusion genannt.
