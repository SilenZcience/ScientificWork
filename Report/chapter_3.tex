\chapter{Maßnahmen}
\label{chapter:massnahmen}

\section{Erkennung \& Vorbeugung}

Um verschiedene Angriffe zu erkennen und damit präventiv zu verhindern, werden verschiedene Techniken vorgeschlagen.
Eine Liste von Abwehrverfahren gegen Social Engineering umfasst Förderung von Sicherheitsschulungen und Steigerung des allgemeinen Bewusstseins für Angrifsvektoren durch entsprechende Aufklärung;
Die beste Methode gegen eine soziale Form des Angriffes ist ein soziales Bewusstsein, angegriffen zu werden \bcite{4_mdpi}.
Derartige Schulungen sollten erklären, wie die Sicherheit kritischer Informationen gewährleistet werden kann, und stetig auf aktuelle Angriffsmuster aufmerksam machen, wie etwa bekannte Phishing Kampagnen \bcite{4_mdpi}.
Regelmäßige Poster, Präsentationen, E-Mails und Informationsschreiben können weiter dazu beitragen, das Bewusstsein zu verbreiten.
Es wird zudem empfohlen, dass Organisationen Penetrationstests ausführen, um die Anfälligkeit für Social Engineering zu ermitteln \bcite{1_enisa}.

Viele Angriffe verlieren an Wirksamkeit, wenn ausreichende Identifizierungs- und Authentifizierungsprozesse vorhanden sind.
Beispielsweise bietet Mehr-Faktor-Authentifizierung eine zusätzliche Sicherheitsebene zu Benutzername und Passwort.
Hierbei stehen Optionen, wie ein Authentifizierungscode, ein Daumenabdruck oder ein Netzhautscan zur Verfügung.
Es sollten verschiedene Anmeldedaten für unterschiedliche Plattformen eingesetzt werden \bcite{1_enisa,3_barracuda}.
Um physischen Angriffen, wie Tailgating (\autoref{tailgating}) entgegenzusetzen, sollte der Zugang zu nicht öffentlichen Bereichen durch Zugangsrichtlinien und/oder den Einsatz von Zugangskontrolltechnologien kontrolliert werden.
Die Pflicht, einen Ausweis zu tragen, die Anwesenheit von Sicherheitspersonal und explizite Türen zum Schutz vor Tailgating,
wie Schleusen mit RFID-Zugangskontrolle\footnote{RFID (Radio Frequency Identification signals) verwendet elektromagnetische Ausweise}, reichen häufig aus, um die meisten Angreifer abzuschrecken \bcite{1_enisa}.

Oftmals werden persönliche Informationen, die freiwilig offen gelegt wurden, von Kriminellen missbraucht, weshalb bereits der verantwortungsvolle Umgang mit den Sozialen Netzwerken eine hilfreiche Gegenmaßnahme darstellen kann.
Unter keinen Umständen sollten private oder berufliche Informationen öffenlich preisgegeben werden.

Angriffen wie Baiting (\autoref{baiting}) kann entgegengewirkt werden, wenn entsprechende Systeme installiert sind, welche unauthorisierte Software und Hardware blockieren.
Grundsetzlich gilt, keinen unbekannten Kontaktaufnahmen zu vertrauen \bcite{1_enisa}, und die Legitimität von Anrufen und E-Mails (oder anderweitigen Quellen) ausreichend zu prüfen.
Insbesondere sollten bei E-Mails die drei kritischen Punkte Absender, Betreff und Anhang vor dem Öffnen bedacht und überprüft werden \bcite{2_bsi}.
Links sollten nicht geöffnet werden, bevor diese verifiziert wurden. Beispielsweise lässt sich die URL eines Links bereits inspizieren, bevor dieser angeklickt wurde, beispielsweise durch das Bewegen des Mauszeigers über den Link.
Merkmale, auf die hierbei geachtet werden sollte, sind, ob die URL semantisch unseriös wirkt, und mit \qqq{https} (nicht \qqq{http}) beginnt. 

\section{Justiz}

\bcite{criminal}
