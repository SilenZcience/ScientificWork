\chapter{Konklusion}
\label{chapter:konklusion}

Zusammengefasst gibt es eine Vielzahl von Faktoren, die dazu beitragen, dass dem kontinuierlichen Anstieg von Social Engineering Angriffen nicht ausreichend entgegengewirkt werden kann.
Hinsichtlich der Forschungsfrage stellt sich heraus, dass unter den primären Gründen die simple quantitative Invasion von Angriffen ist, welche einen immensen Schaden verursacht.

Die Intensität und überwältigende Anzahl von Cybercrime Angriffen steigt jährlich vehement an, weshalb es schwierig ist, gezielt gegen Social Engineering vorzugehen.
\qq{Prognostisch bleibt zu befürchten, dass SE-Fälle\footnote{Social Engineering (SE)} in Zukunft eher ansteigen als abnehmen werden und die Aufklärung problematisch bleibt.}\bcite{10_bka}

Obwohl gegen technische Angriffe teilweise vorgegangen werden kann, etwa durch E-Mail Filter oder anderweitige technische Systeme, ist ein aktueller Shift zu beobachten,
dass soziale Angriffsformen die Mehrheit aller Angriffe bilden. Dies stellt insofern eine Problematik dar, dass soziale Social Engineering Taktiken die größte Erfolgschance aufweisen.
Sie zielen auf die direkte zwischenmenschliche Manipulation ab und weisen eine hohe Effektivität auf.

Auf der psychologischen Ebene hat sich herausgestellt, dass fast jede Person manipulierbar ist. Verschiedene Persönlichkeitsmuster bieten keinen Schutz vor Social Engineering Angriffen, da unter Verwendung
passender Methodiken jede Persönlichkeit auf konkrete Manipulationsprinzipien ansprechbar ist.
Social Engineering nutzt menschliches Verhalten aus, welches gesellschaftlich grundsätzlich erwünscht ist.
Beispielsweise zielt ein Angreifer auf die Hilfsbereitschaft und Freundlichkeit seiner Mitmenschen ab.
Da es nicht problemlos möglich ist, derartiges Verhalten zu unterbinden, besteht in einer sozialen Gesellschaft immer das Risiko von Manipulation und die Ausnutzung dieser.
\newpage
Zuletzt stellt die fehlende Kenntnis in der allgemeinen Bevölkerung ein weiteres Hauptproblem dar.
Die meisten Menschen sind nicht mit dem Begriff von Social Engineering oder dem dahinter stehenden Konzept vertraut.
Ohne die notwendige Aufklärung ist es grundsätzlich schwierig, einen Angriff zu erkennen und diesem folglich entgegenzusetzen, um sich oder andere zu schützen.

Im digitalen Zeitalter ist es notwendig, sich ausreichend mit den Gefahren des Internets auseinanderzusetzen.
Von den genannten Gründen für den fehlenden Aufschwung gegen Social Engineering Angriffe scheint die fehlende Bildung ein zunächst lösbares Problem darzustellen.
Zukünftig sollte mehr in die allgemeine Aufmerksamkeit bezüglich der hier präsentierten Problematik investiert werden.
Es ist essenziell weiterhin mit dem aktuellen Stand des Themas in Kontakt zu bleiben und den durch Social Engineering dargestellten Bedrohungen des Internets maximal entgegenzuwirken.
