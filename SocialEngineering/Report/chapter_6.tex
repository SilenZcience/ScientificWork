\chapter{Konklusion}
\label{chapter:konklusion}

prognose hier ...

% "Prognostisch bleibt zu befürchten, dass SE-Fälle in Zukunft eher ansteigen als abnehmen
% werden und die Aufklärung problematisch bleibt. Gründe hierfür sind insbesondere:

% Die Betrügereien werden weiterhin und zunehmend aus dem Ausland oder von nicht
% zu identifizierenden Rechnern oder Personen begangen. Dadurch sinkt das Entdeckungsrisiko.
% Scham oder die Angst vor Reputationsverlust kann die Anzeigebereitschaft hemmen.
% Die Aussicht auf immense (schwer abzuschöpfende) Gewinne erhöht den Tatanreiz.
% Die Verfügbarkeit relevanter offener Informationen, die für einen SE-Angriff genutzt
% werden können, wird eher ansteigen als abnehmen. Dadurch werden Manipulationen
% erleichtert.
% Der Druck auf einzelne Mitarbeiter in der heutigen Arbeitswelt steigt eher als dass er
% sinkt und der notwendige Rückhalt/ das Vertrauen in die Organisation, sich vermeintlichen Anweisungen zunächst zu widersetzen, ist nicht immer vorhanden.
% Die "Europäisierung des Betruges" wir nicht adäquat mit der Europäisierung der
% Strafverfolgung beantwortet und "die internationale Rechtshilfe ist in hohem Maße defizitär"."\cite{10_bka}