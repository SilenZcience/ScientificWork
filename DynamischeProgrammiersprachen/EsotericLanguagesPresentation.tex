\documentclass{beamer}
\usepackage[T1]{fontenc}
\usepackage[dvipsnames]{xcolor}
\usepackage{graphicx, float}
\graphicspath{{res/}}
\usepackage{amsfonts,amsmath}
\usepackage{listings}

% Modernes Theme verwenden
\usetheme{moderno}

\lstset{
  basicstyle=\ttfamily\footnotesize,
  keywordstyle=\color{blue},
  commentstyle=\color{gray},
  stringstyle=\color{red},
  numbers=left,
  numberstyle=\tiny\color{gray},
  breaklines=true,
  frame=single
}

\title{Let's Take Esoteric Programming Languages Seriously}
\subtitle{Geschrieben von J. Singer \& S. Draper}
\author{Silas A. Kraume}
\institute{Heinrich-Heine-Universität Düsseldorf}
\date{Dynamische Programmiersprachen - Präsentation}

\begin{document}
\maketitle

\begin{frame}[noframenumbering]{Inhaltsverzeichnis}
    \tableofcontents
\end{frame}

\section{Einführung \& Definition}

\begin{frame}{Das Phänomen Esolang}
    \begin{itemize}
        \item Unpraktisch und unnötig Komplex
        \item Dennoch überraschend populär in der Informatik-Community
    \end{itemize}
    \vspace{1cm}
    \begin{block}{Kernfrage}
        Warum gibt es sie und welchen Nutzen haben sie?
    \end{block}
\end{frame}

\begin{frame}{Was ist eine esoterische Programmiersprache $\left(\text{"Esolang"}\right)$?}
    \begin{block}{Konventionelle Definition}
        Eine Sprache, die für Experimente mit seltsamen Ideen, schwierige Programmierung oder als Scherz gedacht ist.
    \end{block}
    \vspace{0.5cm}
    \begin{itemize}
        \item Oftmals absichtlich unbrauchbar
        \item Ungewöhnliche oder obskure Syntax/Semantik
    \end{itemize}
    \vspace{0.5cm}
    \textbf{Unterschied zu Mainstream-Sprachen:} Nicht für praktische Anwendung gedacht.
\end{frame}

\begin{frame}{Das Wesen von Esolangs}
    \begin{itemize}
        \item \textbf{Low-Level Semantik:} Meist primitive arithmetische Operationen, begrenzter I/O
        \vspace{0.5cm}
        \item \textbf{Keine Abstraktion:} Keine komplexen Datentypen, keine Modularität
        \vspace{0.5cm}
        \item \textbf{Unkonventionelle Syntax:} Der Code sieht oft gar nicht aus wie Code
    \end{itemize}
\end{frame}

\section{Klassifizierung \& Beispiele}

% \begin{frame}{Der Urvater: INTERCAL}
%     \begin{itemize}
%         \item Erschaffen 1972 als Parodie auf FORTRAN und COBOL
%         \item Name: "Compiler Language With No Pronounceable Acronym"
%     \end{itemize}
%     \vspace{0.5cm}
%     \begin{block}{Besonderheit: Der PLEASE Befehl}
%         \begin{itemize}
%             \item Zu wenig PLEASE: Fehlermeldung (unhöflich)
%             \item Zu viel PLEASE: Fehlermeldung (unterwürfig)
%         \end{itemize}
%     \end{block}
% \end{frame}

\begin{frame}{Klassifizierung von Esolangs}
    Wir teilen Esolangs in drei Kategorien (nicht exklusiv):
    \vspace{1cm}
    \begin{enumerate}
        \item Natürliche Sprache (Natural Language)
        \vspace{0.5cm}
        \item Exotische Syntax (Quirky Syntax)
        \vspace{0.5cm}
        \item Exotisches Rechenmodell (Quirky Computational Model)
    \end{enumerate}
\end{frame}

\begin{frame}[fragile]{Kategorie 1: Natürliche Sprachen - Shakespeare}
    \begin{itemize}
        \item Programme sehen aus wie Theaterstücke Von William Shakespeare
        \item Variablen sind Charaktere (Romeo, Julia)
        \item Wertezuweisung durch Dialoge
    \end{itemize}
    \vspace{0.3cm}
    \begin{block}{Beispiel}
        \small
        \texttt{[Enter Hamlet and Romeo]}
        \newline
        \texttt{Hamlet:}
        \newline
        \texttt{You lying stupid fatherless big smelly half-witted coward! You are as
        stupid as the difference between a handsome rich brave hero and thyself!
        Speak your mind!}
        \newline
        \texttt{\dots}
    \end{block}
\end{frame}

\begin{frame}{Kategorie 1: Natürliche Sprachen - Chef}
    \begin{itemize}
        \item Code sieht aus wie ein Kochrezept
        \item Variablen sind "Zutaten"
        \item Kochanweisungen sind Methodiken
    \end{itemize}
    \vspace{0.3cm}
    \begin{block}{Beispiel}
        \small
        \texttt{Ingredients.}
        \newline
        \texttt{72 g haricot beans}
        \newline
        \texttt{101 eggs}
        \newline
        \newline
        \texttt{Method.}
        \newline
        \texttt{Put eggs into the mixing bowl.}
        \newline
        \texttt{Put haricot beans into the mixing bowl.}
        \newline
        \texttt{Liquefy contents of the mixing bowl.}
        \newline
        \texttt{Pour contents of the mixing bowl into the baking dish.}
        \newline
        \texttt{Serves 1.}
    \end{block}
\end{frame}

\begin{frame}{Kategorie 2: Exotische Syntax - Whitespace}
    \begin{itemize}
        \item Syntax besteht nur aus Leerzeichen, Tabs und Zeilenumbrüchen
        % \vspace{0.5cm}
        \item Für das menschliche Auge unsichtbar
        % \vspace{0.5cm}
        \item Alle anderen Zeichen werden als Kommentare ignoriert
    \end{itemize}
    \vspace{0.3cm}
    \begin{block}{Beispiel}
        \small
        \texttt{}
        \newline
        \newline
        \newline
        \newline
        \newline
        \texttt{}
    \end{block}
\end{frame}

\begin{frame}{Kategorie 2: Exotische Syntax - Piet}
    \begin{itemize}
        \item Code ist ein Bitmap-Bild (ähnlich Mondrian-Kunst)
        % \vspace{0.5cm}
        \item \textbf{Program Flow:} Ein Zeiger bewegt sich 2-dimensional über die Farbfläche
        % \vspace{0.5cm}
        \item Befehle entstehen beim Wechsel von Farben/Helligkeiten
    \end{itemize}
    \vspace{0.3cm}
    \begin{figure}[H]
        \centering
        \includegraphics[width=0.2\textwidth]{Piet_Hello_World.png}
        \caption{Ein "Hello World" Programm in Piet}
    \end{figure}
\end{frame}

\begin{frame}[fragile]{Kategorie 3: Exotische Rechenmodelle - Brainfuck}
    \begin{itemize}
        \item Minimalistische Sprache, simuliert Turing-Maschine
        \item Nur 8 Befehle: \texttt{> < + - . , [ ]}
        \item Extrem begrenzter Speicher (typischerweise 30KB Array)
        \vspace{0.5cm}
        \item \textbf{Jedes "interessante" Programm wird extrem lang und unleserlich}
    \end{itemize}
        \begin{block}{Beispiel}
        \small
        \texttt{++++++++[>++++[>++>+++>+++>+<<<<-]>+>+>->>+[<]<-]>>.>---.+++++++..+++.>>.<-.<.+++.}
        \texttt{------.--------.>>+.>++.}
    \end{block}
\end{frame}

\begin{frame}{Das extremste Beispiel: Malbolge}
    \begin{itemize}
        \item Benannt nach dem 8. Kreis der Hölle in Dantes Inferno
        \item \textbf{Designziel:} Die schwerstmögliche Sprache überhaupt
        \item Selbstmodifizierend: Befehle ändern sich nach Ausführung ("Verschlüsselung")
        \item Erstes Programm wurde erst Jahre später durch einen Such-Algorithmus generiert
    \end{itemize}
    \vspace{0.3cm}
    \begin{block}{Beispiel}
        \small
        \texttt{(=BA\#9"=<;:3y7x54-21q/p-,+*)"!h\%B0/.}
        \newline
        \texttt{$\sim$P<}
        \newline
        \texttt{<:(8\&}
        \newline
        \texttt{66\#"!$\sim$\}|\{zyxwvu}
        \newline
        \texttt{gJ\%}
    \end{block}
\end{frame}

\section{Der Reiz von Esolangs}

\begin{frame}{Warum sind Esolangs so beliebt?}
    Vier Hauptgründe für die Popularität:
    \vspace{1cm}
    \begin{enumerate}
        \item Verspieltheit (Playfulness)
        \vspace{0.3cm}
        \item Nostalgie
        \vspace{0.3cm}
        \item Zugehörigkeitsgefühl (Cultic initiation)
        \vspace{0.3cm}
        \item Künstlerischer Ausdruck
    \end{enumerate}
\end{frame}

\begin{frame}{1. Verspieltheit}
    \begin{quote}
            "Deep down, we like confusing people who are stuffier and less mentally agile than we are [...]" \cite{hacker}
        \end{quote}
    \begin{itemize}
        \item Anti-autoritäre Haltung: Verwirrung stiften mit "snazzy special effects"
        \vspace{0.5cm}
        \item \textbf{Advent of Code:} Viele lösen Rätsel in Esolangs als zusätzliche Herausforderung
    \end{itemize}
\end{frame}

\begin{frame}{2. Nostalgie ("Das Goldene Zeitalter")}
    \begin{itemize}
        \item Sehnsucht nach der Zeit, als Programmieren noch "schwer" war
        \vspace{0.5cm}
        \item \textbf{Das "Haarhemd" tragen (Peyton Jones):} Stolz darauf, unter schwierigen Bedingungen Code zu produzieren
        \begin{itemize}
            \item "Technomasochismus"
        \end{itemize}
        \vspace{0.5cm}
        \item Gegenbewegung zu modernen IDEs, Autocomplete und KI-Hilfen
    \end{itemize}
\end{frame}

\begin{frame}{3. Zugehörigkeit \& Kult}
    \begin{itemize}
        \item Sprache als soziales Bindemittel (wie Klingonisch)
        \vspace{0.5cm}
        \item \textbf{Initiationsritus:} Wer Brainfuck lesen kann, gehört zum "Inner Circle"
        \vspace{0.5cm}
        \item Wissen wird oft mündlich oder in versteckten Wikis weitergegeben, nicht in Lehrbüchern
    \end{itemize}
\end{frame}

\begin{frame}{4. Künstlerischer Ausdruck}
    \begin{itemize}
        \item Code als Artefakt vs. ausführbare Entität
        \vspace{0.5cm}
        \item Esolangs betonen das Artefakt (das Bild bei Piet, das Rezept bei Chef)
        \vspace{0.5cm}
        \item \textbf{Knuth:} "Literate Programming" – Code wird für Menschen geschrieben
        \vspace{0.5cm}
        \item Steganographie: Code versteckt in Bildern oder Texten
    \end{itemize}
\end{frame}

% {\showsectionsubtitlefalse
% \begin{frame}{Exkurs: Sind Esolangs dynamische Sprachen?}
%     \begin{block}{Sind Esolangs dynamisch?}
%         \textbf{Nein.} Esolangs teilen einzelne Eigenschaften mit dynamischen Sprachen (z.B. Interpretation), erfüllen aber nicht das Gesamtpaket.
%     \end{block}
%     \vspace{0.3cm}
%     \begin{block}{Sind Esolangs statisch?}
%         \textbf{Auch nein.} Die meisten haben gar kein Typsystem im klassischen Sinne.
%     \end{block}
%     \vspace{0.5cm}
%     \begin{block}{Die Wahrheit}
%         Esolangs bilden eine \textbf{eigene Kategorie}, die sich der traditionellen Klassifikation entzieht:
%         \begin{itemize}
%             \item Minimalistisch statt abstrakt
%             \item Obfuskation statt Klarheit
%             \item Kunst/Experiment statt Produktivität
%         \end{itemize}
%     \end{block}
%     \vspace{0.2cm}
%     \small
%     Sie sind der "Anti-Typ" – weder dynamisch noch statisch, sondern \textit{absichtlich unpraktisch}.
% \end{frame}
% }

\section{Pädagogischer Wert}

\begin{frame}{Lernen durch Esolangs?}
    \begin{block}{Konzept von "Hard Fun"}
        Lernen ist besonders effektiv, wenn es herausfordernd, aber lohnend ist
    \end{block}
    \vspace{0.5cm}
    \begin{block}{Variationstheorie}
        Dasselbe Konzept in verschiedenen Kontexten sehen, fördert das Verständnis
    \end{block}
    \begin{itemize}
        \item Wer Schleifen in Brainfuck versteht, hat das Konzept "Schleife" \textbf{wirklich} verstanden
    \end{itemize}
\end{frame}


\begin{frame}{Lektionen durch Esolangs}
    \begin{block}{1. Die Gefahr von "GOTO"}
        \begin{itemize}
            \item Dijkstras Kritik an \textit{GOTO} wird greifbar
            \item Beispiel Shakespeare: \textit{Let us proceed to Scene N} erzeugt schnell Spaghetti-Code
            \item Als Anfänger erkennt man dies sofort als problematisch
        \end{itemize}
    \end{block}
    \vspace{0.5cm}
    \begin{block}{2. Das Turing-Tarpit}
        \begin{itemize}
            \item \textit{"Beware of the Turing tar-pit in which everything is possible but nothing of interest is easy"} \cite{turingtarpit}
            \item Turing-Vollständigkeit allein reicht nicht für eine gute Sprache
        \end{itemize}
    \end{block}
\end{frame}

\begin{frame}{Einblick in die Maschine}
    \begin{itemize}
        \item Esolangs entfernen den Komfort
        \vspace{0.5cm}
        \item Man lernt das zugrunde liegende Rechenmodell (Stack, Speicherzellen, Pointer) direkt kennen
        \vspace{0.5cm}
        \item Entmystifizierung dessen, was "unter der Haube" passiert
    \end{itemize}
\end{frame}

\begin{frame}{Empathie für Lehrende}
    \begin{itemize}
        \item Erfahrene Programmierer fühlen sich in Esolangs wieder wie Anfänger
        \vspace{1cm}
        \item Hilft Dozenten, die Frustration von Erstsemestern wieder nachzuvollziehen
    \end{itemize}
\end{frame}

\section{Design von Esolangs \& Sprachdesign}

\begin{frame}{Wer entwickelt Esolangs?}
    \begin{itemize}
        \item Oft Informatik-Studenten (z.B. INTERCAL in Princeton, Whitespace in Durham)
        \vspace{0.5cm}
        \item Oft als Projekt, um Kompilerbau zu lernen
        \vspace{0.5cm}
        \item Demografisch oft männlich dominiert
        \vspace{0.5cm}
        \item Kultur die "Härte" und "Technomasochismus" glorifiziert
    \end{itemize}
\end{frame}

\begin{frame}{Warum entwickelt man eine Esolang?}
    \begin{enumerate}
        \item \textbf{Comedy:} Parodie auf existierende Sprachen (INTERCAL vs. Fortran)
        \vspace{0.5cm}
        \item \textbf{Herausforderung:} Sich selbst extreme Beschränkungen auferlegen
        \vspace{0.5cm}
        \item \textbf{Kreativität:} Eine "eigene" Sprache haben ("A room of one's own")
    \end{enumerate}
\end{frame}

\begin{frame}{Die 5 Motive für Sprachdesign (Die 5 E's)}
    Warum werden Sprachen generell entwickelt?
    \vspace{0.5cm}
    \begin{enumerate}
        \item \textbf{E}xpressivity (Ausdrucksstärke)
        \item \textbf{E}fficiency (Effizienz)
        \item \textbf{E}ducation (Bildung)
        \item \textbf{E}conomy (Wirtschaftlichkeit)
        \item \textbf{E}xploration (Forschung/Neugier)
    \end{enumerate}
\end{frame}

\begin{frame}{Motiv 1: Expressivity}
    \begin{block}{Ziel}
        Komplexe Konzepte kürzer ausdrücken
    \end{block}
    \vspace{0.5cm}
    \textbf{Beispiele:} Lisp, Prolog
    \newline
    % \vspace{0.5cm}
    \textbf{Esolangs:} Das Gegenteil. Sie fügen Komplexität hinzu
\end{frame}

\begin{frame}{Motiv 2: Efficiency}
    \begin{block}{Ziel}
        Nähe zur Hardware, Geschwindigkeit
    \end{block}
    \vspace{0.5cm}
    \textbf{Beispiele:} C, Rust, Zig
    \newline
    \textbf{Esolangs:} Ineffizient, da das Rechenmodell oft abstrakt und umständlich ist
\end{frame}

\begin{frame}{Motiv 3: Education}
    \begin{block}{Ziel}
        Leicht zu lernen
    \end{block}
    \vspace{0.5cm}
    \textbf{Beispiele:} BASIC, Scratch, Logo
    \newline
    \textbf{Esolangs:} Explizit nicht für Anfänger
    \newline
    \small
    (INTERCAL Manual: "besser geeignet, Anfänger dazu zu bringen, den Beruf zu wechseln" \cite{intercal})
\end{frame}

\begin{frame}{Motiv 4: Economy}
    \begin{block}{Ziel}
        finanzieller Gewinn, Marktdominanz
    \end{block}
    \vspace{0.5cm}
    \textbf{Beispiel:} C\# (als Rivale zu Java)
    \newline
    \textbf{Esolangs:} Kein kommerzieller Nutzen (außer ggf. Buchverkäufe)
\end{frame}

\begin{frame}{Motiv 5: Exploration}
    \begin{block}{Ziel}
        Intellektuelle Übung, Grenzen testen
    \end{block}
    \vspace{0.5cm}
    \textbf{Beispiel:} Haskell (ursprünglich)
    \newline
    \newline
    \newline
    \newline
    \Large
    \textbf{Hier leben die Esolangs. Sie sind reine Exploration.}
\end{frame}

% \begin{frame}{Esolangs als Sprachkritik}
%     \begin{itemize}
%         \item Sie zwingen uns, darüber nachzudenken, was eine Sprache "gut" macht
%         \vspace{1cm}
%         \item Sie sind ein Gegenentwurf zu den "bland, bloated multiparadigm languages"
%     \end{itemize}
% \end{frame}

\section{KI und Zukunft}

\begin{frame}{KI und Esolangs - Codegenerierung}
    \begin{itemize}
        \item LLMs (wie ChatGPT) sind gut in Python, aber schlecht in Esolangs
        \vspace{0.5cm}
        \item \textbf{Grund:} Mangelnde Trainingsdaten (Es gibt kaum Brainfuck oder Piet Repositories)
        \vspace{0.5cm}
        \item Experimente zeigen: Generierter Code für Piet oder Shakespeare enthält meist fundamentale Syntaxfehler
    \end{itemize}
\end{frame}

\begin{frame}{KI und Esolangs - Spracherfindung}
    \begin{itemize}
        \item Kann eine KI eine neue Esolang erfinden?
        \vspace{0.5cm}
        \item Versuche ergaben: Sprachen waren oft nicht Turing-vollständig oder plagiierten existierende Ideen (z.B. Musik-Sprachen)
        \vspace{0.5cm}
        \item \textbf{Kreativität und "Witz" fehlen der KI hier noch}
    \end{itemize}
\end{frame}

% \begin{frame}{Software Engineering \& KI}
%     \begin{block}{Erkenntnis}
%         Die Tatsache, dass KI an Esolangs scheitert, zeigt:
%     \end{block}
%     \vspace{0.5cm}
%     \begin{itemize}
%         \item KI wird Software Engineering nicht irrelevant machen
%         \vspace{0.5cm}
%         \item Es gibt Nischen, die menschliches Verständnis und kreative "Unvernunft" erfordern
%     \end{itemize}
% \end{frame}

\section{Fazit}

\begin{frame}{Zusammenfassung}
    \begin{itemize}
        \item Esolangs sind Kunst, Kult und pädagogisches Werkzeug
        \vspace{0.5cm}
        \item Sie stellen sich den Mainstream-Sprachen entgegen
        \vspace{0.5cm}
        \item Sie lehren uns kritisches Denken über Design und Konzepte
    \end{itemize}
    \nocite{esoteric}
\end{frame}

\begin{frame}{Abschließendes Zitat}
    \begin{block}{Alan Perlis (Epigramm 19)}
        "A language that doesn't affect the way you think about programming, is not worth knowing."
        \cite{turingtarpit}
    \end{block}
    \vspace{1cm}
    \Large
    \textbf{Esolangs verändern definitiv, wie wir denken – also sind sie es wert, gekannt zu werden.}
\end{frame}

\begin{frame}
    \frametitle{Vielen Dank!}
    \begin{center}
        \LARGE{Vielen Dank für Eure Aufmerksamkeit!}
    \end{center}
\end{frame}

\section{Quellen}

\begin{frame}[allowframebreaks]
    \frametitle{Quellenverzeichnis}
    \bibliographystyle{amsalpha}
    \bibliography{references}
\end{frame}

\appendix
\section{Anhang I}

\begin{frame}[fragile]{Deep Dive: Shakespeare}
    \begin{itemize}
        \item Titel frei wählbar, wird als Kommentar behandelt
        \item Variablen sind Charaktere und müssen in Shakespeare-Dramen vorkommen (152 insgesamt)
        \item Variablen müssen zunächst deklariert werden (deren Beschreibung ist verpflichtend aber wird ignoriert)
    \end{itemize}
    \begin{lstlisting}[language=C]
The Infamous Hello World Program

Romeo, a young man with a remarkable patience.
Juliet, a likewise young woman of remarkable grace.
Ophelia, a remarkable woman much in dispute with Hamlet.
Hamlet, the flatterer of Andersen Insulting A/S.
    \end{lstlisting}
\end{frame}

\begin{frame}[fragile]{Deep Dive: Shakespeare}
    \begin{itemize}
        \item Folgend werden Akte und Szenen definiert
        \item Diese dienen als Sprungziele für "Goto"-artige Befehle
        \item Zwischen Akten kann nicht gesprungen werden, nur zwischen Szenen innerhalb eines Akts
    \end{itemize}
    \begin{lstlisting}[language=C, firstnumber=8]
Act I: Hamlets insults and flattery.
Scene I: The insulting of Romeo.
    \end{lstlisting}
\end{frame}

\begin{frame}[fragile]{Deep Dive: Shakespeare}
    \begin{itemize}
        \item Es können immer nur zwei Variablen gleichzeitig angesprochen werden
        \item Variablen werden über "Enter" und "Exit" ein- und ausgeführt
    \end{itemize}
    \begin{lstlisting}[language=C, firstnumber=11]
[Enter Hamlet and Romeo]
    \end{lstlisting}
\end{frame}

\begin{frame}[fragile]{Deep Dive: Shakespeare}
    \begin{itemize}
        \item Variablen/Charaktere werden durch Dialoge manipuliert
        \item Die beiden aktiven Charaktere sprechen sich gegenseitig Werte zu
        \item Die Anzahl der Adjektive und das Nomen bestimmen den Wert
        \item Adjektive haben den Wert 2, Nomen haben den Wert 1 (nette Worte) oder -1 (beleidigende Worte)
        \item Die Werte werden multipliziert (big smelly coward = 2*2*-1 = -4)
        \item Im Folgenden entsteht der Wert 2*2*2*2*2*2*-1 = -64
    \end{itemize}
    \begin{lstlisting}[language=C, firstnumber=13]
Hamlet:
You lying stupid fatherless big smelly half-witted coward!
    \end{lstlisting}
\end{frame}

\begin{frame}[fragile]{Deep Dive: Shakespeare}
    \begin{itemize}
        \item Arithmetik durch "as" über Schlüsselwörter "difference", "sum", "product", "quotient"
        \item Im Folgenden entsteht 2*2*2*1 - thyself = 8 - (-64) = 72
    \end{itemize}
    \begin{lstlisting}[language=C, firstnumber=15]
You are as stupid as the difference between a handsome rich brave hero and thyself!
    \end{lstlisting}
\end{frame}

\begin{frame}[fragile]{Deep Dive: Shakespeare}
    \begin{itemize}
        \item Eingabe durch "Open your mind" (als Char) oder "Listen to your heart" (als Zahl)
        \item Ausgabe durch "Speak your mind" (als Char) oder "Open your heart" (als Zahl)
        \item Im Folgenden wird das ASCII-Zeichen 72 = 'H' ausgegeben
    \end{itemize}
    \begin{lstlisting}[language=C, firstnumber=16]
Speak your mind!
    \end{lstlisting}
\end{frame}

\begin{frame}[fragile]{Deep Dive: Shakespeare}
    \begin{itemize}
        \item Aussagenlogik durch "X as good as Y" (X=Y), "X better than Y" (X>Y), "X worse than Y" (X<Y)
        \item Sprünge durch "Let us proceed to Scene N" oder "Let us return to Scene N"
        \item bedingte Sprünge durch "If so" und "If not"
    \end{itemize}
    \begin{lstlisting}[language=C]
Juliet:

 Am I better than you?

Hamlet:

 If so, let us proceed to scene III.
    \end{lstlisting}
\end{frame}

\section{Anhang II}

\begin{frame}[fragile]{Deep Dive: Chef}
    \begin{itemize}
        \item Rezeptname und -beschreibung frei wählbar, werden als Kommentare behandelt
        \item Variablen sind "Zutaten" und müssen zunächst deklariert werden (mit Startwert)
        \item Es gibt trockene (kg, g, lb) und flüssige (l, ml, cup) Zutaten
        \item trockene Zutaten werden als Ganzzahlen, Flüssige als ASCII Characters gespeichert
    \end{itemize}
    \begin{lstlisting}[language=C]
Factorial and Chips.
This recipe calculates the factorial of a number.

Ingredients.
1000 g fish
1 kg Chips
500g breading
    \end{lstlisting}
\end{frame}

\begin{frame}[fragile]{Deep Dive: Chef}
    \begin{itemize}
        \item Es folgt die "Method." Sektion mit Kochanweisungen / Programminstruktionen
        \item "Take X from refrigerator.": X wird ein Wert durch Eingabe zugewiesen (User Input)
    \end{itemize}
    \begin{lstlisting}[language=C, firstnumber=9]
Method.
Take fish from refrigerator.
    \end{lstlisting}
\end{frame}

\begin{frame}[fragile]{Deep Dive: Chef}
    \begin{itemize}
        \item Schüsseln dienen als Stapelspeicher / Stack (LIFO)
        \item "Put X into the mixing bowl.": Wert von X auf den Stack legen
        \item "Add X to mixing bowl.": Wert von X zum obersten Stack-Wert addieren
        \item "Combine X into the mixing bowl.": Wert von X zum obersten Stack-Wert multiplizieren
    \end{itemize}
    \begin{lstlisting}[language=C, firstnumber=11]
Put chips into the mixing bowl.
    \end{lstlisting}
\end{frame}

\begin{frame}[fragile]{Deep Dive: Chef}
    \begin{itemize}
        \item Schleifen durchlaufen solange Variable > 0 ist (solange bis X = 0)
        \item "\_ the X." beginnt eine Schleife
        \item "\_ X until \_." endet eine Schleife
    \end{itemize}
    \begin{lstlisting}[language=C, firstnumber=12]
Bread the fish.
    Combine fish into the mixin bowl.
Coat fish until breaded.
    \end{lstlisting}
\end{frame}

\begin{frame}[fragile]{Deep Dive: Chef}
    \begin{itemize}
        \item "Pour contents of the mixing bowl into the baking dish.": Wert(e) vom Stack nehmen und für Ausgabe vorbereiten
        \item "Serves X.": Ausgabe der letzten X Werte (als Zahl oder Char, je nach Zutatentyp)
    \end{itemize}
    \begin{lstlisting}[language=C, firstnumber=15]
Pour contents of the mixing bowl into the baking dish

Serves 1.
    \end{lstlisting}
\end{frame}

\begin{frame}[fragile]{Deep Dive: Chef}
    \begin{itemize}
        \item "Stir for X minutes": Den obersten Stack-Wert um X nach unten verschieben
    \end{itemize}
    \begin{lstlisting}[]
Input int to ascii converter.

Ingredients.
1 l water

Method.
Carbonate the water.
    Take water from refrigerator.
    Put water into mixing bowl.
    Stir for 999 minutes.
Filter until carbonated.
Pour contents of the mixing bowl into the baking dish.

Serves 1.
    \end{lstlisting}
\end{frame}

\begin{frame}[fragile]{Deep Dive: Chef}
    \begin{itemize}
        \item "Mix the mixing bowl well": Stack random permutieren
        \item "Clean the mixing bowl": Stack leeren
        \item "Liquify contents of the mixing bowl": Alle Werte im Stack zu ASCII-Chars konvertieren
    \end{itemize}
\end{frame}

\section{Anhang III}

\begin{frame}[fragile]{Deep Dive: Whitespace}
    \begin{itemize}
        \item 3 Lexeme: Space (S), Tab (T), Newline (N), alles andere sind Kommentare
        \item Jede Instruktion hat einen Instruction Modification Parameter (IMP)
        \item Nach dem IMP folgt der Befehl selbst
        \item Nach dem Befehl folgt ggf. ein Parameter
    \end{itemize}
    \begin{table}[h]
        \centering
        \small
        \begin{tabular}{|l|l|}
            \hline
            \textbf{IMP} & \textbf{Bereich} \\
            \hline
            S & Stack Manipulation \\
            \hline
            TS & Arithmetik \\
            \hline
            TT & Heap-Zugriff \\
            \hline
            N & Flow Control \\
            \hline
            TN & I/O \\
            \hline
        \end{tabular}
        \caption{Whitespace IMPs}
    \end{table}
\end{frame}

\begin{frame}[fragile]{Deep Dive: Whitespace}
    \begin{itemize}
        \item Zahl Parameter werden als binäre Zahlen kodiert (Space=0, Tab=1)
        \item Zahlen beginnen mit Vorzeichen (Space=+, Tab=-) und enden mit Newline
        \item Dadurch sind Stack Manipulationen möglich
    \end{itemize}
    \begin{table}[h]
        \centering
        \small
        \begin{tabular}{|l|l|l|}
            \hline
            \textbf{Befehl} & \textbf{Parameter} & \textbf{Verhalten} \\
            \hline
            S & Zahl & Push Zahl auf den Stack \\
            \hline
            NS & - & Dupliziere TOS (Top of Stack) \\
            \hline
            NT & - & Tausche TOS und NOS (Top und Next on Stack) \\
            \hline
            NN & - & Entferne TOS \\
            \hline
            TS & Zahl & Kopiere die n-te Zahl vom Stack \\
            \hline
            TN & Zahl & Schiebe die obersten n Zahlen vom Stack \\
            \hline
        \end{tabular}
        \caption{Whitespace Stack Manipulations Befehle}
    \end{table}
\end{frame}

\begin{frame}[fragile]{Deep Dive: Whitespace}
    \begin{itemize}
        \item Arithmetik verwendet die obersten zwei Stack-Werte
        \item Ergebnis von TOS op NOS (z.B. $TOS \div NOS$) wird wieder auf den Stack gelegt
    \end{itemize}
    \begin{table}[h]
        \centering
        \small
        \begin{tabular}{|l|l|l|}
            \hline
            \textbf{Befehl} & \textbf{Parameter} & \textbf{Verhalten} \\
            \hline
            SS & - & Addition \\
            \hline
            ST & - & Subtraktion \\
            \hline
            SN & - & Multiplikation \\
            \hline
            TS & - & Division (Int) \\
            \hline
            TT & - & Modulo \\
            \hline
        \end{tabular}
        \caption{Whitespace Arithmetik Befehle}
    \end{table}
\end{frame}

\begin{frame}[fragile]{Deep Dive: Whitespace}
    \begin{itemize}
        \item Heap verwendet Adressierung über Stack-Werte
        \item TOS = Adresse(, NOS = Wert)
    \end{itemize}
    \begin{table}[h]
        \centering
        \small
        \begin{tabular}{|l|l|l|}
            \hline
            \textbf{Befehl} & \textbf{Parameter} & \textbf{Verhalten} \\
            \hline
            S & - & Speichern \\
            \hline
            T & - & Auslesen \\
            \hline
        \end{tabular}
        \caption{Whitespace Heap Befehle}
    \end{table}
\end{frame}

\begin{frame}[fragile]{Deep Dive: Whitespace}
    \begin{itemize}
        \item Flow Control steuert den Programmfluss
        \item Labels sind beliebige Kombinationen von Space und Tab, enden mit Newline
    \end{itemize}
    \begin{table}[h]
        \centering
        \small
        \begin{tabular}{|l|l|l|}
            \hline
            \textbf{Befehl} & \textbf{Parameter} & \textbf{Verhalten} \\
            \hline
            SS & Label & Erstelelt eine Sprungmarke \\
            \hline
            ST & Label & Ruft Subroutine auf \\
            \hline
            SN & Label & Sprung zu Label (ohne Bedingung) \\
            \hline
            TS & Label & Sprung zu Label (wenn TOS = 0) \\
            \hline
            TT & Label & Sprung zu Label (wenn TOS < 0) \\
            \hline
            TN & - & Beende Subroutine, übergib Kontrolle auf Aufrufenden \\
            \hline
            NN & - & Beendet ds Programm \\
            \hline
        \end{tabular}
        \caption{Whitespace Flow Control Befehle}
    \end{table}
\end{frame}

\begin{frame}[fragile]{Deep Dive: Whitespace}
    \begin{itemize}
        \item Die Eingabe erfolgt über die Addressierung im Heap (TOS = Adresse)
        \item Die Ausgabe verwendet den obersten Stack-Wert
        \item TOS wird immer entfernt nach Ein-/Ausgabe
    \end{itemize}
    \begin{table}[h]
        \centering
        \small
        \begin{tabular}{|l|l|l|}
            \hline
            \textbf{Befehl} & \textbf{Parameter} & \textbf{Verhalten} \\
            \hline
            TS & - & Speichert Eingabe als ASCII an Heap Adresse (TOS) \\
            \hline
            TT & - & Speichert Eingabe als Zahl an Heap Adresse (TOS) \\
            \hline
            SS & - & Ausgabe von TOS als ASCII \\
            \hline
            ST & - & Ausgabe von TOS als Zahl \\
            \hline
        \end{tabular}
        \caption{Whitespace I/O Befehle}
    \end{table}
\end{frame}

\section{Anhang IV}

\begin{frame}[fragile]{Deep Dive: Piet}
    \begin{itemize}
        \item Stack basiert (LIFO)
        \item 18 Farben (6 Farbtöne x 3 Helligkeitsstufen) + Schwarz (Blocker) + Weiß (kein Befehl)
    \end{itemize}
    \begin{figure}[H]
        \centering
        \includegraphics[width=\textwidth]{Piet_Colors.png}
        \caption{Farben in Piet}
    \end{figure}
\end{frame}

\begin{frame}[fragile]{Deep Dive: Piet}
    \begin{itemize}
        \item Direction Pointer (DP) zeigt die aktuelle Bewegungsrichtung an (Rechts, Unten, Links, Oben)
        \item DP beginnt oben links und zeigt nach rechts
        \item Codel Chooser (CC) bestimmt, ob bei Blockaden / Entscheidungsproblemen im Uhrzeigersinn oder gegen den Uhrzeigersinn gedreht wird
        \item CC beginnt gegen den Uhrzeigersinn
        \item Bei Blockaden (Schwarz oder Rand) wird zuerst der CC gewechselt, dann der DP gedreht
        \item Nach 8 erfolglosen Versuchen wird das Programm beendet
    \end{itemize}
\end{frame}

\begin{frame}[fragile]{Deep Dive: Piet}
    \begin{itemize}
        \item Befehle/Instruktionen entstehen durch Farbwechsel oder Helligkeitswechsel
        \item Die Wechsel sind kreisförmig angeordnet, also kommt Rot nach Magenta, und Light nach Dark (1 darker ist identisch zu 2 lighter)
    \end{itemize}

    \begin{table}[H]
        \centering
        \small
        \begin{tabular}{|l|l|l|l|}
            \hline
             & \multicolumn{3}{c|}{\textbf{Lightness change}} \\
            \hline
            \textbf{Hue change} & \textbf{No change} & \textbf{1 darker/2 lighter} & \textbf{2 darker/1 lighter} \\
            \hline
            No change & N/A & Push & Pop \\
            \hline
            1 step & Add & Subtract & Multiply \\
            \hline
            2 steps & Divide & Modulo & Not \\
            \hline
            3 steps & Greater & Pointer & Switch \\
            \hline
            4 steps & Duplicate & Roll & Input num \\
            \hline
            5 steps & Input char & Output num & Output char \\
            \hline
        \end{tabular}
    \end{table}
\end{frame}

\begin{frame}[fragile]{Deep Dive: Piet}
    \begin{itemize}
        \item Wenn eine Instruktion einen Wert benötigt (z.B. Push), wird die Anzahl der Kacheln im vorherigen Farbbereich als Wert verwendet
        \item Ein Farbbereich ist eine zusammenhängende Fläche gleicher Farbe (nur kardinale Richtung zählt, also nicht diagonal zusammenhängend)
        \item Das Folgende Beispiel zeigt die Ausgabe der Zahl 12 nach der Addition von 6 und 6
    \end{itemize}
    \begin{figure}[H]
        \centering
        \includegraphics[width=0.45\textwidth]{Piet_Bsp.png}
        \caption{Beispielprogramm in Piet}
    \end{figure}
\end{frame}

\section{Anhang V}

\begin{frame}[fragile]{Deep Dive: Brainfuck}
    \begin{itemize}
        \item Direkter Zugriff auf Speicherzellen (Array von Bytes)
        \item Alle Zellen sind initialisiert auf 0
        \item Pointer zeigt auf die aktuelle Zelle
    \end{itemize}

\end{frame}

\begin{frame}[fragile]{Deep Dive: Brainfuck}
    \begin{itemize}
        \item 8 Instruktionen
    \end{itemize}
    \begin{table}[h]
        \centering
        \small
        \begin{tabular}{|l|l|l|}
            \hline
            \textbf{Instruktion} & \textbf{Verhalten} \\
            \hline
            > & Bewege den Pointer nach rechts \\
            \hline
            < & Bewege den Pointer nach links \\
            \hline
            + & Erhöhe den Wert der aktuellen Zelle um 1\\
            \hline
            - & Verringere den Wert der aktuellen Zelle um 1\\
            \hline
            . & Gib den ASCII-Wert der aktuellen Zelle aus \\
            \hline
            , & Lese ein ASCII-Zeichen ein und speichere es in der aktuellen Zelle \\
            \hline
            [ & Springe zum entsprechenden ] wenn der Wert der aktuellen Zelle 0 ist \\
            \hline
            ] & Springe zum entsprechenden [ wenn der Wert der aktuellen Zelle nicht 0 ist \\
            \hline
        \end{tabular}
        \caption{Brainfuck Befehle}
    \end{table}
\end{frame}

\begin{frame}[fragile]{Deep Dive: Brainfuck}
    \begin{lstlisting}[basicstyle=\ttfamily\tiny]
+++++ +++               Set Cell #0 to 8
[
    >++++               Add 4 to Cell #1; this will always set Cell #1 to 4
    [                   as the cell will be cleared by the loop
        >++             Add 4*2 to Cell #2
        >+++            Add 4*3 to Cell #3
        >+++            Add 4*3 to Cell #4
        >+              Add 4 to Cell #5
        <<<<-           Decrement the loop counter in Cell #1
    ]                   Loop till Cell #1 is zero
    >+                  Add 1 to Cell #2
    >+                  Add 1 to Cell #3
    >-                  Subtract 1 from Cell #4
    >>+                 Add 1 to Cell #6
    [<]                 Move back to the first zero cell you find; this will
                        be Cell #1 which was cleared by the previous loop
    <-                  Decrement the loop Counter in Cell #0
]                       Loop till Cell #0 is zero

The result of this is:
Cell No :   0   1   2   3   4   5   6
Contents:   0   0  72 104  88  32   8
Pointer :   ^
    \end{lstlisting}
\end{frame}

\begin{frame}[fragile]{Deep Dive: Brainfuck}
    \begin{lstlisting}[basicstyle=\ttfamily\tiny, firstnumber=24]
>>.                     Cell #2 has value 72 which is 'H'
>---.                   Subtract 3 from Cell #3 to get 101 which is 'e'
+++++ ++..+++.          Likewise for 'llo' from Cell #3
>>.                     Cell #5 is 32 for the space
<-.                     Subtract 1 from Cell #4 for 87 to give a 'W'
<.                      Cell #3 was set to 'o' from the end of 'Hello'
+++.----- -.----- ---.  Cell #3 for 'rl' and 'd'
>>+.                    Add 1 to Cell #5 gives us an exclamation point
>++.                    And finally a newline from Cell #6
    \end{lstlisting}
\end{frame}

\end{document}
