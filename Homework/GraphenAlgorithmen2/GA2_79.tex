\documentclass{report}

%%%%%%%%%%%%%%%%%%%%%%%%%%%%%%%%%
% PACKAGE IMPORTS
%%%%%%%%%%%%%%%%%%%%%%%%%%%%%%%%%


\usepackage{graphicx, float}
\graphicspath{{res/}}
\usepackage{tikz}     % tableofcontents
\usepackage{titletoc} % tableofcontents
\usepackage{mathptmx}
\usepackage[skip=0.8em, indent=0pt]{parskip}

\usepackage[hidelinks]{hyperref}


%%%%%%%%%%%%%%%%%%%%%%%%%%%%%%
% MODIFIED COMMANDS
%%%%%%%%%%%%%%%%%%%%%%%%%%%%%%


\newcommand{\bcite}[1]{\textbf{\cite{#1}}}
\renewcommand{\footnoterule}{\vfill\kern -3pt \hrule width 0.4\columnwidth \kern 2.6pt}

\renewcommand{\abstractname}{Zusammenfassung}
\renewcommand{\contentsname}{Inhalt}
\newcommand{\replacechaptername}{Kapitel}
\renewcommand{\chaptername}{\replacechaptername}
\renewcommand{\bibname}{Quellenverzeichnis}
\newcommand{\replacepagename}{Seite}


%%%%%%%%%%%%%%%%%%%%%%%%%%%%%%
% SELF MADE COLORS
%%%%%%%%%%%%%%%%%%%%%%%%%%%%%%


\definecolor{doc}{RGB}{0,60,110}


%%%%%%%%%%%%%%%%%%%%%%%%%%%%%%%%%%%%%%%%%%%
% TABLE OF CONTENTS
%%%%%%%%%%%%%%%%%%%%%%%%%%%%%%%%%%%%%%%%%%%


\contentsmargin{0cm}
\titlecontents{chapter}[3.7pc]
{\addvspace{30pt}
	\begin{tikzpicture}[remember picture, overlay]
		\draw[fill=doc!60,draw=doc!60] (-7,-.1) rectangle (-0.9,.5);
		\pgftext[left,x=-3.5cm,y=0.2cm]{\color{white}\Large\sc\bfseries \replacechaptername\ \thecontentslabel};
	\end{tikzpicture}\color{doc!60}\large\sc\bfseries}
{}
{}
{\;\titlerule\;\large\sc\bfseries \replacepagename\space\thecontentspage
	\begin{tikzpicture}[remember picture, overlay]
		\draw[fill=doc!60,draw=doc!60] (2pt,0) rectangle (4,0.1pt);
	\end{tikzpicture}}
\titlecontents{section}[3.7pc]
{\addvspace{2pt}}
{\contentslabel[\thecontentslabel]{2pc}}
{}
{\hfill\small \thecontentspage}
[]
\titlecontents{subsection}[4.7pc]
{\addvspace{-1pt}\small}
{}
{}
{\hfill\small \thecontentspage}
[]
\titlecontents{bibliography}[3.7pc]
{\addvspace{30pt}
	\color{doc!60}\large\sc\bfseries}
{}
{}
{\;\titlerule\;\large\sc\bfseries \replacepagename\space\thecontentspage
	\begin{tikzpicture}[remember picture, overlay]
		\draw[fill=doc!60,draw=doc!60] (2pt,0) rectangle (4,0.1pt);
	\end{tikzpicture}}

\makeatletter
\renewcommand{\tableofcontents}{
	\chapter*{
	  \vspace*{-20\p@}
	  \begin{tikzpicture}[remember picture, overlay]
		  \pgftext[right,x=15cm,y=0.2cm]{\color{doc!60}\Huge\sc\bfseries \contentsname};
		  \draw[fill=doc!60,draw=doc!60] (13.5,-.75) rectangle (20,1);
		  \clip (13.5,-.75) rectangle (20,1);
		  \pgftext[right,x=15cm,y=0.2cm]{\color{white}\Huge\sc\bfseries \contentsname};
	  \end{tikzpicture}}
	\@starttoc{toc}}
\makeatother


\input{macros}
\input{letterfonts}

\newtheorem{theorem}{Theorem}

\title{\Huge{Graphenalgorithmen 2}\\Aufgaben}
\author{\huge{Silas Alexander Kraume}\\sikra111}
\date{}

\begin{document}

\maketitle
\newpage% or \cleardoublepage

\section{Aufgabe 79}

\begin{theorem}
\leavevmode\newline
Für jeden chordalen Graphen $G = \left(V, E\right)$ gilt
\[
\kappa\left(G\right) = \alpha\left(G\right)
\]
wobei
$\kappa\left(G\right)$ die minimale Anzahl von Cliquen in einer Cliquenüberdeckung und
$\alpha\left(G\right)$ die Größe einer maximalen unabhängigen Menge ist.
\end{theorem}

\begin{proof}
\leavevmode\newline
\noindent\textbf{Ungleichung 1 ($\alpha\left(G\right)\leq\kappa\left(G\right)$).}
\newline
Dies gilt für alle Graphen, nicht nur chordale Graphen.
Sei $\mathcal{C} = \{C_1, C_2, \ldots, C_k\}$
eine beliebige Cliquenüberdeckung von $G$.
Eine unabhängige Menge $I$ kann höchstens einen Knoten aus jeder Clique $C_i$ enthalten, da sonst zwei Knoten der unabhängigen Menge adjazent wären.
Also gilt für jede unabhängige Menge $I \subseteq V$ in $G$:
\[
    |I| \leq k = |\mathcal{C}|
\]
Da $|\mathcal{C}| = \kappa\left(G\right)$ minimal ist, folgt
\[
    \alpha\left(G\right) \leq \kappa\left(G\right)
\]

\medskip
\noindent\textbf{Ungleichung 2 ($\alpha\left(G\right)\geq\kappa\left(G\right)$).}
\newline
Es gilt: \begin{quote}
    Ein Graph $G$ ist genau dann chordal, wenn er eine perfekte Eliminationsordnung besitzt.
\end{quote}
Das bedeutet, es existiert eine Ordnung der Knoten $v_1, v_2, \ldots, v_n \in V$ so, dass für jeden Knoten $v_i$ die späteren Nachbarn
\[
    N^+ \left(v_i\right) = \{ v_j \mid j > i, \{v_i, v_j\} \in E \}
\]
eine Clique bilden, also alle $v_i$ simplizial sind.

Mithilfe der Eliminationsordnung lässt sich eine Cliquenüberdeckung konstruieren. Wir definieren für jeden Knoten $v_i$ die Clique
\[
    C_i = \{ v_i \} \cup N^+ \left(v_i\right).
\]

Die Menge der Cliquen $\{C_i \mid i \in \mathcal{I}\}$ mit der Menge $\mathcal{I} = \{i \mid v_i \notin C_j, j < i\}$ aller Indizes bildet eine Cliquenüberdeckung von $G$,
da jeder Knoten $v_i$ entweder in seiner eigenen Clique $C_i$ oder in einer vorherigen Clique $C_j$ mit $j < i$ enthalten ist.

Sei
\[
    \mathcal{S} = \{v_i \mid i \in \mathcal{I}\}
\]
eine unabhängige Menge in $G$ mit den entsprechenden $v_i$ als Repräsentanten ihrer jeweiligen Clique $C_i$.
$\mathcal{S}$ ist unabhängig, denn
angenommen, es gäbe zwei Knoten $v_i, v_j \in \mathcal{S}$ mit $i<j$ und $\{v_i,v_j\}\in E$.
Dann gilt $v_j \in N^+\left(v_i\right)$ und somit $v_j \in C_i$, was im Widerspruch dazu steht, dass $j \in \mathcal{I}$.
Also existiert keine Kante zwischen zwei Knoten aus $\mathcal{S}$.
\newline

Da $\mathcal{S}$ unabhängig ist, gilt:
\[
    |\mathcal{S}| \leq \alpha\left(G\right).
\]

Die Cliquen $\{C_i \mid v_i \in \mathcal{S}\}$ bilden eine Cliquenüberdeckung der Größe $|\mathcal{S}|$.
Somit gilt:
\[
    \kappa\left(G\right) \leq |\mathcal{S}| \leq \alpha\left(G\right).
\]

\noindent\textbf{Beweisschluss ($\alpha\left(G\right)=\kappa\left(G\right)$).}
\newline
Aus den beiden Ungleichungen folgt:
\[
    \alpha\left(G\right) \leq \kappa\left(G\right) \leq \alpha\left(G\right)
\]
Also gilt:
\[
    \alpha\left(G\right) = \kappa\left(G\right)
\]

\end{proof}

\end{document}
