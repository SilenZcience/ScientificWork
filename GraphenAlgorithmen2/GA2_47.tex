\documentclass{report}

%%%%%%%%%%%%%%%%%%%%%%%%%%%%%%%%%
% PACKAGE IMPORTS
%%%%%%%%%%%%%%%%%%%%%%%%%%%%%%%%%


\usepackage{graphicx, float}
\graphicspath{{res/}}
\usepackage{tikz}     % tableofcontents
\usepackage{titletoc} % tableofcontents
\usepackage{mathptmx}
\usepackage[skip=0.8em, indent=0pt]{parskip}

\usepackage[hidelinks]{hyperref}


%%%%%%%%%%%%%%%%%%%%%%%%%%%%%%
% MODIFIED COMMANDS
%%%%%%%%%%%%%%%%%%%%%%%%%%%%%%


\newcommand{\bcite}[1]{\textbf{\cite{#1}}}
\renewcommand{\footnoterule}{\vfill\kern -3pt \hrule width 0.4\columnwidth \kern 2.6pt}

\renewcommand{\abstractname}{Zusammenfassung}
\renewcommand{\contentsname}{Inhalt}
\newcommand{\replacechaptername}{Kapitel}
\renewcommand{\chaptername}{\replacechaptername}
\renewcommand{\bibname}{Quellenverzeichnis}
\newcommand{\replacepagename}{Seite}


%%%%%%%%%%%%%%%%%%%%%%%%%%%%%%
% SELF MADE COLORS
%%%%%%%%%%%%%%%%%%%%%%%%%%%%%%


\definecolor{doc}{RGB}{0,60,110}


%%%%%%%%%%%%%%%%%%%%%%%%%%%%%%%%%%%%%%%%%%%
% TABLE OF CONTENTS
%%%%%%%%%%%%%%%%%%%%%%%%%%%%%%%%%%%%%%%%%%%


\contentsmargin{0cm}
\titlecontents{chapter}[3.7pc]
{\addvspace{30pt}
	\begin{tikzpicture}[remember picture, overlay]
		\draw[fill=doc!60,draw=doc!60] (-7,-.1) rectangle (-0.9,.5);
		\pgftext[left,x=-3.5cm,y=0.2cm]{\color{white}\Large\sc\bfseries \replacechaptername\ \thecontentslabel};
	\end{tikzpicture}\color{doc!60}\large\sc\bfseries}
{}
{}
{\;\titlerule\;\large\sc\bfseries \replacepagename\space\thecontentspage
	\begin{tikzpicture}[remember picture, overlay]
		\draw[fill=doc!60,draw=doc!60] (2pt,0) rectangle (4,0.1pt);
	\end{tikzpicture}}
\titlecontents{section}[3.7pc]
{\addvspace{2pt}}
{\contentslabel[\thecontentslabel]{2pc}}
{}
{\hfill\small \thecontentspage}
[]
\titlecontents{subsection}[4.7pc]
{\addvspace{-1pt}\small}
{}
{}
{\hfill\small \thecontentspage}
[]
\titlecontents{bibliography}[3.7pc]
{\addvspace{30pt}
	\color{doc!60}\large\sc\bfseries}
{}
{}
{\;\titlerule\;\large\sc\bfseries \replacepagename\space\thecontentspage
	\begin{tikzpicture}[remember picture, overlay]
		\draw[fill=doc!60,draw=doc!60] (2pt,0) rectangle (4,0.1pt);
	\end{tikzpicture}}

\makeatletter
\renewcommand{\tableofcontents}{
	\chapter*{
	  \vspace*{-20\p@}
	  \begin{tikzpicture}[remember picture, overlay]
		  \pgftext[right,x=15cm,y=0.2cm]{\color{doc!60}\Huge\sc\bfseries \contentsname};
		  \draw[fill=doc!60,draw=doc!60] (13.5,-.75) rectangle (20,1);
		  \clip (13.5,-.75) rectangle (20,1);
		  \pgftext[right,x=15cm,y=0.2cm]{\color{white}\Huge\sc\bfseries \contentsname};
	  \end{tikzpicture}}
	\@starttoc{toc}}
\makeatother


\input{macros}
\input{letterfonts}

\title{\Huge{Graphenalgorithmen 2}\\Aufgaben}
\author{\huge{Silas Alexander Kraume}\\sikra111}
\date{}

\begin{document}

\maketitle
\newpage% or \cleardoublepage

\section{Aufgabe 47}

\begin{algorithm}[H]
\KwIn{2-reduzierbarer Graph $G = \left(V, E\right)$}
\KwOut{\textbf{true} wenn Hamilton-Kreis existiert, sonst \textbf{false}}

\SetKwFunction{FHamilton}{CheckHamiltonCycleIn2ReducibleGraph}
\SetKwProg{Fn}{Function}{:}{}
\Fn{\FHamilton{$G$}}{
\While{$|\text{V}| > 3$}{
    $u \gets$ node in $V$ with $\deg_{G}\left(u\right) \leq 2$\;

    \If{$\deg_{G}\left(u\right) < 2$}{
        \Return \textbf{false}\;
    }

    Seien $v, w$ die beiden Nachbarn von $u$ in $G$\;

    \tcp{Führe 2-Reduktion durch:}
    $E \gets E \setminus \{(u,v)\}$\;
    $E \gets E \setminus \{(u,w)\}$\;
    $E \gets E \cup \{(v,w)\}$
    \tcp{Insofern nicht bereits vorhanden}
    $V \gets V \setminus \{u\}$\;
}
\If{$G \neq K_3$}{
    \Return \textbf{false}\;
}

\Return \textbf{true}\;
}
\caption{Algorithmus zum Test der Existenz eines Hamilton-Kreises in einem 2-reduzierbaren Graphen}
\end{algorithm}

\subsection*{Beweis der Korrektheit}

\begin{proof}
Wir zeigen durch Induktion über die Anzahl der Knoten, dass der Algorithmus korrekt entscheidet, ob ein 2-reduzierbarer Graph $G$ einen Hamilton-Kreis besitzt.

\textbf{Invariante:} In jedem Reduktionsschritt gilt: $G$ hat einen Hamilton-Kreis $\Leftrightarrow$ $G'$ (reduzierter Graph) hat einen Hamilton-Kreis.

\textbf{Induktionsanfang:} Für $|V| = 3$ gilt: $G$ hat einen Hamilton-Kreis $\Leftrightarrow$ $G = K_3$ ($K_3$ ist der kleinstmögliche Hamilton-Kreis).

\textbf{Induktionsschritt:} Angenommen, die Invariante gilt für Graphen mit $k$ Knoten. Betrachte Graph $G$ mit $k{+}1$ Knoten. Sei $u$ ein Knoten mit $\deg\left(u\right) = 2$ und Nachbarn $v, w$. Sei $G'$ der Graph nach 2-Reduktion von $u$.

\textit{($\Rightarrow$):} Hat $G$ einen Hamilton-Kreis $H$, so verwendet $H$ beide Kanten $\{u,v\}$ und $\{u,w\}$ (da $u$ Grad 2 in $H$ haben muss). Nach Entfernung von $u$ und Hinzufügen von $\{v,w\}$ entsteht
\[H' = H \setminus \{\{u,v\}, \{u,w\}\} \cup \{\{v,w\}\},\]
ein Hamilton-Kreis in $G'$.

\textit{($\Leftarrow$):} Hat $G'$ einen Hamilton-Kreis $H'$ mit Kante $\{v,w\}$, so ersetzen wir $\{v,w\}$ durch den Pfad $v - u - w$ und erhalten
\[H = H' \setminus \{\{v,w\}\} \cup \{\{u,v\}, \{u,w\}\},\]
einen Hamilton-Kreis in $G$.

Somit gilt die Invariante auch für Graphen mit $k{+}1$ Knoten.

\textbf{Korrektheit der Abbruchbedingungen:}

Falls $\deg\left(u\right) < 2$: Knoten mit Grad $< 2$ können nicht in einem Hamilton-Kreis liegen $\Rightarrow$ kein Hamilton-Kreis existiert.

Falls $|V| < 3$: Ein Hamilton-Kreis benötigt mindestens 3 Knoten $\Rightarrow$ kein Hamilton-Kreis existiert / Die Schleife wird nie durchlaufen und der Algorithmus gibt \textbf{false} zurück, da $G \neq K_3$.


Durch vollständige Induktion folgt die Korrektheit des Algorithmus.
\end{proof}

\subsection*{Laufzeitanalyse}

\begin{itemize}
    \item Jede Iteration entfernt genau 1 Knoten $\Rightarrow |V| - 3$ Iterationen bis Abbruchbedingung $|V| = 3$
    \item Jede Iteration: $O\left(1\right)$ zum Finden eines Knotens mit $\deg \leq 2$, $O\left(1\right)$ für 2-Reduktionsschritt
    \item \textbf{Gesamtzeitkomplexität: $O\left(|V|\right)$}
\end{itemize}

\end{document}
