\documentclass{beamer}
\usepackage[utf8]{inputenc} 
\usepackage[T1]{fontenc}
\usepackage[upright]{fourier} 
\usepackage[usenames,dvipsnames]{xcolor}
\usepackage{tkz-kiviat,numprint,fullpage} 
\usepackage{amsfonts,amsmath,oldgerm}
\usetikzlibrary{arrows}
\thispagestyle{empty}
\usetheme{sintef}

\usefonttheme[onlymath]{serif}
\newcommand{\hrefcol}[2]{\textcolor{cyan}{\href{#1}{#2}}}
\newcommand{\testcolor}[1]{\colorbox{#1}{\textcolor{#1}{test}}~\texttt{#1}}

\titlebackground*{res/background}

\title{Social Engineering}
\subtitle{Bachelor Seminar - Billion Dollar Heist}
\author{Silas A. Kraume}
% \IDnumber{1234567}
\date{}

\begin{document}
\maketitle

\begin{frame}
    This template is a based on \hrefcol{https://github.com/TOB-KNPOB/Beamer-LaTeX-Themes}{Beamer-LaTeX-Themes} and its modified by ARCW
    \vspace{\baselineskip}
    In the following you find a brief introduction on how to use \LaTeX\ and the beamer package to prepare slides, based on the one written by \hrefcol{mailto:federico.zenith@sintef.no}{Federico Zenith} for \hrefcol{https://www.overleaf.com/latex/templates/sintef-presentation/jhbhdffczpnx}{SINTEF Presentation}
\end{frame}

\section{Introduction}

\begin{frame}{Beamer for SINTEF slides}
    \begin{itemize}
        \item We assume you can use \LaTeX; if you cannot,
              \hrefcol{http://en.wikibooks.org/wiki/LaTeX/}{you can learn it here}
        \item Beamer is one of the most popular and powerful document
              classes for presentations in \LaTeX
        \item Beamer has also a detailed
              \hrefcol{http://www.ctan.org/tex-archive/macros/latex/contrib/beamer/doc/beameruserguide.pdf}{user
                  manual}
        \item Here we will present only the most basic features to get you up to speed
    \end{itemize}
\end{frame}

\begin{frame}{Beamer vs. PowerPoint}
    Compared to PowerPoint, using \LaTeX\ is better because:
    \begin{itemize}
        \item It is not What-You-See-Is-What-You-Get, but
              What-You-\emph{Mean}-Is-What-You-Get:\\
              you write the content, the computer does the typesetting
        \item Produces a \texttt{pdf}: no problems with fonts, formulas,
              program versions
        \item Easier to keep consistent style, fonts, highlighting, etc.
        \item Math typesetting in \TeX\ is the best:
              \begin{equation*}
                  \mathrm{i}\,\hslash\frac{\partial}{\partial t} \Psi(\mathbf{r},t) =
                  -\frac{\hslash^2}{2\,m}\nabla^2\Psi(\mathbf{r},t)
                  + V(\mathbf{r})\Psi(\mathbf{r},t)
              \end{equation*}

    \end{itemize}
\end{frame}

\begin{frame}{THIS IS ME}
    Compared to PowerPoint, using \LaTeX\ is better because:
    \begin{center}
        \begin{tikzpicture}
            \tkzKiviatDiagram[scale=1.5,label distance=1cm,
                radial  = 1,
                gap     = 0.3,
                lattice = 5, label space = 0.7]{        Offenheit,Gewissenhaftigkeit,Extraversion,Verträglichkeit,Neurotizismus}
            \tkzKiviatLine[thick,color=blue,mark=none,
                fill=blue!20,opacity=.5](4.25,4.65,2.8,4.35,2.8) % 85, 93, 56, 87, 56
            \tkzKiviatGrad[unity=20](1)
        \end{tikzpicture}
    \end{center}

\end{frame}

% \begin{frame}[fragile]{Getting Started}
%     \framesubtitle{Selecting the SINTEF Theme}
%     To start working with \texttt{sintefbeamer}, start a \LaTeX\ document with the
%     preamble:
%     \begin{block}{Minimum SINTEF Beamer Document}
%         \verb|\documentclass{beamer}|\\
%         \verb|\usetheme{sintef}|\\
%         \verb|\begin{document}|\\
%         \verb|\begin{frame}{Hello, world!}|\\
%         \verb|\end{frame}|\\
%         \verb|\end{document}|\\
%     \end{block}
% \end{frame}

\begin{frame}[fragile]{Title page}
    To set a typical title page, you call some commands in the preamble:
    \begin{block}{The Commands for the Title Page}
        \begin{verbatim}
    \title{Sample Title}
    \subtitle{Sample subtitle}
    \author{First Author, Second Author}
    \date{\today} % Can also be (ab)used for conference name &c.
    \end{verbatim}
    \end{block}
    You can then write out the title page with \verb|\maketitle|.

    To set a \textbf{background image} use the \verb|\titlebackground| command
    before \verb|\maketitle|; its only argument is the name (or path) of a graphic
    file.

    If you use the \textbf{starred version} \verb|\titlebackground*|, the image
    will be clipped to a split view on the right side of the title slide.

\end{frame}

\begin{frame}[fragile]{Writing a Simple Slide}
    \framesubtitle{It's really easy!}
    \begin{itemize}[<+->]
        \item A typical slide has bulleted lists
        \item These can be uncovered in sequence
    \end{itemize}
    \begin{block}{Code for a Page with an Itemised List}<+->
        \begin{verbatim}
    \begin{frame}{Writing a Simple Slide}
      \framesubtitle{It's really easy!}
      \begin{itemize}[<+->]
        \item A typical slide has bulleted lists
        \item These can be uncovered in sequence
      \end{itemize}\end{frame}
    \end{verbatim}
    \end{block}
\end{frame}

\section{Personalization}

% \footlinecolor{sintefyellow}
% \begin{frame}[fragile]{Changing Slide Style}
%     \begin{itemize}
%         \item You can select the white or \textit{maincolor} \textbf{slide style} \emph{in the
%                   preamble} with \verb|\themecolor{white}| (default) or \verb|\themecolor{main}|
%               \begin{itemize}
%                   \item You should \emph{not} change these within the document: Beamer does
%                         not like it
%                   \item If you \emph{really} must, you may have to add
%                         \verb|\usebeamercolor[fg]{normal text}| in the slide
%               \end{itemize}
%         \item You can change the \textbf{footline colour} with
%               \verb|\footlinecolor{color}|
%               \begin{itemize}
%                   \item Place the command \emph{before} a new \verb|frame|
%                   \item There are four ``official'' colors:
%                         \testcolor{maincolor}, \testcolor{sintefyellow},
%                         \testcolor{sintefgreen}, \testcolor{sintefdarkgreen}
%                   \item Default is no footline; you can restore it with
%                         \verb|\footlinecolor{}|
%                   \item Others may work, but no guarantees!
%                   \item Should \emph{not} be used with the \verb|maincolor| theme!
%               \end{itemize}
%     \end{itemize}
% \end{frame}

% \begin{frame}[fragile]{Blocks}
%     \begin{columns}
%         \begin{column}{0.3\textwidth}
%             \begin{block}{Standard Blocks}
%                 These have a color coordinated with the footline (and grey in the blue theme)
%                 \begin{verbatim}
%     \begin{block}{title}
%     content...
%     \end{block}
%     \end{verbatim}
%             \end{block}
%         \end{column}
%         \begin{column}{0.7\textwidth}
%             \begin{colorblock}[black]{sinteflightgreen}{Colour Blocks}
%                 Similar to the ones on the left, but you pick the colour. Text will be white by
%                 default, but you may set it with an optional argument.
%                 \small
%                 \begin{verbatim}
%     \begin{colorblock}[black]{sinteflightgreen}{title}
%     content...
%     \end{colorblock}
%     \end{verbatim}
%             \end{colorblock}
%             The ``official'' colours of colour blocks are: \testcolor{sinteflilla},
%             \testcolor{maincolor}, \testcolor{sintefdarkgreen}, and
%             \testcolor{sintefyellow}.
%         \end{column}
%     \end{columns}
% \end{frame}

% \footlinecolor{}
% \begin{frame}[fragile]{Using Colours}
%     \begin{itemize}[<alert@2>]
%         \item You can use colours with the
%               \verb|\textcolor{<color name>}{text}| command
%         \item The colours are defined in the \texttt{sintefcolor} package:
%               \begin{itemize}
%                   \item Primary colours: \testcolor{maincolor} and its sidekick
%                         \testcolor{sintefgrey}
%                   \item Three shades of green: \testcolor{sinteflightgreen},
%                         \testcolor{sintefgreen}, \testcolor{sintefdarkgreen}
%                   \item Additional colours: \testcolor{sintefyellow}, \testcolor{sintefred},
%                         \testcolor{sinteflilla}
%                         \begin{itemize}
%                             \item These may be shaded---see the \verb|sintefcolor| documentation or
%                                   the \hrefcol{https://sintef.sharepoint.com/sites/stottetjenester/%
%                                       kommunikasjon/grafisk-profil-new/Sider/default.aspx}{SINTEF profile
%                                       manual}
%                         \end{itemize}
%               \end{itemize}
%         \item Do \emph{not} abuse colours: \verb|\emph{}| is usually enough
%         \item Use \verb|\alert{}| to bring the \alert<2->{focus} somewhere
%         \item<2- | alert@2> If you highlight too much, you don't highlight at all!
%     \end{itemize}
% \end{frame}

% \begin{frame}[fragile]{Adding images}
%     \begin{columns}
%         \begin{column}{0.7\textwidth}
%             Adding images works like in normal \LaTeX:
%             \begin{block}{Code for Adding Images}
%                 \begin{verbatim}
%     \usepackage{graphicx}
%     % ...
%     \includegraphics[width=\textwidth]
%     {assets/logo_RGB} %hhu
%     \end{verbatim}
%             \end{block}
%         \end{column}
%         \begin{column}{0.3\textwidth}
%             \includegraphics[width=\textwidth]
%             {assets/logo_RGB} %hhu
%         \end{column}
%     \end{columns}
% \end{frame}

\begin{frame}[fragile]{Splitting in Columns}
    Splitting the page is easy and common;
    typically, one side has a picture and the other text:
    \begin{columns}
        \begin{column}{0.6\textwidth}
            This is the first column
        \end{column}
        \begin{column}{0.3\textwidth}
            And this the second
        \end{column}
    \end{columns}
    \begin{block}{Column Code}
        \begin{verbatim}
    \begin{columns}
        \begin{column}{0.6\textwidth}
            This is the first column
        \end{column}
        \begin{column}{0.3\textwidth}
            And this the second
        \end{column}
        % There could be more!
    \end{columns}
    \end{verbatim}
    \end{block}
\end{frame}


% \footlinecolor{}
\begin{frame}
    \frametitle{Fonts}
    \begin{itemize}
        \item The paramount task of fonts is being readable
        \item There are good ones...
              \begin{itemize}
                  \item {\textrm{Use serif fonts only with high-definition projectors}}
                  \item {\textsf{Use sans-serif fonts otherwise (or if you simply prefer
                            them)}}
              \end{itemize}
        \item ... and not so good ones:
              \begin{itemize}
                  \item {\texttt{Never use monospace for normal text}}
                  \item {\frakfamily Gothic, calligraphic or weird fonts: should always: be
                        avoided}
              \end{itemize}
    \end{itemize}
\end{frame}

\begin{frame}[fragile]{Look}
    \begin{itemize}
        \item To insert a final slide with the title and final thanks, use \verb|\backmatter|.
              \begin{itemize}
                  \item The title also appears in footlines along with the author name, you can change this text with \verb|\footlinepayoff|
                  \item You can remove the title from the final slide with \verb|\backmatter[notitle]|
              \end{itemize}
        \item The aspect ratio defaults to 16:9, and you should not change it to 4:3
              for old projectors as it is inherently impossible to perfectly convert a
              16:9 presentation to 4:3 one; spacings \emph{will} break
              \begin{itemize}
                  \item The \texttt{aspectratio} argument to the \texttt{beamer} class is
                        overridden by the SINTEF theme
                  \item If you \emph{really} know what you are doing, check the package
                        code and look for the \texttt{geometry} class.
              \end{itemize}
    \end{itemize}
\end{frame}

\section{Summary}

\begin{frame}
    \frametitle{Good Luck!}
    \begin{itemize}
        \item Enough for an introduction! You should know enough by now
        \item If you have corrections or suggestions,
              \hrefcol{mailto:andrea@gasparini.cloud}{send them to me!}
    \end{itemize}
\end{frame}


\end{document}
