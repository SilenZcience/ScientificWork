\documentclass{report}

%%%%%%%%%%%%%%%%%%%%%%%%%%%%%%%%%
% PACKAGE IMPORTS
%%%%%%%%%%%%%%%%%%%%%%%%%%%%%%%%%


\usepackage{graphicx, float}
\graphicspath{{res/}}
\usepackage{tikz}     % tableofcontents
\usepackage{titletoc} % tableofcontents
\usepackage{mathptmx}
\usepackage[skip=0.8em, indent=0pt]{parskip}

\usepackage[hidelinks]{hyperref}


%%%%%%%%%%%%%%%%%%%%%%%%%%%%%%
% MODIFIED COMMANDS
%%%%%%%%%%%%%%%%%%%%%%%%%%%%%%


\newcommand{\bcite}[1]{\textbf{\cite{#1}}}
\renewcommand{\footnoterule}{\vfill\kern -3pt \hrule width 0.4\columnwidth \kern 2.6pt}

\renewcommand{\abstractname}{Zusammenfassung}
\renewcommand{\contentsname}{Inhalt}
\newcommand{\replacechaptername}{Kapitel}
\renewcommand{\chaptername}{\replacechaptername}
\renewcommand{\bibname}{Quellenverzeichnis}
\newcommand{\replacepagename}{Seite}


%%%%%%%%%%%%%%%%%%%%%%%%%%%%%%
% SELF MADE COLORS
%%%%%%%%%%%%%%%%%%%%%%%%%%%%%%


\definecolor{doc}{RGB}{0,60,110}


%%%%%%%%%%%%%%%%%%%%%%%%%%%%%%%%%%%%%%%%%%%
% TABLE OF CONTENTS
%%%%%%%%%%%%%%%%%%%%%%%%%%%%%%%%%%%%%%%%%%%


\contentsmargin{0cm}
\titlecontents{chapter}[3.7pc]
{\addvspace{30pt}
	\begin{tikzpicture}[remember picture, overlay]
		\draw[fill=doc!60,draw=doc!60] (-7,-.1) rectangle (-0.9,.5);
		\pgftext[left,x=-3.5cm,y=0.2cm]{\color{white}\Large\sc\bfseries \replacechaptername\ \thecontentslabel};
	\end{tikzpicture}\color{doc!60}\large\sc\bfseries}
{}
{}
{\;\titlerule\;\large\sc\bfseries \replacepagename\space\thecontentspage
	\begin{tikzpicture}[remember picture, overlay]
		\draw[fill=doc!60,draw=doc!60] (2pt,0) rectangle (4,0.1pt);
	\end{tikzpicture}}
\titlecontents{section}[3.7pc]
{\addvspace{2pt}}
{\contentslabel[\thecontentslabel]{2pc}}
{}
{\hfill\small \thecontentspage}
[]
\titlecontents{subsection}[4.7pc]
{\addvspace{-1pt}\small}
{}
{}
{\hfill\small \thecontentspage}
[]
\titlecontents{bibliography}[3.7pc]
{\addvspace{30pt}
	\color{doc!60}\large\sc\bfseries}
{}
{}
{\;\titlerule\;\large\sc\bfseries \replacepagename\space\thecontentspage
	\begin{tikzpicture}[remember picture, overlay]
		\draw[fill=doc!60,draw=doc!60] (2pt,0) rectangle (4,0.1pt);
	\end{tikzpicture}}

\makeatletter
\renewcommand{\tableofcontents}{
	\chapter*{
	  \vspace*{-20\p@}
	  \begin{tikzpicture}[remember picture, overlay]
		  \pgftext[right,x=15cm,y=0.2cm]{\color{doc!60}\Huge\sc\bfseries \contentsname};
		  \draw[fill=doc!60,draw=doc!60] (13.5,-.75) rectangle (20,1);
		  \clip (13.5,-.75) rectangle (20,1);
		  \pgftext[right,x=15cm,y=0.2cm]{\color{white}\Huge\sc\bfseries \contentsname};
	  \end{tikzpicture}}
	\@starttoc{toc}}
\makeatother



\begin{document}

\title{\Huge{Social Engineering}}
\author{Silas A. Kraume}
\date{\today}

\maketitle

\tableofcontents

\begin{abstract}
    Die modernen Fortschritte in der digitalen Technologie machen die Kommunikation zwischen Menschen immer zugänglicher.
    Immer mehr Menschen sind durch das Internet vernetzt und stellen Informationen online zur Verfügung.
    Oftmals fehlt es Systemen allerdings an ausreichenden Sicherheitsmaßnahmen zum Schutz dieser Informationen.
    Kommunikationssysteme sind anfällig und können mit böswilligen Absichten durch Social Engineering Angriffe leicht durchdrungen werden.
    Diese Form der Angriffe zielt darauf ab, die Schwachstelle Mensch auszunutzen und Individuen dahingehend zu manipulieren, Aktionen gegen ihre eigenen Interessen auszuführen.

    Social Engineering stellt inhärent eine der größten Herausforderungen für die Sicherheit im Internet dar,
    da es keine technisch behebbaren Aspekte von informatischen Systemen ausnutzt, sondern gezielt Menschen manipuliert, um in eben jene Systeme einzudringen.
    Auf diese Weise schaffen sich Cyberkriminelle Zugang zu vertraulichen und sensiblen Informationen wie Passwörtern, Bankinformationen und Sozialversicherungsnummern
    und richten damit einen immensen Schaden an Individuen und Unternehmen an.

    Dieser Report bietet einen detaillierten Überblick über Social Engineering und seine Angriffsvektoren sowie dessen Konsequenzen.
    Er stellt sich die Frage, wieso es keine effektiven Gegenmaßnahmen zu geben scheint, um Social Engineering Angriffe effektik zu verhindern
    beziehungsweise nachhaltig zu stoppen.
    Diesbezüglich werden die Ausbreitung von Social Engineering und dessen global verursachter Schaden technisch analysiert sowie die Psychologie hinter zwischenmenschlicher Manipulation untersucht,
    um einen präzisen Einblick zu erhalten, weshalb Social Engineering in seiner Quantität und Intensität beständig zunimmt.
\end{abstract}

\chapter{Einleitung}
\label{Einleitung}

Social Engineering ist konträr zu seiner modernen Namensgebung sehrwohl bereits seit
Menschengedenken existent. Es lassen sich Beispiele von Social Engineering in der Mythologie,
Religion und Geschichte der Menschheit finden.
Unter den prominäntest\-en Beispielen ist das Trojanische Pferd\footnote{Es wird erzählt, dass
die Griechen den Krieg gegen Troja gewannen,
indem sich Odysseus die Social Engingeering Taktik ausdachte, das hölzerne Pferd zu bauen,
und die Trojaner zu manipulieren, dieses in die eigene Stadt zu bringen.}\bcite{origins,origins2}.

Social Engineering Angriffe dienen also seit Langem als Grundlage für die unterschiedlichsten Betrugsmaschen,
aber nehmen im digitalen Zeitalter quantitativ kontinuierlich zu.
Sie zielen darauf ab durch Manipulation an sensible oder wertvolle Daten zu gelangen
und richten damit immensen Schaden an \bcite{seofwnep,4_mdpi,2_bsi}.
Diese Form von Angriffen richtet sich nicht nur gegen Unternehmen und Regierungsinstitutionen,
sondern auch gegen Individuen (insbesondere bezüglich Identitätsdiebstahl) \bcite{7_mdpi,verizon2012}.

Mit der Entwicklung heutiger ICT\footnote{Informationen and Communication Technology} entwickeln sich auch
Social Engineering Taktiken beständig weiter und mit neuen technologischen Möglichkeiten werden auch
konsequent neue Formen des Social Engineering ermöglicht. Heutzutage verwenden die meisten
Cyber-Angriffe eine Form des Social Engineerings \bcite{1_enisa,evolving}.

\section{Motivation}

Das primäre Motiv von Cyber-Kriminellen ist finanziell. Im Durchschnitt richtet ein 'Data Breach'
4.45 Millionen US-Dollar an Schaden an, wobei Social Engineering als initialer Angriffsvektor noch
über diesem Wert liegt \bcite{6_ibmsecurity}.
Im Jahr 2016 wurde die Bangladesh Bank gehackt, was zu einem immensen finanziellen Verlust führte.
Der Angriff wurde langwierig geplant und begannt bereits ein Jahr zuvor.
Es gelang den Cyber-Kriminellen in das \textit{SWIFT} Bank Netzwerk einzudringen, welches für Geld-überweisungen
genutzt wird. Insgesamt sollten 1 Milliarden US-Dollar transferiert werden, wobei es den Angreifern
letztendlich nur möglich war, 81 Millionen US-Dollar zu stehlen.

Der Verlust, nach erfolgreichen Social Engineering Angriffen, ist jedoch für Unterneh\-men weitreichender.
Neben dem direkten finanziellen Verlust, durch den Diebstahl der Angreifer erleiden Unternehmen zusätzlich
Wiederherstellungskosten, da etwaige Daten verloren gegangen sind, und Sicherheitslücken gefunden und repariert
werden müssen. Des Weiteren entsteht eine Betriebsdisruption, was zu indirektem finanziellen Schaden, durch
Verlust von Produktivität, führt. Zuletzt erleiden Unternehmen einen Reputationsschaden, was in vielen Fällen
den verheerendsten Faktor ausmacht, insbesondere für kleinere Firmen \bcite{agony}.
Für Unternehmen können Cyber-Angriffe auch eine Form der Espionage darstellen, weshalb der Schaden eines
Unternehmens zusätzlich einen kompetitiven Schaden in der Marktwirtschaft darstellen kann.

Individuen erleiden ebenfalls, neben finanziellen-, auch weitere Formen von Schäden.
Nicht außer Acht zu lassen ist der emotionale Schaden, da Personen oft, in Folge einer erfolgreichen
Manipulation, als naiv dargestellt werden.


% TODO: verweis auf explizite SE angriffe im film



% \begin{figure}[H]
%     \centering
%     \includegraphics[width=5in]{IBM_Data.Breach.Report.png}
%     \caption{IBM - Measured in USD millions}
% \end{figure}
% Avg. Kosten pro Breach (2022)


% \begin{figure}[H]
%     \centering
%     \includegraphics[scale=.5]{Barracuda_Social.Engineering.Attacks.png}
%     \caption{Barracuda - Social Engineering Attacks}
% \end{figure}


% \newpage

% "Actor Motives Financial (89\%), Espionage (11\%) (breaches)"\cite{verizon2022}
% "Actor Motives Financial (95\%), Espionage (5\%) (breaches)"\cite{verizon2024}

% conversation hijacking wenig ,denn verlangt etwas erfolg bei vorherigen angriffen.
% z.b. folgt onversation hijacking oftmals auf account takeover.

% "Hackers are starting to increasingly use phishing as part of their
% ransomware attacks."\cite{3_barracuda}

% "Extortion attacks make up only 2\% of the total number of
% targeted phishing attacks we have seen in the past year. These
% attacks were mostly sextortion email threats, where hackers
% threaten to expose sensitive or embarrassing content to their
% victim’s contacts unless a ransom is paid. Demands are usually
% a few hundred or a few thousand dollars and need to be paid
% in bitcoin, which is difficult to trace. In the UK, the number of
% sextortion cases reported to National Crime Agency increased
% by 88\% between 2018 and 2020, and the number is expected to
% continue to increase"\cite{3_barracuda} stand 2021
% stand 2024: Extortion: ($\sim$ 25\%) \cite{verizon2024}
% Darunter fällt auch Ransomware

% Pretexting: 2022 (27\%), 2024 (more than 40\%)
% Phishing: 2022 ($\sim$ 70\%), 2024 (31\%)\cite{verizon2024,verizon2022}

% "Account takeover is a form of identity theft and fraud where a
% malicious third party successfully gains access to a user’s account
% credentials"\cite{3_barracuda}
% "Account takeover is one of the fastest growing threats. In 2021,
% roughly 1 in 5 organizations (20\%) had at least one of their
% Microsoft 365 accounts compromised. This means that in 2021
% hackers managed to compromise around 500,000 Microsoft 365
% accounts around the globe."\cite{3_barracuda}

% Social Engineering MTTI (Mean Time To Identify): 218 Days
% Social Engineering MTTC (Mean Time To Contain) :  80 Days\cite{6_ibmsecurity}




% Subsection \ref{next_subsection} is not useless, it shows how to include figures.


% \subsection{Next Subsection}
% \label{next_subsection}


\chapter{Was ist Social Engineering}

"has significantly evolved with ICT technologies"\cite{1_enisa}
(Informationen and Communication Technology)

"most cyber attacks nowadays include some form of social engineering"\cite{1_enisa}

"Social Engineering ist an sich nichts Neues und dient seit Menschengedenken als Grundlage
für die unterschiedlichsten Betrugsmaschen. Im Zeitalter der digitalen Kommunikation ergeben
sich jedoch äußerst effektive, neue Möglichkeiten für Kriminelle, mit denen sie Millionen von
potenziellen Opfern erreichen können."\cite{2_bsi}

first we need to tlook what se is. different sources define different things ...

\section{Definition}

"Social engineering refers to all techniques aimed at talking a target into revealing specific information or performing a
specific action for illegitimate reasons."\cite{1_enisa}

"Beim Social Engineering werden menschliche Eigenschaften wie Hilfsbereitschaft, Vertrauen, Angst oder Respekt vor Autorität
ausgenutzt, um Personen geschickt zu manipulieren. Cyber-Kriminelle verleiten das Opfer auf diese Weise beispielsweise dazu,
vertrauliche Informationen preiszugeben, Sicherheitsfunktionen auszuhebeln, Überweisungen zu tätigen oder Schadsoftware auf dem
privaten Gerät oder einem Computer im Firmennetzwerk zu installieren"\cite{2_bsi}

"Der Studie liegt die Definition des Verfassungsschutzes Brandenburg zugrunde: "Social Engineering ist der Versuch unter Ausnutzung menschlicher Eigenschaften Zugang zu Knowhow zu erhalten. Der Angreifer nutzt dabei Dankbarkeit, Hilfsbereitschaft, Stolz, Karrierestreben, Geltungssucht, Bequemlichkeit oder Konfliktvermeidung aus. Dabei bieten häufig soziale Netzwerke oder auch Firmenwebseiten Möglichkeiten, um sich auf sein Opfer gründlich
vorzubereiten. Zu diesen "Vorfeldermittlungen" können auch Anrufe im Unternehmen gehören. Professionelle Angreifer versuchen dabei nicht, mit einem Anruf alle gewünschten Informationen zu erlangen, dies könnte misstrauisch stimmen. Der Angerufene wird dabei im
Gespräch nach vermeintlich nebensächlich erscheinenden Informationen gefragt." Kurzfassung Definition Studie: Social Engineering ist eine zwischenmenschliche Manipulation, bei
der ein Unbefugter unter Vortäuschung falscher Tatsachen versucht, unberechtigten Zugang
zu Informationen oder IT-Systemen zu erlangen."\cite{10_bka}

Subsection~\ref{next_subsection} is not useless, it shows how to include figures.


\subsection{Next Subsection}
\label{next_subsection}


\chapter{Maßnahmen}
\label{chapter:massnahmen}

\section{Erkennung \& Vorbeugung}

Um verschiedene Angriffe zu erkennen und damit präventiv zu verhindern, werden verschiedene Techniken vorgeschlagen.
Eine Liste von Abwehrverfahren gegen Social Engineering umfasst Förderung von Sicherheitsschulungen und Steigerung des allgemeinen Bewusstseins für Angriffsvektoren durch entsprechende Aufklärung;
die beste Methode gegen eine soziale Form des Angriffes ist ein soziales Bewusstsein, angegriffen zu werden \bcite{4_mdpi}.
Derartige Schulungen sollten erklären, wie die Sicherheit kritischer Informationen gewährleistet werden kann und stetig auf aktuelle Angriffsmuster aufmerksam machen, wie etwa bekannte Phishing Kampagnen \bcite{4_mdpi}.
Regelmäßige Poster, Präsentationen, E-Mails und Informationsschreiben können weiter dazu beitragen, das Bewusstsein zu verbreiten.
Es wird zudem empfohlen, dass Organisationen Penetrationstests ausführen, um die Anfälligkeit für Social Engineering zu ermitteln \bcite{1_enisa}.

Viele Angriffe verlieren an Wirksamkeit, wenn ausreichende Identifizierungs- und Authentifizierungsprozesse vorhanden sind.
Beispielsweise bietet Mehr-Faktor-Authentifizierung eine zusätzliche Sicherheitsebene zu Benutzername und Passwort.
Hierbei stehen Optionen wie ein Authentifizierungscode, ein Daumenabdruck oder ein Netzhautscan zur Verfügung.
Es sollten verschiedene Anmeldedaten für unterschiedliche Plattformen eingesetzt werden \bcite{1_enisa,3_barracuda}.
Um physischen Angriffen wie Tailgating (\autoref{tailgating}) entgegenzusetzen, sollte der Zugang zu nicht öffentlichen Bereichen durch Zugangsrichtlinien und/oder den Einsatz von Zugangskontrolltechnologien kontrolliert werden.
Die Pflicht, einen Ausweis zu tragen, die Anwesenheit von Sicherheitspersonal und explizite Türen zum Schutz vor Tailgating,
wie Schleusen mit RFID-Zugangskontrolle\footnote{RFID (Radio Frequency Identification signals) verwendet elektromagnetische Ausweise} reichen häufig aus, um die meisten Angreifer abzuschrecken \bcite{1_enisa}.

Oftmals werden persönliche Informationen, die freiwillig offen gelegt wurden, von Kriminellen missbraucht, weshalb bereits der verantwortungsvolle Umgang mit den sozialen Netzwerken eine hilfreiche Gegenmaßnahme darstellen kann.
Unter keinen Umständen sollten private oder berufliche Informationen öffentlich preisgegeben werden.

Angriffen wie Baiting (\autoref{baiting}) kann entgegengewirkt werden, wenn entsprechende Systeme installiert sind, welche unautorisierte Software und Hardware blockieren.
Grundsätzlich gilt, keinen unbekannten Kontaktaufnahmen zu vertrauen \bcite{1_enisa} und die Legitimität von Anrufen und E-Mails (oder anderweitigen Quellen) ausreichend zu prüfen.
Insbesondere sollten bei E-Mails die drei kritischen Punkte Absender, Betreff und Anhang vor dem Öffnen bedacht und überprüft werden \bcite{2_bsi}.
Links sollten nicht geöffnet werden, bevor diese verifiziert wurden. Beispielsweise lässt sich die URL eines Links bereits durch das Bewegen des Mauszeigers über den Link inspizieren, bevor dieser angeklickt wurde.
Merkmale, auf die hierbei geachtet werden sollte, sind, ob die URL semantisch unseriös wirkt und/oder mit \qqq{http} anstelle von \qqq{https} beginnt. 

\section{Juristik}

Die strafrechtlichen Folgen von Social Engineering haben im digitalen Zeitalter beständig angepasst und verbessert werden müssen.
Wie zuvor etabliert ist Social Engineering zudem umfangreich in seinen Möglichkeiten, weshalb Tatbestände detailreich und sachbezogen behandelt werden müssen.
Es folgt keine vollständige strafrechtliche Beurteilung, vielmehr ein grundlegender Überblick über die essenziellsten juristischen Grundlagen.
Des Weiteren ist die Handhabung von Social Engineering als Straftat nicht uniform sondern variiert in unterschiedlichen Ländern.

In Österreich wird Social Engineering beispielsweise juristisch zumeist nur mit Geldbußen bestraft.
\qq{Da sich der Angreifer Zugang zu einem Computersystem verschafft, über das er nicht oder
nicht alleine verfügen darf, ist ein Teil des objektiven Tatbestandes des §118a StGB erfüllt.
Dieser erfordert jedoch [zusätzlich], dass die Zugangsverschaffung durch die Verletzung einer
Sicherheitsvorkehrung erfolgt, die sich \qqq{im Computersystem} befindet. Da sich die durch
Social Engineering Angriffe verletzte Sicherheitsvorkehrung Mensch jedoch nicht \qqq{im
Computersystem} befindet, ist der objektive Tatbestand des §118a StGB idR nicht erfüllt.}\bcite{criminal}
Die Fassung von §118a StGB wurde in den Jahren 2007, 2015 und 2023 neu aufgelegt, enthält in jeder dieser Auflagen jedoch dieselbe Formulieren bezüglich der Verletzung von Sicherheitsvorkehrungen.
Social Engineering ist hier rechtlich gesehen somit allenfalls eine Täuschung \bcite{criminal}.

In Deutschland fällt Social Engineering zumeist unter die \qqq{Verletzung des persönlichen Lebens- und Geheimbereichs}, insbesondere §202a\footnote{Ausspähen von Daten}, §202b\footnote{Abfangen von Daten} und §202c\footnote{Vorbereiten des Ausspähens und Abfangens von Daten}.
Nach §202a Abs 1 StGB gilt:
\qq{Wer unbefugt sich oder einem anderen Zugang zu Daten, die nicht für ihn bestimmt und die gegen unberechtigten Zugang besonders gesichert sind, unter Überwindung der Zugangssicherung verschafft, wird mit Freiheitsstrafe bis zu drei Jahren oder mit Geldstrafe bestraft.}
Für technische Angriffsvektoren (\autoref{technisch}) gilt zudem explizit §202b, welches das unbefugte Verschaffen von nichtöffentlichen Daten unter Anwendung technischer Mittel behandelt.
§202c beschreibt das Vorbereiten der in §202a und §202b erklärten Tatbestände, etwa durch den Besitz von \qq{Computerprogramme[n], deren Zweck die Begehung einer solchen Tat ist}. 

% Insofern Social Engineering aber das Ziel erreicht, Authentifizierungsdaten zu ermitteln, gilt §126c Abs 1 Fall 2 des Strafgesetzbuches:
% \qq{Wer [\dots] ein Computerpasswort, einen Zugangscode oder vergleichbare Daten, die den Zugriff auf ein Computersystem oder einen Teil davon ermöglichen,
% mit dem Vorsatz [\dots] verschafft [\dots], dass sie zur Begehung einer der [\dots] strafbaren Handlungen gebraucht werden, ist mit Freiheitsstrafe bis zu sechs Monaten oder mit Geldstrafe bis zu 360 Tagessätzen zu bestrafen.}
% Erlangt der Angreifer jedoch anderweitige Daten, wie etwa Konfigurationsinformationen des Sicherheitssystems, bleibt es hierbei grundsätzlich bei Straflosigkeit \bcite{criminal}. % -> Österreich!

Social Engineering kann ebenfalls eine Straftat gegen die öffentliche Ordnung darstellen, wenn eine Gefährdung der Verbreitung personenbezogener Daten (§126a) vorliegt.
Insofern bei einem Social Engineering Angriff beispielsweise ein Passwort erlangt wird, muss geprüft werden, ob die Entwendung ein konkretes Individualrecht verletzt.
Dies ist der Fall, insofern das Passwort eine personenbezogene Informationen darstellt, also die Identität einer Person in Zusammenhang des öffentlich bekannten Benutzernamen bestimmbar ist.
Ist das Passwort hingegen für einen administrativen Account eines unpersönlichen Computersystems, so liegen bei der Entwendung keine personenbezogenen Daten vor.
% Bezüglich des Individualrechtes des Datenschutzes gilt, dass das Erlangen von personenbezogenen Daten selbst rechtswidrig ist.
% Dass durch spätere Verwendung derartiger Daten weitere Rechte verletzt werden (können), hat außer Betracht zu bleiben.
% Das Ermitteln von Passwörtern fällt unter 

Bei dem Straftatbestand des Datendiebstahls sowie der Datenhehlerei (§202d) handelt es sich um ein sogenanntes Antragsdelikt, sodass die Polizei erst nach einer entsprechenden Strafanzeige ein Ermittlungsverfahren eröffnen kann \bcite{kotz}.


Selbstverständlich ist die typische Folge eines gelungenen Social Engineering Angriffes oftmals eine weitere Straftat wie Identitätsdiebstahl, Verbraucherbetrug oder Diebstahl.
Diese werden als eigenständige Tatbestände behandelt \bcite{illegal}.

\chapter{Psychologie}

"Es wurden 6 soziale Einfallstore und Mental Shortcuts identifiziert:
o Hilfsbereitschaft
o Leichtgläubigkeit
o Neugier
o (Wunsch nach) Anerkennung
o Druck
o Angst.
"\cite{10_bka}

"Der Mensch reagiert auf bestimmte Auslösemerkmale mit automatisiertem Sozialverhalten. Regeln mit solch hoher gesellschaftlicher Durchschlagkraft lassen sich leicht
missbrauchen:
12
o Die Regel der Reziprozität (Wechselseitigkeit, d.h. wir müssen uns für erhaltene Gefälligkeiten, Geschenke etc. revanchieren. Auf Zugeständnisse müssen wir mit Zugeständnissen reagieren). Falls aber durch Bewusstheit/ Sensibilität erkannt wird, dass der Gefallen oder das Geschenk in Wirklichkeit nur
ein Manöver war, um Vorteile zu erlangen, verliert die Reziprozitätsregel ihre
Durchschlagskraft.
o Das Kontrastprinzip (Kontraste erscheinen durch eine geschickte Präsentation
größer als sie unter anderen Umständen erscheinen würden).
o Die Regel des Commitments und der Konsistenz (d.h. den Menschen wohnt
ein geradezu zwanghaftes Verhalten inne, in Konsistenz mit ihren früheren
Handlungen zu erscheinen - also konsequent zu sein. Wurde eine Entscheidung getroffen, treten intra- und interpsychische Vorgänge in Kraft, die uns
dazu drängen, konsistent zu bleiben. In der Sozialpsychologie und dem Marketing arbeitet man daher mit der sog. "Fuß-in-der-Tür-Taktik". Man beginnt
mit einer kleinen Bitte und arbeitet sich dann zur großen vor oder verändert
das Selbstbild des Gegenübers in die gewünscht Richtung. Hat man das
Selbstbild einer Person erst einmal in eine neue Rolle manipuliert, tut die Person nahezu alles um mit dem neuen Selbstbild konsistent zu bleiben). Zumeist spürt der Mensch, dass er betrogen oder ausgenutzt werden soll, achtet
aber nicht auf dieses "Bauch-Gefühl". Durch Awareness-Training kann hier,
als Gegenmaßnahme, eingegriffen werden.
o Das Prinzip der sozialen Bewährtheit (Das Verhalten anderer wird als richtig
angenommen und gegebenenfalls kopiert bzw. adaptiert).
o Sympathie (Uns sympathische Menschen können uns eher zu einem bestimmten Verhalten verleiten). Sympathie verstärkt alle anderen eingesetzten
Überzeugungstricks. Dazu gehören auch Attraktivität, Ähnlichkeit, gleiche
Herkunft, ähnliche Interessen, Schmeicheleien, Sympathiebekundungen, Flirts.
o Autorität (Autoritätssymbole: Titel, Uniform, Luxus).
o Knappheit (je knapper eine Ware ist, desto mehr gewinnt sie an Wert). "\cite{10_bka}

"Es gibt keinen Abwehrzauber gegen Social-Engineering, denn dabei handelt es sich um Verhalten,
das in der Regel sozial erwünscht ist. Technische
Maßnahmen sind nicht in der Lage, derartige Vorfälle zu verhindern, da es
sich um ein soziales Problem handelt. Zur Abwehr wird die Fähigkeit benötigt,
soziale Beziehungen und Kontexte zu deuten. Es ist notwendig, in Organisationen
ein Sicherheitsbewusstsein im Rahmen einer Sicherheitskultur zu schaffen"\cite{10_bka}

"Prognostisch bleibt zu befürchten, dass SE-Fälle in Zukunft eher ansteigen als abnehmen
werden und die Aufklärung problematisch bleibt. Gründe hierfür sind insbesondere:

 Die Betrügereien werden weiterhin und zunehmend aus dem Ausland oder von nicht
zu identifizierenden Rechnern oder Personen begangen. Dadurch sinkt das Entdeckungsrisiko.
 Scham oder die Angst vor Reputationsverlust kann die Anzeigebereitschaft hemmen.
 Die Aussicht auf immense (schwer abzuschöpfende) Gewinne erhöht den Tatanreiz.
 Die Verfügbarkeit relevanter offener Informationen, die für einen SE-Angriff genutzt
werden können, wird eher ansteigen als abnehmen. Dadurch werden Manipulationen
erleichtert.
 Der Druck auf einzelne Mitarbeiter in der heutigen Arbeitswelt steigt eher als dass er
sinkt und der notwendige Rückhalt/ das Vertrauen in die Organisation, sich vermeintlichen Anweisungen zunächst zu widersetzen, ist nicht immer vorhanden.
 Die "Europäisierung des Betruges" wir nicht adäquat mit der Europäisierung der
Strafverfolgung beantwortet und "die internationale Rechtshilfe ist in hohem Maße defizitär"."\cite{10_bka}

"The Dual Process Model of Persuasion [19] defines two different ways how we process information: the peripheral route or heuristic processing via intuition
(system 1) and the central route via reasoning (system 2) (cf. [18]). Attackers can target both systems."\cite{7_mdpi}

"six principles of influence":\cite{7_mdpi}

"Authority
Most people comply to authorities (cf. Milgram experiments [24]), even if they persuade them to act against their beliefs and ethics. It also works for symbols of authority, e.g. uniforms, badges, and titles or in telephone conversations where authority can easily be claimed. Two types of authority exist: one based on expertise and one relying on the relative hierarchical position in an organisation or society [19].

Commitment \& Consistency
Commitment is an act of stating what one person thinks he is and does, while consistency makes that same person behave consistently according to his or her commitments and beliefs revealing a a highly successful influence principle [19].

Reciprocity
A strong social norm that obliges us to repay others for what we have received from them. Relationships rely and societies are built on it. Reciprocity helps establishing trust with others and refers to our need for equity. The power of reciprocity can be so high that the target would return an even greater favour than what was received.

Liking
'If you make it plain you like people, it's hard for them to resist liking you back” [25]. We prefer to comply with requests from people we know and like due to the fundamental motive to create and maintain social relationships. Perceived similarity enhances compliance as it can originate from a potential friend. These can be as superficial as shared names or birthdays.

Social Proof
Besides adapting beliefs and behaviour of people around in order to become socially “accepted”, social proof also implies higher trust levels towards people who share alike opinions, especially in ambiguous situations.

Scarcity
We assign more value to less available opportunities due to a short-cut from availability to quality. Moreover, if something becomes scarce, we sense losing freedoms. Reactance Theory [26] suggests that we respond to scarcity by wanting to have what has become rare more than before. Even information with limited access persuades better."\cite{7_mdpi}

"In psychology, personality is defined as a person's relatively stable feelings, thoughts, and behavioural patterns. These are predominantly determined by inheritance, social and environmental influence, and experience, and are therefore unique for every individual [27]."\cite{7_mdpi}


"The FFM consists of five broad, empirically derived personality dimensions or traits, which split in several subtraits and are used across research areas with high validity: these traits are defined as Conscientiousness which focus on competence, self-discipline, self-control, persistence, and dutifulness as well as following standards and rules. Extraversion comprises positive emotions, sociability, dominance, ambition, and excitement seeking. Agreeableness includes compassion, cooperation, belief in the goodness of mankind, trustfulness, helpfulness, compliance, and straightforwardness. Openness to experience encompasses as a preference for creativity, flexibility’ fantasy as well as an appreciation of new experiences and different ideas and beliefs. Neuroticism describes the tendency to experience negative emotions, anxiety, pessimism, impulsiveness, vulnerability to stress, and personal insecurity."\cite{7_mdpi}


\chapter{Psychologie}
\label{chapter:psychologie}

\section{psychologische Grundlagen}

"Es wurden 6 soziale Einfallstore und Mental Shortcuts identifiziert:
o Hilfsbereitschaft
o Leichtgläubigkeit
o Neugier
o (Wunsch nach) Anerkennung
o Druck
o Angst.
"\cite{10_bka}

"Es gibt keinen Abwehrzauber gegen Social-Engineering, denn dabei handelt es sich um Verhalten,
das in der Regel sozial erwünscht ist. Technische
Maßnahmen sind nicht in der Lage, derartige Vorfälle zu verhindern, da es
sich um ein soziales Problem handelt. Zur Abwehr wird die Fähigkeit benötigt,
soziale Beziehungen und Kontexte zu deuten. Es ist notwendig, in Organisationen
ein Sicherheitsbewusstsein im Rahmen einer Sicherheitskultur zu schaffen"\cite{10_bka}

"Prognostisch bleibt zu befürchten, dass SE-Fälle in Zukunft eher ansteigen als abnehmen
werden und die Aufklärung problematisch bleibt. Gründe hierfür sind insbesondere:

Die Betrügereien werden weiterhin und zunehmend aus dem Ausland oder von nicht
zu identifizierenden Rechnern oder Personen begangen. Dadurch sinkt das Entdeckungsrisiko.
Scham oder die Angst vor Reputationsverlust kann die Anzeigebereitschaft hemmen.
Die Aussicht auf immense (schwer abzuschöpfende) Gewinne erhöht den Tatanreiz.
Die Verfügbarkeit relevanter offener Informationen, die für einen SE-Angriff genutzt
werden können, wird eher ansteigen als abnehmen. Dadurch werden Manipulationen
erleichtert.
Der Druck auf einzelne Mitarbeiter in der heutigen Arbeitswelt steigt eher als dass er
sinkt und der notwendige Rückhalt/ das Vertrauen in die Organisation, sich vermeintlichen Anweisungen zunächst zu widersetzen, ist nicht immer vorhanden.
Die "Europäisierung des Betruges" wir nicht adäquat mit der Europäisierung der
Strafverfolgung beantwortet und "die internationale Rechtshilfe ist in hohem Maße defizitär"."\cite{10_bka}


\section{menschliche Beeinflussung}

\subsection{Einflussprinzipien}

Ein Angreifer kann die Entscheidungsfähigkeit zu seinem Vorteil beeinflussen, denn Menschen reagieren oftmals mit automatisiertem Sozialverhalten \bcite{10_bka}.
Der Psychologe Robert Cialdini entwickelte die sechs Prinzipien der Überzeugung (\qq{Six Principles of Persuasion}),
welche in weitreichenden Studien demonstiert wurden\footnote{Da in der Psychologie Situationsfaktoren und menschliche Emotionen immer eine Rolle spielen, ist der Erfolg bei Anwendung dieser Prinzipien nicht garantiert.}\bcite{7_mdpi}.
Die sechs Prinzipien bestehen aus:

\subsubsection{Authority}
Die meisten Menschen neigen dazu, Autoritätspersonen, beziehungsweise Personen, mit Fachwissen oder ergiebigen Erfahrungen, zu glauben, und gehorchen den Anweisungen eben jener.
Autorität (engl. \qqq{Authority}) kann Menschen selbst dazu verleiten, gegen ihren Glauben oder ihre moralische Vorstellung zu handeln.
Eine Person gilt als Autoritätssymbol, wenn sie als legitimer Experte wahrgenommen wird. Symbole der Autorität, wie etwa Titel, äu\ss eres Erscheinungsbild oder Statussymbole, wie
luxuriöse Gegenstände, erhöhen die Gefolgsamkeit bei Anderen \bcite{7_mdpi,psyprinciples,10_bka}.

\subsubsection{Commitment \& Consistency}
Konsistenz (engl. \qqq{Consistency}) sorgt dafür, dass Personen sich konsequent zu ihren Verpflichtungen (engl. \qqq{Commitments}) und Überzeugungen verhalten.
Das menschliche Verlangen nach Konsistenz gegenüber eingegangenen Verpflichtungen kann das Verhalten einer Person langzeitlich beeinflussen.
Beispielsweise lässt sich dieses Verhalten insofern feststellen, dass Personen die eine kleine Petition unterschreiben später wesentlich gewillter sind, sich auch anderweitig zu diesem Zweck zu engagieren \bcite{7_mdpi,psyprinciples,10_bka}.

\subsubsection{Reciprocity}
Das Prinzip der Gegenseitigkeit (engl \qqq{Reciprocity}) beruht auf der fundamentalen Tendenz, dass sich Menschen dazu verpflichtet fühlen, Gefallen oder Geschenke zu erwiedern.
Dieses Prinzip ist derartig stark, dass die Erwiederung vehementer ausfallen kann, als den initial erhaltenen Gefallen \bcite{7_mdpi,psyprinciples,10_bka}.

\subsubsection{Liking}
Aufgrund des grundlegenden Motivs, soziale Beziehungen aufzubauen und aufrechtzuhalten, führen Menschen die Anfragen Anderer eher aus, wenn sie diese Person kennen oder mögen (engl. \qqq{like}).
Wahrgenommene Ähnlichkeiten erhöhen die Fügsamkeit einer Person. Diese können durchaus oberflächlicher Natur sein, wie etwa ein gemeinsamer Geburtstag oder Name.
Andere Faktoren, die dazu beitragen von einer Person gemocht zu werden, sind physische Attraktivität und positive Assoziation, etwa durch Komplimente \bcite{7_mdpi,psyprinciples,10_bka}.

\subsubsection{Social Proof}
Das Prinzip des sozialen Beweises (engl. \qqq{social proof}) ist ein Phänomen, dass in der Psychologie als \qqq{informativer sozialer Einfluss} bekannt ist.
Es geht um eine mächtige Überzeugungstaktik, die ausnutzt, dass Menschen eine natürliche (bzw. primitive) Tendenz haben, dem Beispiel Anderer folgen zu wollen, um sozial akzeptiert zu werden \bcite{7_mdpi,psyprinciples,10_bka}.

\subsubsection{Scarcity}
Das letzte Prinzip der Überzeugung ist die Knappheit (engl. \qqq{scarcity}), und beschreibt die Tatsache, dass Menschen einer geringeneren Quantität eine höhere Qualität zuschreiben.
Dieses Prinzip ist nicht ausschließlich auf Materielles anzuwenden, sondern gilt beispielsweise auch bei verhaltenstechnisch weniger verfügbaren Möglichkeiten, oder bei Informationen, die nicht allgemein verfügbar sind.
So wirken Informationen, die einem im Geheimen anvertraut werden, oft spektakulärer \bcite{7_mdpi,psyprinciples,10_bka}.

\subsection{Persönlichkeitsmerkmale}

Das Fünf-Faktor-Modell ist ein Modell der Persönlichkeitspsychologie, welches fünf Kernaspekte der Persönlichkeit identifiziert und besagt, dass jeder Mensch sich auf den Skalen dieser Kernaspekte einordnen lässt.
Das Modell wird auch als OCEAN-Modell bezeichnet (nach den entsprechenden Anfangsbuchstaben Openness, Conscientiousness, Extraversion, Agreeableness, Neuroticism\footnote{zu Deutsch: Offenheit, Gewissenhaftigkeit, Extraversion, Verträglichkeit, Neurotizismus}).
Die folgende Tabelle beschreibt knapp die verschiedenen Kerneigenschaften:

\begin{table}[!h]
    \begin{center}
        \caption{Charakteristiken der OCEAN-Modell Persönlichkeitseigenschaften \bcite{psyse}}
        \begin{tabular}{ |c|m{11em}|m{11em}| }
            \hline
            \textbf{Eigenschaft} & \textbf{hoher Wert}                                                      & \textbf{niedriger Wert}                                   \\
            \hline \hline
            Offenheit                           & intellektuell, fantasievoll, kontaktfreudik. Offen für Neues             & praktisch, konventoniell, skeptisch, rational             \\
            \hline
            Gewissenhaftigkeit                  & organisiert, eigenständig, gründlich, zuverlässig, aber kontrollierend   & desorganisiert, nachlässig, kann anfällig für Sucht sein  \\
            \hline
            Extraversion                        & aufgeschlossen, enthusiastisch, aktiv; sucht Neues                       & distanziert, ruhig, unabhängig; vorsichtig, zurückgezogen \\
            \hline
            Verträglichkeit                     & vertrauensvoll, unkompliziert, empathisch, nachgiebig, umgänglich        & unkooperativ, feindselig; unempathisch                    \\
            \hline
            Neurotizismus                       & anfällig für Stress, Angst, Befangenheit, Launenhaftigkeit, Impulsivität & emotional stabil, ruhig und sicher.                       \\
            \hline
        \end{tabular}
    \end{center}
\end{table}

\subsection{Anfälligkeit für Social Engineering}






\chapter{Konklusion}
\label{chapter:konklusion}

Zusammengefasst gibt es eine Vielzahl von Faktoren, die dazu beitragen, dass dem kontinuierlichen Anstieg von Social Engineering Angriffen nicht ausreichend entgegengewirkt werden kann.
Hinsichtlich der Forschungsfrage stellt sich heraus, dass unter den primären Gründen die simple quantitative Invasion von Angriffen ist, welche einen immensen Schaden verursacht.

Die Intensität und überwältigende Anzahl von Cybercrime Angriffen steigt jährlich vehement an, weshalb es schwierig ist, gezielt gegen Social Engineering vorzugehen.
\qq{Prognostisch bleibt zu befürchten, dass SE-Fälle\footnote{Social Engineering (SE)} in Zukunft eher ansteigen als abnehmen werden und die Aufklärung problematisch bleibt.}\bcite{10_bka}

Obwohl gegen technische Angriffe teilweise vorgegangen werden kann, etwa durch E-Mail Filter oder anderweitige technische Systeme, ist ein aktueller Shift zu beobachten,
dass soziale Angriffsformen die Mehrheit aller Angriffe bilden. Dies stellt insofern eine Problematik dar, dass soziale Social Engineering Taktiken die größte Erfolgschance aufweisen.
Sie zielen auf die direkte zwischenmenschliche Manipulation ab und weisen eine hohe Effektivität auf.

Auf der psychologischen Ebene hat sich herausgestellt, dass fast jede Person manipulierbar ist. Verschiedene Persönlichkeitsmuster bieten keinen Schutz vor Social Engineering Angriffen, da unter Verwendung
passender Methodiken jede Persönlichkeit auf konkrete Manipulationsprinzipien ansprechbar ist.
Social Engineering nutzt menschliches Verhalten aus, welches gesellschaftlich grundsätzlich erwünscht ist.
Beispielsweise zielt ein Angreifer auf die Hilfsbereitschaft und Freundlichkeit seiner Mitmenschen ab.
Da es nicht problemlos möglich ist, derartiges Verhalten zu unterbinden, besteht in einer sozialen Gesellschaft immer das Risiko von Manipulation und die Ausnutzung dieser.
\newpage
Zuletzt stellt die fehlende Kenntnis in der allgemeinen Bevölkerung ein weiteres Hauptproblem dar.
Die meisten Menschen sind nicht mit dem Begriff von Social Engineering oder dem dahinter stehenden Konzept vertraut.
Ohne die notwendige Aufklärung ist es grundsätzlich schwierig, einen Angriff zu erkennen und diesem folglich entgegenzusetzen, um sich oder andere zu schützen.

Im digitalen Zeitalter ist es notwendig, sich ausreichend mit den Gefahren des Internets auseinanderzusetzen.
Von den genannten Gründen für den fehlenden Aufschwung gegen Social Engineering Angriffe scheint die fehlende Bildung ein zunächst lösbares Problem darzustellen.
Zukünftig sollte mehr in die allgemeine Aufmerksamkeit bezüglich der hier präsentierten Problematik investiert werden.
Es ist essenziell weiterhin mit dem aktuellen Stand des Themas in Kontakt zu bleiben und den durch Social Engineering dargestellten Bedrohungen des Internets maximal entgegenzuwirken.


\phantomsection
\addcontentsline{toc}{bibliography}{\bibname}
\bibliographystyle{plain}
\bibliography{references}

\end{document}