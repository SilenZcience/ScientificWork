

\section{Konklusion}

Zusammenfassend lässt sich festhalten, dass die Restrukturierung des bestehenden Systems erfolgreich umgesetzt werden konnte.
Die Z3-Komponente wurde gänzlich von der Hauptanwendung ProB entkoppelt und in einen seperaten Prozess ausgelagert.
Durch die Einführung einer Kommunikationsstruktur ist die Interaktion zwischen den Komponenten gewährleistet.
Die durchgeführten Tests bestätigen die vollständige Funktionalität des Systems nach der durchgeführten Architekturänderung,
da alle Tests der relevanten Testkategorien fehlerfrei bestanden worden sind.

Durch die Entkopplung der Komponenten konnte die Wartbarkeit und Erweiterbarkeit des Systems verbessert werden.
Insbesondere behebt die neue Architektur bereits eine Vielzahl von potenziellen Problemen,
wobei speziell die Überarbeitung des Exception-Handlings bei der Verwendung des Z3-Solvers hervorzuheben ist.

Während der gesamten Implementierung wurde Rücksicht genommen auf zukünftige Erweiterungen und Verbesserungen des Gesamtsystems,
sodass zukünftig die Option besteht, das Z3-Interface dahingehend zu erweitern, dass gegebenenfalls mehrere Instanzen des Z3-Prozesses gestartet werden und das Lösen der Prädikate parallelisiert wird.

Eine überraschende Erkenntnis war die erkennbare Laufzeitverbesserung durch die neue Architektur.
Die Entkopplung und Optimierung der Komponenten führte
zu einer messbaren Effizienzsteigerung bei dem Lösen von Prädikaten.

Aufgrund der gelungenen Umsetzung der Architekturänderung und der positiven Ergebnisse aller durchgeführten Tests
sowie der erzielten Erweiterbarkeit hat die Arbeit ihre angestrebten Ziele erreicht und alle gesetzten Anforderungen erfüllt.
