%%%%%%%%%%%%%%%%%%%%%%%%%%%%%%%%%%%%%%%%%%%%%%%%%%%%%%%%%%%%%%%%%%%%%%%%%%%%%%%%
% Universität Düsseldorf                                                       %
% Lehrstuhl für Softwaretechnik und Programmiersprachen                        %
% Vorlage für Bachelor- und Masterarbeiten                                     %
% Erstellt: 2019-09-03                                                         %
%%%%%%%%%%%%%%%%%%%%%%%%%%%%%%%%%%%%%%%%%%%%%%%%%%%%%%%%%%%%%%%%%%%%%%%%%%%%%%%%
\documentclass{hhuthesis}


%%%%%%%%%%%%%%%%%%%%%%%%%%%%%%%%%%%%%%%%%%%%%%%%%%%%%%%%%%%%%%%%%%%%%%%%%%%%%%%%
%% Einstellungen zur Personalisierung                                         %%
%%                                                                            %%
%% Im Folgenden können Sie Ihre Arbeit personalisieren.                       %%
%%%%%%%%%%%%%%%%%%%%%%%%%%%%%%%%%%%%%%%%%%%%%%%%%%%%%%%%%%%%%%%%%%%%%%%%%%%%%%%%

%% Spracheinstellung
%% Kommentieren Sie die entsprechende Zeile ein bzw. aus.
%% Wir empfehlen jedem sich an einer englischen Arbeit zu versuchen.
% \usepackage[ngerman,english]{babel} % English
\usepackage[english,ngerman]{babel} % Deutsch

%% Ihr Name
\author{Silas Alexander Kraume}

%% Der Titel der Arbeit
\title{Entkopplung der Z3 Komponente in ProB mit ZeroMQ}
% \subtitle{Usually not needed}

%% Der zu erreichende Abschluss, entweder Bachelor oder Master
\graduationtype{Bachelor}
% \graduationtype{Master}

%% Ihr Studienfach
\subject{Informatik}

%% Beginn- und Abgabedaten der Arbeit
\begindate{22.~Oktober~2024} % Beginn
\duedate{22.~Januar~2025} % Abgabe

%% Erst- und Zweitgutachter
\firstexaminer{Prof.~Dr.~Michael~Leuschel}
\secondexaminer{Dr.~Carl~Friedrich~Bolz-Tereick}

%% Farb- oder Schwarzweißdruck
% Benutzen Sie das Kommando \blackwhiteprint,
% wenn sie in schwarzweiß drucken möchten.
% Im Farbdruck ist jede farbige Seite idR teurer.
% \blackwhiteprint % Kommentarzeichen entfernen für Schwarzweißdruck

%%%%%%%%%%%%%%%%%%%%%%%%%%%%%%%%%%%%%%%%%%%%%%%%%%%%%%%%%%%%%%%%%%%%%%%%%%%%%%%%
%% (Ende) Einstellungen zur Personalisierung                                  %%
%%%%%%%%%%%%%%%%%%%%%%%%%%%%%%%%%%%%%%%%%%%%%%%%%%%%%%%%%%%%%%%%%%%%%%%%%%%%%%%%
%% LaTeX Packages in Nutzung                                                  %%
%%                                                                            %%
%% Im folgenden können Sie für die Niederschrift Ihrer Arbeit benötigte       %%
%% LaTeX-Pakete einbinden.                                                    %%
%% Diese Vorlage kommt bereits mit einigen nützlichen inkludierten Paketen.   %%
%%%%%%%%%%%%%%%%%%%%%%%%%%%%%%%%%%%%%%%%%%%%%%%%%%%%%%%%%%%%%%%%%%%%%%%%%%%%%%%%

%% Macht den \todo-Befehl verfügbar.
%% Hiermit können Sie Abschnitte annotieren,
%% welche weiterer Bearbeitung bedürfen.
\usepackage[textsize=scriptsize]{todonotes}

%% Zeige Zeilennummern in der Arbeit an.
%% Der \linenumbers Befehl muss hierzu aufgerufen werden.
%% Praktisch für Feedback Ihrer potentiellen Korrekturleser!
\usepackage{lineno}
% \linenumbers % <- Kommentar entfernen!


%% Häufig benutzte mathematische Packages.
\usepackage{amsfonts}
\usepackage{amsmath}
\usepackage{amssymb}

\usepackage{siunitx} % \num Befehl zum einfacheren Formatieren von Zahlen.
\usepackage{enumitem} % Leichter konfigurierbare enumerate-Umgebungen.
\usepackage{subcaption} % Unterteilung von Figures in Subfigures.
\usepackage[colorlinks]{hyperref} % Klickbare Links (z.B. Inhaltsverzeichnis).
\usepackage[hypcap=true]{caption} % Setzt Hyperref-Links an den Float-Anfang.
\usepackage{xurl} % \url Kommando für Darstellung von Links
\usepackage{csquotes} % Improved quoting.
\usepackage{microtype} % Verbessertes Kerning zwischen Wörtern.

%% Tabellen
\usepackage{tabularx} % tabularx Umgebung für mehr Kontrolle über Tabellen.
\usepackage{booktabs} % \toprule, \midrule, \bottomrule
\usepackage{multirow}
\usepackage{multicol}
\usepackage{longtable} % Große Tabellen gehen über mehrere Seiten.

%% Quellcode
\usepackage{listings} % Einbindung von Code.

%% Algorithmen in Pseudocode
\usepackage{algorithm} % Float-Umgebung für angegebene Algorithmen.
\usepackage{algorithmicx} % Angabe von Algorithmen in Pseudocode.
\usepackage{algpseudocode} % Standard Pseudocode-Elemente für Algorithmen.

%% Intelligenteres Referenzieren mittels \cref.
%% \languagename um dynamisch zwischen ngerman oder english zu wechseln.
\usepackage[\languagename,capitalize,noabbrev]{cleveref}

%% macht \FloatBarrier verfügbar
\usepackage{placeins}

%% TikZ und PGFPlots f+r Grafik der Architektur
\usepackage{tikz, pgfplots}
\usetikzlibrary{arrows,fit,calc,positioning,fadings}
\pgfplotsset{compat=1.18}

\usepackage[all]{nowidow}

%%%%%%%%%%%%%%%%%%%%%%%%%%%%%%%%%%%%%%%%%%%%%%%%%%%%%%%%%%%%%%%%%%%%%%%%%%%%%%%%
%% (Ende) LaTeX Packages in Nutzung                                           %%
%%%%%%%%%%%%%%%%%%%%%%%%%%%%%%%%%%%%%%%%%%%%%%%%%%%%%%%%%%%%%%%%%%%%%%%%%%%%%%%%


\begin{document}
%% Set up title page, declaration of authorship, abstract, acknowledgements
\frontmatter
\makefrontmatter

%%%%%%%%%%%%%%%%%%%%%%%%%%%%%%%%%%%%%%%%%%%%%%%%%%%%%%%%%%%%%%%%%%%%%%%%%%%%%%%%
%% Danksagungen                                                               %%
%%%%%%%%%%%%%%%%%%%%%%%%%%%%%%%%%%%%%%%%%%%%%%%%%%%%%%%%%%%%%%%%%%%%%%%%%%%%%%%%
\begin{acknowledgements}
  Ich möchte meinen herzlichen Dank an meinen Betreuer Dr. Philipp Körner aussprechen,
  dessen unermüdliche Bereitschaft und Geduld, sich meinen unzähligen Fragen zu widmen, maßgeblich zum Erfolg dieser Arbeit beigetragen hat.
  Sein enormes Engagement und seine wertvollen Ratschläge haben sich mir als eine große Hilfe erwiesen.
  Ich wurde \enquote{bei keiner Bachelorarbeit bisher so gut betreut} und bin ihm für seine Unterstützung sehr verbunden.

  Ich bedanke mich bei David Geleßus für das schnelle Handeln bei technischen Problemen auf der Seite der Universitätssysteme.

  Ebenso möchte ich mich zutiefst bei meinem Freund Axel Andrèe bedanken,
  der mir eine tatkräftige Unterstützung bei der Erstellung der Arbeit war und mit Sorgfalt das Korrekturlesen übernommen hat.
  Er stand mir speziell bei mathematischen Problemen stets zur Seite und hat mir bei der wissenschaftlichen Korrektheit von Formulierungen im Text geholfen.

  Einen ganz lieben Dank an meine Mutter,
  die mir eine Vielzahl von Ablenkungen ersparen konnte,
  welche es mir ermöglichten mich vollständig auf die Arbeit zu konzentrieren.
  Sie hat mich auch grundsätzlich in meinem Studium immer stark unterstützt und mich stets in meinem Vorhaben bestärkt;
  ohne sie wäre ich nicht da, wo ich heute bin.
  

\end{acknowledgements}
%%%%%%%%%%%%%%%%%%%%%%%%%%%%%%%%%%%%%%%%%%%%%%%%%%%%%%%%%%%%%%%%%%%%%%%%%%%%%%%%
%% (Ende) Danksagungen                                                        %%
%%%%%%%%%%%%%%%%%%%%%%%%%%%%%%%%%%%%%%%%%%%%%%%%%%%%%%%%%%%%%%%%%%%%%%%%%%%%%%%%

\tableofcontents

%% Listings of figures, tables, etc. Delete what is not needed.
\clearpage
\listoftables\thispagestyle{headings}
\listoffigures
% \listofalgorithms % Algorithms
\lstlistoflistings % Code Listings

\mainmatter

%%%%%%%%%%%%%%%%%%%%%%%%%%%%%%%%%%%%%%%%%%%%%%%%%%%%%%%%%%%%%%%%%%%%%%%%%%%%%%%%
%% Der Inhalt der Arbeit                                                      %%
%%                                                                            %%
%% Hier können Sie die schriftliche Ausarbeitung ihrer Arbeit                 %%
%% niederschreiben. Der Übersicht halber bietet sich jedoch an, dies in einer %%
%% oder mehreren separaten Dateien zu tun, welche mittels \input eingebunden  %%
%% werden --- wie auch in der Vorlage geschieht.                              %%
%%%%%%%%%%%%%%%%%%%%%%%%%%%%%%%%%%%%%%%%%%%%%%%%%%%%%%%%%%%%%%%%%%%%%%%%%%%%%%%%



\todo{verfiziere zeitform}
\section{Einführung}

Die digitale Transformation hat unsere Welt grundlegend verändert und macht Software-Systeme zu einem unverzichtbaren Bestandteil des täglichen Lebens.
Von sicherheitskritischen Anwendungen wie der Steuerung autonomer Fahrzeuge bis hin zu Finanzsystemen und medizinischen Geräten sind wir zunehmend auf Software angewiesen,
die zuverlässig und fehlerfrei funktioniert.
Die Gewährleistung von Korrektheit und Stabilität ist jedoch eine anspruchsvolle Aufgabe
insbesondere angesichts der Komplexität moderner Systeme.
Ein zentraler Baustein zur Bewältigung dieser Herausforderung ist der Einsatz von Modellierungs- und Verifikationswerkzeugen.
Diese ermöglichen es, komplexe Systeme systematisch zu analysieren und sicherzustellen,
dass sie den gewünschten Spezifikationen entsprechen.
Besonders hervorzuheben ist der Einsatz von SMT\footnote{Satisfiability Modulo Theories}-Solvern,
die sich als leistungsfähige Werkzeuge etabliert haben,
um schwierige logische Probleme effizient zu lösen.
SMT-Solver wie Z3 \cite{10.1007/978-3-540-78800-3_24} bieten durch ihre Fähigkeit zur präzisen und schnellen Verarbeitung logischer Ausdrücke eine wertvolle Unterstützung bei der Verifikation und Validierung.
Ein prominentes Beispiel für die Integration eines solchen Solvers ist die Software ProB \cite{leuschel2003prob}.
Der Animator, Constraint-Solver und Model-Checker ProB nutzt den SMT-Solver Z3, um formale Modelle effizient zu analysieren und zu überprüfen.
Dies macht die Software zu einer wichtigen Instanz in der Welt der formalen Methoden,
insbesondere im Kontext von Modellierungs- und Verifikationsaufgaben.

\subsection{Motivation}

Innerhalb von ProB birgt der Einsatz des Z3-Solvers jedoch auch Herausforderungen, die die Effizienz und Zuverlässigkeit der Anwendung beeinträchtigen können.
Ein bekanntes Problem besteht in dem sporadischen Auftreten von Speicherlecks und Segmentation Faults,
die sowohl die Stabilität als auch die Nutzbarkeit von ProB's Z3-Interface negativ beeinflussen.
Diese technischen Mängel erschweren nicht nur die Durchführung formaler Verifikationen,
sondern können auch zu einer zeitraubenden Verwendung der Z3-Solver Komponente sowie Unterbrechung von Arbeitsprozessen führen.
Die Problematik wird in einem Papier \cite{10.1007/978-3-031-25803-9_5}, geschrieben von Körner und Leuschel, genauer erörtert.

Ein weiterer Mangel liegt in der aktuellen sequenziellen Lösung mehrerer Prädikate.
Dieser Ansatz, bei dem die Prädikate nacheinander gelöst werden,
ist in seiner Natur ressourcenintensiv und zeitaufwendig.
Angesichts der steigenden Komplexität formaler Modelle und der wachsenden Nachfrage nach schnellerer Verifikation wird die Limitierung durch die sequenzielle Verarbeitung immer offensichtlicher.
Eine Parallelisierung der Lösung von Prädikaten könnte hier erhebliche Leistungsverbesserungen bringen,
indem moderne Mehrkernarchitekturen effizienter ausgenutzt werden,
um den Anforderungen der Nutzer und der immer komplexer werdenden Modelle gerecht zu werden.

Die Kombination dieser Herausforderungen (sporadische technische Instabilitäten und begrenzte Effizienz durch sequenzielle Verarbeitung) macht es notwendig,
alternative Ansätze oder Verbesserungen für die Integration des Z3-Solvers in ProB zu erforschen und
bildet die Grundlage und Motivation für die vorliegende Arbeit.

\subsection{Ziele}
\label{sec:goals}

Das Hauptziel dieser Arbeit ist es, die Integration des Z3-Solvers in ProB zu verbessern,
indem die bestehende Vorgehensweise, die Prädikate direkt im Z3-Interface von ProB zu lösen, verworfen wird.
Stattdessen wird eine neue Architektur vorgeschlagen und implementiert,
welche eine vollständige Entkopplung der Z3-Solver-Komponente von ProB vorsieht.
Hierzu wird also der Z3-Solver in einen eigenständigen, separaten Prozess ausgelagert,
wodurch ein System eingeführt wird, bei dem ProB und der Z3-Solver als zwei unabhängige Prozesse agieren,
die über eine Kommunikationsschnittstelle miteinander verbunden sind.
Prädikate werden hierdurch innerhalb des Z3-Interfaces an den Z3-Solver gesendet, wo diese gelöst und zurückgeschickt werden.
Es entsteht somit ein simples Client-Server Modell, welches eine chronologische Folge von Anfragen und Antworten implementiert.
ProB stellt hierbei den Client und der Z3-Solver den Server dar.
Diese geplante Architekturänderung ist in der folgenden \cref{fig:architecture} visualisiert.
\newline

\begin{figure}[!htp]
    \centering
    \begin{tikzpicture}[
            HOT/.style={rectangle, draw=red!60, fill=red!5, very thick, minimum size=40, align=center},
            PB/.style={rectangle, draw=blue!60, fill=blue!5, very thick, minimum size=40, align=center},
            COLD/.style={rectangle, draw=black!40, fill=black!3, very thick, minimum size=40, align=center},
        ]
        \begin{scope}
            \node[PB]    (ProB)                             {ProB};
            \node[HOT]    (C_Interface)       [right=of ProB] {C-Interface\\Z3-Solver};
            \node[draw=none,fill=none,rectangle,above=0.5cm of ProB,xshift=-0.5cm,anchor=south west]
            (Arch_A){Bestehende Architektur};

            \draw[->, very thick] ([yshift=0.4cm]ProB.east)  to node[above,scale=0.8] {\tiny{constraint}} node[below,scale=0.8] {\tiny{posting}} ([yshift=0.4cm]C_Interface.west);
            \draw[<-, very thick] ([yshift=-0.4cm]ProB.east) to node[below,scale=0.9] {\tiny{Solution}} ([yshift=-0.4cm]C_Interface.west);
        \end{scope}
        \node[draw,inner xsep=0.5cm, inner ysep=0.5cm,fit=(Arch_A) (ProB) (C_Interface)] (LeftScope){};
        \node[draw,dashed,inner xsep=0.2cm,inner ysep=0.2cm,fit=(ProB) (C_Interface)] (P0){};

        \begin{scope}[xshift=6.5cm]
            \node[PB]    (ProB)        {ProB};
            \node[HOT]    (C_Interface)       [right=of ProB] {C-Interface};
            \node[HOT]    (Z3_Solver)       [right=of C_Interface] {Z3-Solver};
            \node[COLD]    (Z3_Solver_G1)       [above=of Z3_Solver] {Z3-Solver};
            \node[COLD]    (Z3_Solver_G2)       [below=of Z3_Solver] {Z3-Solver};
            \node[draw=none,fill=none,rectangle,above=1cm of ProB,xshift=1cm,anchor=south west]
            (Arch_B){Zielarchitektur};

            \draw[->, very thick] ([yshift=0.4cm]ProB.east)  to node[above,scale=0.8] {\tiny{constraint}} node[below,scale=0.8] {\tiny{posting}} ([yshift=0.4cm]C_Interface.west);
            \draw[<-, very thick] ([yshift=-0.4cm]ProB.east) to node[below,scale=0.9] {\tiny{Solution}} ([yshift=-0.4cm]C_Interface.west);
            \draw[<->, very thick] (C_Interface) to node[below,scale=0.9] {\tiny{ZMQ}} (Z3_Solver);
            \draw[<->, dashed, thick] (C_Interface.north east) to node[right,scale=0.9] {\tiny{ZMQ}} (Z3_Solver_G1.south west);
            \draw[<->, dashed, thick] (C_Interface.south east) to node[right,scale=0.9] {\tiny{ZMQ}} (Z3_Solver_G2.north west);
        \end{scope}
        \node[draw,dashed,inner xsep=0.2cm,inner ysep=0.2cm,fit=(ProB) (C_Interface)] (P1){};
        \node[draw,dashed,inner xsep=0.2cm,inner ysep=0.2cm,fit=(Z3_Solver)] (P2){};
        \node[draw,dashed,inner xsep=0.2cm,inner ysep=0.2cm,fit=(Z3_Solver_G1)] (P3){};
        \node[draw,dashed,inner xsep=0.2cm,inner ysep=0.2cm,fit=(Z3_Solver_G2)] (P4){};
        \node[draw,inner xsep=0.5cm,inner ysep=0.5cm,fit=(ProB) (Z3_Solver_G1) (Z3_Solver_G2)] (RightScope){};

        \draw[->, double, thick, shorten <= 2pt, shorten >= 2pt] (LeftScope.east) -- (LeftScope-|RightScope.west);

    \end{tikzpicture}
    \caption{Die umgesetzte Architekturänderung (Die Komponenten-Entkopplung ist in den roten Boxen dargestellt. Die gestrichelten Boxen zeigen die verschiedenen Prozesse an.)}
    \label{fig:architecture}
\end{figure}
% \FloatBarrier

Diese Arbeit wird einerseits mit dem Interesse der Erweiterbarkeit verrichtet, sodass zukünftig die Option besteht,
gegebenenfalls mehrere Instanzen des Z3-Prozesses zu starten und das Lösen der Prädikate zu parallelisieren.
Andererseits dient die Entkopplung selbst bereits zur Verbesserung der Stabilität und Zuverlässigkeit von ProB,
da bei eventuellen Fehlern im Z3-Solver-Prozess dieser unabhängig von ProB neu gestartet werden kann,
was zu einem robusteren Gesamtsystem führt.

Die genauen Technologien und Konzepte, die hierfür zum Einsatz kommen und Relevanz zeigen,
sowie ihre Funktionsweise und Vorteile werden im folgenden Kapitel detailliert erläutert.
Daraufhin wird die Planung und Implementierung der neuen Architektur beschrieben
und die Leistungsfähigkeit der vorgenommenen Entkopplung anhand von Benchmarks und Tests hinsichtlich des Vergleichs zur vorherigen Systemstruktur evaluiert.
Zuletzt wird auf zukünftige Erweiterungen und Verbesserungen eingegangen
und eine abschließende Konklusion genannt.




\section{Grundlagen}

Zur Förderung eines einheitlichen Verständnisses werden in diesem Abschnitt zunächst die erforderlichen Hintergrundinformationen illustriert.
Im Folgenden werden die drei zentralen Konzepte behandelt, die für das Verständnis dieser Arbeit von Bedeutung sind: ProB, Z3 und ZeroMQ.

\subsection{ProB}

Die B-Methode, entwickelt von J.-R. Abrial \cite{abrial1996b}, ist eine formale Methode zur Entwicklung von Softwaresystemen,
die auf der Idee der abstrakten Maschinen basiert.
Mit abstrakten Maschinen lassen sich Zustände und deren Veränderungen mithilfe mathematischer Konzepte wie Mengen,
Relationen und Funktionen modellieren \cite{leuschel2003prob}.
Durch sogenannte Verfeinerungen wird schrittweise von einer abstrakten Beschreibung zu einer konkreten Implementierung übergegangen.
Dabei stellt die Methode sicher, dass Invarianten stets eingehalten werden
um die Korrektheit des Systems zu garantieren.

Die an der HHU am Lehrstuhl der Softwaretechnik und Programmiersprachen entwickelte Software ProB \cite{leuschel2003prob} ist ein Validierungs-Toolset für Modelle der B-Methode.
Als solcher unterstützt ProB mitunter die Modellierung, Animation und Verifikation von B-Modellen,
indem Funktionalitäten wie Consistency Checking und Constraint Solving bereitgestellt werden.

Der Animator in ProB ermöglicht es formale Spezifikationen zu visualisieren und zu animieren.
Nutzer können durch die Simulation in Echtzeit einen Einblick in die Zustandsübergänge einer Maschine erhalten und schrittweise die Veränderungen nachvollziehen.
Der aktuelle Zustand der Maschine wird dabei in einer grafischen Benutzeroberfläche dargestellt.

Ein weiterer Kernbestandteil von ProB ist das Consistency Checking, welches in zwei Ansätzen realisiert wird: Temporal Model Checking und Constraint-Based Checking.

Beim Temporal Model Checking wird versucht, eine Sequenz von Operationen zu finden, die, ausgehend von einem Anfangszustand,
zu einer Verletzung der Invariante, oder einem anderen Fehler führt.
Im Gegensatz dazu fokussiert sich das Constraint-based Checking auf die Suche nach einem Zustand des Systems,
der die Invariante noch erfüllt. Von dort aus wird geprüft, ob es eine einzelne Operation gibt,
welche die Invariante verletzt oder einen anderen Fehler erzeugt.

Während das Model Checking eine umfassende Exploration aller Zustände ermöglicht, ist das Constraint-based Checking spezifischer,
da es nur auf Fehler bei einzelnen Operationen fokussiert ist.
Zusammen lassen sich so vollständige Fehler und problematische Operationen identifizieren.

Beide Ansätze bieten wertvolle Instrumente für die Konsistenzprüfung von B-Modellen, und sind in der Lage die Verletzung von Invarianten und daraus folgenden Bedingungen sowie
die Abwesenheit von Deadlocks und das Erreichen von spezifizierten Zielprädikaten zu überprüfen \cite{leuschel2008prob}.

Zuletzt bietet ProB auch eine Constraint-Solving-Funktionalität, die es ermöglicht, unter Berücksichtigung von gegebenen Constraints (Einschränkungen) Lösungen für spezifische Prädikate zu finden.
Derartige Einschränkungen oder Bedingungen können in Form von logischen Ausdrücken oder Gleichungen gegeben sein, die es zu erfüllen gilt.
Ein Constraint-Solver ist ein Algorithmus oder System, welches darauf abzielt unter Berücksichtigung eben jener Beindungen eine Belegung aller Variablen zu finden, die die gegebenen Prädikate erfüllt,
und somit ein Problem auf dessen Erfüllbarkeit zu prüfen.
ProB implementiert hierfür verschiedene Constraint-Solving-Strategien, die auf unterschiedlichen Algorithmen basieren und es ermöglichen, Prädikate effizient zu lösen.
Einerseits wird CLP(FD)\footnote{Constraint Logic Programming over Finite Domains} verwendet, um auf endlichen Domänen beispielsweise Gleichheits- und Ungleichheitsbedingungen, sowie arithmetische Relationen zu lösen.
Ein weiterer Ansatz ist die Integration des SAT-basierten Kodkod,
einem effizienten Constraint-Solver für die Prädikatenlogik erster Ordnung mit Relationen, transitiven Hüllen, Bit-Vektor-Arithmetik und partiellen Modellen \cite{torlak2007kodkod}.
Zuletzt wird auch der SMT-Solver Z3 in ProB integriert, um komplexere Prädikate zu lösen, die über simple boolsche und arithmetische Ausdrücke hinausgehen.

ProB ist im Kern in SICStus Prolog \cite{carlsson1988sicstus} implementiert, bietet jedoch verschiedene Programmerweiterungen, welche zumeist in C oder C++ geschrieben sind.
Einer dieser Erweiterungen ist das Z3-Interface, welches die Integration des Z3-Solvers in ProB ermöglicht.

\subsection{Z3 Solver}
\cite{10.1007/978-3-031-65627-9_2} \cite{10.1007/978-3-540-78800-3_24}





\subsection{ZeroMQ}
\label{sec:zeromq}
\cite{hintjens2013zeromq} \cite{sustrik2015zeromq}





\section{Architekturänderung}
\todo[]{Eigentlicher Kern der Arbeit, mache ich, sobald ich mit Prokrastination fertig bin.}

\subsection{Planung}




\subsubsection{Prolog Datentypen}

- atoms string
- integers longs (laut dokumentation bis version blabla)
- floats doubles
- typerefs problem, weil kann alles sein

\subsubsection{Struktur der Nachrichten}

- function identifier
- status identifier
- message

\subsubsection{Server Struktur}

long damn switch case
threading





\subsection{Implementierung}

\subsubsection{Interfacefunktionen}

porting of all 53 interface function



\subsubsection{Hilfsfunktionen}

inbesondere $mk_type$
statemachines




\subsubsection{Optimierungen}
\label{subsec:optimizations}

Nach der erfolgreichen Portierung der Schnittstelle und der Implementierung aller notwendigen Funktionen
lassen sich zusätzlich kleine Optimierungen und Verbesserungen vornehmen, um Effizienz, Lesbarkeit und Wartbarkeit des Codes zu erhöhen.

Eine der elementarsten Optimierungsmöglichkeiten ist die Vermeidung von Nachrichtenaustausch beider Prozesse
innerhalb von Schleifen. Dieses Verhalten tritt insbesondere dann auf, wenn Prolog Datenstrukturen übermittelt werden,
die iterierbar sind, wie Listen, Vektoren und Records.
In diesen Fällen wird für jedes Element der Struktur eine Nachricht an den Server gesendet,
um das aktuelle Element zu übermitteln.
Dieses Verhalten kann in einzelnen Fällen durch die Übermittlung der gesamten Struktur in einer einzigen Nachricht vermieden werden.
In dem Folgenden \cref{lst:loops-optimization} ist gezeigt, wie eine Prolog Liste zunächst in einem String Vektor akkumuliert wird,
um anschließend in einer einzigen Nachricht an den Server gesendet zu werden.

\begin{lstlisting}[
    float, caption={Ein Ausschnitt einer Interfacefunktion zur Demonstration von Schleifenoptimierung.}, label={lst:loops-optimization}, language=C++
  ]
  std::vector<std::string> string_vec = 
    term_ref_prolog_list_to_string_vector(element_names);
  for (int i = 0; i < cardinality; i++) {
      zmsg_addstr(request, string_vec[i].c_str());
  }
  // send request to server
\end{lstlisting}

Eine weitere Optimierungsmöglichkeit ist die Vermeidung von unnötigen Nachrichten an den Server.
Beispielsweise illustriert \cref{lst:unnecessary-ctx-data} vermeidbare Komplexität durch die Verwendung unnötiger Datenobjekte.
Anhand der Variablen $translation\_type\_atom$ wird das Objekt $ctx\_data$ ermittelt, welches an die Funktion $prolog\_type\_list\_to\_sort\_vector$ übergeben wird.
Diese verwendet das Objekt ausschließlich zur Ermittlung der Variablen $translation\_type\_atom$. Das Problem hierbei ist,
dass $ctx\_data$ in der neuen Architektur auf der Seite des Server-Prozesses liegt und somit einen Nachrichtenaustausch erfordert.
Das Problem lässt sich umgehen, indem die Funktion $prolog\_type\_list\_to\_sort\_vector$ dahingehen refaktorisiert wird,
direkt mit $translation\_type\_atom$ aufgerufen zu werden.

\begin{lstlisting}[
    float, caption={Ein Ausschnitt einer redundanten Objektverwaltung.}, label={lst:unnecessary-ctx-data}, language=C++
  ]
  // function: mk_op_comprehension_set_multi
  ContextData ctx_dta =
    get_translation_representant_ctx_data(translation_type_atom);
  prolog_type_list_to_sort_vector(ctx_data, couple_types);
  // ...
  
  // function: prolog_type_list_to_sort_vector
  mk_sort(ctx_data->get_translation_type_atom());
  
\end{lstlisting}

\begin{lstlisting}[
    float, caption={Die Hilfsfunktion $escape\_string$.}, label={lst:escape-string}, language=C++
  ]
  std::string escape_string(const std::string to_escape) {
    return "|" + to_escape + "|";
  }
\end{lstlisting}

Zuletzt wurde das DRY\footnote{Don't Repeat Yourself}-Prinzip angewendet, um die Wartbarkeit des Codes zu erhöhen,
indem gewisse Hilfsfunktionen ausschließlich auf entweder der Prolog- oder der Serverseite implementiert wurden.
Der ursprüngliche Kontrollfluss verlangte zum Beispiel die Implementierung von der in \cref{lst:escape-string} gezeigten Funktion
in beiden Prozessen. Durch die Anwendung des Programmierprinzips wurde die Funktion ausschließlich an dem Z3-Prozess implementiert.


\subsubsection{Serveranbindung}

server als subprozess
starting as needed

\subsubsection{Logging}

via sys argv
stdout not captureable in sicstus prolog







\section{Zusätzliche Ergebnisse}



\subsection{Softlock}
\label{subsec:softlock}

endless loop fixed by interrupting on every reset


\subsection{Versionsinkompatibilität}

makefile hell
glibc (OS) incompatible with z3lib.so -> darwin12


\section{Leistungsbewertung}
\label{sec:performance-evaluation}

Nach Abschluss der durchgeführten Architekturänderung ist es elementar, die Auswirkungen auf die Laufzeitperformance zu bewerten.
Da sich die grundlegende Funktionsweise des Z3-Solvers in der neuen Architektur nicht geändert hat,
ist es zu erwarten, dass die Laufzeitperformance der ProB-Systemerweiterung durch die Einführung der ZeroMQ-Kommunikation negativ beeinflusst wird.
Zusätzlich zu den Aufrufen des Z3-Interfaces müssen zur Lösung eines einzelnen Prädikates nun mehrere Anfragen und Antworten über den Socket serialisiert werden.
Diese zusätzliche Kommunikation führt zu einem Performance-Overhead, welcher quantifiziert und evaluiert werden muss.
Ebenfalls besteht ein Interesse zum Vergleich verschiedener ZeroMQ-Protokolle und deren Auswirkungen auf die Performance.
Hierbei ist zu erwarten, dass das Inter-Process-Communication (IPC) Protokoll schneller ist als das Transmission-Control-Protocol (TCP) Protokoll,
da es auf dem gleichen Rechner arbeitet und keine Netzwerkkommunikation benötigt, sondern das Dateisystem verwendet.
Im nachfolgenden Abschnitt werden die Methodik der Leistungsbewertung,
die erzielten Ergebnisse und ihre Interpretation detailliert beschrieben.
\clearpage
\subsection{Performance-Overhead}

Um eine Bewertung des Performance-Overheads zu ermöglichen, müssen zunächst empirische Daten erhoben werden.
Hierzu werden die Tests zur Verifikation der Funktionsweise des Z3-Interfaces umfunktioniert, um die Laufzeit der einzelnen Anfragen zu messen.
Im Code des Z3-Interfaces wird ein Zeitstempel bei Beginn und Ende des Lösungsvorgangs eines Prädikates gesetzt, dessen Differenz berechnet und gespeichert.
Insgesamt stehen 53 Tests zur Verfügung, die in der Testumgebung des Z3-Solvers ausgeführt werden können.
Diese Tests umfassen insgesamt 679 Prädikate, welche eine ausreichende Grundgesamtheit zur Bewertung der Performance bieten.
Ebenfalls wird die Anzahl der Anfragen und Antworten, die über das Netzwerk gesendet werden, gemessen.
Ein kleiner Auszug dieser Messdaten sind in \cref{tab:performance-data} dargestellt.

\begin{table}[!htp]
    \centering
    \caption{Auszug der Daten der Performance-Messung.}
    \label{tab:performance-data}
    \resizebox{\textwidth}{!}{ % Adjust table size to fit the page
        \begin{tabular}{ cccccc }
            \toprule
            \textbf{TestID} & \textbf{QueryID} & \textbf{Old(ns)} & \textbf{New(IPC, ns)} & \textbf{New(TCP, ns)} & \textbf{Req. Count} \\
            \midrule
            \noalign{\vskip -1mm}
            \vdots          & \vdots           & \vdots           & \vdots                & \vdots                & \vdots              \\
            \noalign{\vskip -1mm}
            1510            & 1                & \num{114815014}  & \num{283392117}       & \num{113503997}       & 191                 \\
            1510            & 2                & \num{52048678}   & \num{59273375}        & \num{44489012}        & 166                 \\
            1510            & 3                & \num{30853103}   & \num{24983820}        & \num{25212332}        & 286                 \\
            1511            & 1                & \num{69240686}   & \num{199325313}       & \num{61823314}        & 21                  \\
            1511            & 2                & \num{61404523}   & \num{62220124}        & \num{48522297}        & 20                  \\
            1513            & 1                & \num{80829324}   & \num{202694845}       & \num{75877790}        & 25                  \\
            \noalign{\vskip -1mm}
            \vdots & \vdots & \vdots    & \vdots    & \vdots    & \vdots \\
            \noalign{\vskip -1mm}
            \bottomrule
        \end{tabular}
    } % End of resizebox
\end{table}
% \FloatBarrier

Die Zeitmessungen sind in Nanosekunden (ns) angegeben und zeigen die Laufzeit der Anfragen in den verschiedenen Konfigurationen.
Alle Messwerte liegen in einer Größenordnung von mindestens Millisekunden.
Damit sind die zu erwartenden Messfehler in Bereich von Nanosekunden gering genug, um die Daten adäquat analysieren zu können.
Innerhalb der eigentlichen Daten wurden die Messungen mehrfach unabhängig voneinander wiederholt, um eine statistische Aussagekraft zu gewährleisten\footnote{Über alle Systemvarianten hinweg wurden insgesamt 13 Messreihen durchgeführt.}.
Da dennoch von Messunsicherheiten und Schwankungen auszugehen ist, werden die Daten in einem statistischen Kontext betrachtet.
Es wird angenommen, dass sowohl die tatsächlichen Laufzeiten innerhalb eines Tests als auch deren Messunsicherheiten normalverteilt sind.

Um sich zunächst einen Überblick über die Rohdaten zu verschaffen,
werden in \cref{fig:performance-overview} die durchschnittlichen Laufzeiten in den verschiedenen Konfigurationen dargestellt.
Die obere Subgrafik zeigt die durchschnittlichen Laufzeiten aller Anfragen in sowohl der alten als auch der neuen Architektur,
wobei die neue Architektur in das IPC-Protokoll und TCP-Protokoll unterteilt ist.
Wider Erwarten zeigt sich, dass die durchschnittliche Laufzeit in der neuen Architektur bei vielen Prädikaten geringer ausfällt als in der alten Architektur.
Insgesamt sind von den 679 Prädikaten nur 172 Prädikate in der alten Architektur schneller gelöst worden.
265 Prädikate wurden in der IPC-Konfiguration und 242 Prädikate in der TCP-Konfiguration am schnellsten gelöst.
\FloatBarrier
\begin{figure}[!ht]
    \centering
    \includegraphics[scale=.55]{./PerformanceEvaluation/processingtime.pdf}
    \caption{Durchschnittliche Laufzeiten der Anfragen in den verschiedenen Architekturen.}
    \label{fig:performance-overview}
\end{figure}
% \FloatBarrier

Die unteren beiden Subgrafen stellen interessante Bereiche der Rohdaten erneut dar und zeigen zusätzlich
die Standardfehler der Mittelwerte in Form der Fehlerbalken. Diese geben an, wie sehr die
Mittelwerte der Laufzeiten von den tatsächlichen Mittelwerten der Grundgesamtheit abweichen.
Kalkuliert wird dieser Standardfehler mittels der folgenden Formel:
\begin{equation}
    \sigma_{\overline{x}} = \frac{\sigma}{\sqrt{n}}
\end{equation}
Hierbei ist $\sigma_{\overline{x}}$ der Standardfehler des Mittelwertes, $\sigma$ die Standardabweichung der Grundgesamtheit und $n$ die Anzahl der Messungen.
Somit werden ausschließlich statistische Fehler betrachtet und systematische Fehler nicht berücksichtigt.

So zeigt der untere linke Subgraf die Rohdaten im Bereich der Prädikate 60 bis 115 erneut,
wobei klar zu erkennen ist, dass hier die neue Server-Architektur konstant schneller ist.
Dieser Trend ist über die gesamte Messung hin unterhalb einer gewissen Grenze zu beobachten.
Ebenfalls ersichtlich ist der nur minimal ausfallende Unterschied zwischen IPC und TCP.

Auf der anderen Seite zeigt der untere rechte Subgraf die Rohdaten im Bereich der Prädikate 555 bis 610.
Hier haben die Laufzeit Messungen einen vergleichsweise hohen Wert und es ist zu erkennen,
dass die alte Architektur in diesem Bereich schneller ist.

Entsprechend wurde durch die Protokollkommunikation ein Performance-Overhead eingeführt, der sich in den Rohdaten dahingehend widerspiegelt,
dass Prädikate, welche bereits eine hohe Laufzeit aufgewiesen haben, nun in der neuen Architektur noch langsamer gelöst werden.
Neben diesem Overhead wurde jedoch eine signifikante Laufzeitverbesserung erworben, die bis zu einer gewissen Gesamtlaufzeitgrenze den
Overhead nicht nur annulliert, sondern überkompensiert.
Insgesamt scheint die neu eingeführte Server-Architektur hinsichtlich der Laufzeitperformance um einen gewissen Faktor effizienter geworden zu sein.

Ebenfalls ist zu erkennen, dass der Unterschied zwischen IPC und TCP nur minimal ist und keine signifikanten Unterschiede aufweist.
Innerhalb der 679 Testprädikate ist IPC in 365 ($53.76\%$) Fällen schneller als TCP und entsprechend TCP in 314 ($46.24\%$) Fällen schneller als IPC.
Da zur TCP-Kommunikation die Adresse \texttt{tcp://127.0.0.1} verwendet wurde, ist eine mögliche Erklärung hierfür, dass die Kommunikation über das Loopback-Interface
einen Großteil des Netzwerk-Stacks umgeht und vom Betriebssystem speziell optimiert wird.
In der weiteren Analyse wird daher nur das IPC-Protokoll betrachtet.

% \clearpage

Um den Overhead der neuen Architektur zu quantifizieren,
wird die Differenz der durchschnittlichen Laufzeiten der neuen ($\overline{t_\text{IPC}}$) und alten ($\overline{t_\text{Old}}$) Architektur berechnet
und gegen die Anzahl der ZeroMQ-Anfragen analysiert.
Der induzierte Overhead $t_\text{induced}$ wird also wie folgt definiert:

\begin{equation}
    t_\text{induced} = \overline{t_\text{IPC}} - \overline{t_\text{Old}}
\end{equation}

Der oberste Subgraf in \cref{fig:overhead} zeigt den induzierten Overhead der neuen Architektur in Abhängigkeit der Anzahl der ZeroMQ-Anfragen auf einer logarithmischen Skala.
Im groben Verlauf ist zu erkennen, dass der Overhead mit steigender Anzahl der Anfragen annähernd linear zunimmt.
Zusätzlich scheint jedoch eine ausgeprägte Systematik vorzuliegen, welche sich inhaltlich durch die horizontale Verzerrung der Daten ausprägt.
Einige wenige Datenpunkte weichen stark von jeglicher Systematik ab und sind als Ausreißer zu klassifizieren.

Im zweiten Subgrafen zum induzierten Overhead wird die lineare Abhängigkeit des Overheads von der Anzahl der Anfragen untersucht, indem der Zusammenhang linear gefittet\footnote{Der lineare Fit wurde mit dem von scipi.optimize bereitgestellten curve\_fit berechnet.} wird.
Dies geschieht unter Berücksichtigung des kombinierten Standardfehlers der Mittelwerte der alten und neuen IPC-Architektur.
Die Unsicherheitsberechnung wird berechnet mit der gaußschen Fehlerfortpflanzung

\begin{equation}
    \sigma_{\overline{comb}} = \sqrt{\sigma_{\overline{IPC}}^2 + \sigma_{\overline{Old}}^2} \hspace{0.3cm},
\end{equation}

welche ebenfalls in den Fehlerbalken dargestellt wird.
Das Ergebnis des linearen Fits zeigt eine Steigung von $0.03 ms$ pro Anfrage, was bedeutet, dass der Overhead pro Anfrage um $0.03 ms$ steigt.
Zusätzlich gibt es einen y-Achsenabschnitt von $-13.13 ms$, was bedeutet, dass der Overhead bei $0$ Anfragen $-13.13 ms$ beträgt.
Dieser Wert beschreibt die zuvor erfasste Laufzeitverbesserung der neuen Architektur.
Insgesamt ergibt sich die folgende lineare Funktion zur Beschreibung des Overheads:

\begin{equation}
    Overhead = 0.03ms \cdot Anfragenanzahl - 13.13ms
\end{equation}

Durch das Kalkulieren der Nullstellen dieser Overhead-Funktion ergibt sich eine Anzahl an Anfragen von etwa $438$, bis der Overhead nicht länger von der Laufzeitverbesserung kompensiert werden kann.

Eine separate Messreihe wurde durchgeführt, um die durchschnittliche Laufzeit einer einzelnen Anfrage an den Z3-Server zu ermitteln.
Hierbei wurde die durchschnittliche Laufzeit von $0.0054 ms$ bestimmt.
Da für jede Anfrage zwei Nachrichten (Anfrage und Antwort) über den Socket gesendet werden, ergibt sich eine Laufzeit von $0.0216 ms$ pro Anfrage
unter der Annahme, dass das Erhalten einer ZeroMQ-Nachricht die gleiche Laufzeit aufweist wie das Senden einer Nachricht.
Die Differenz von $0.0084 ms$ zur ermittelten Steigung des linearen Fits könnte beispielsweise auf die Größe der Nachrichten zurückzuführen, welche
in der kontrollierten Messreihe einer einzelnen Anfrage kleiner ausfällt als in den 52 Testszenarien.
Zudem sind Messwerte dieser Größenordnung zunehmend ungenau, weshalb sie annehmbar plausibel bezüglich der Steigung des linearen Fits sind.


\begin{figure}[!ht]
    \centering
    \includegraphics[scale=.55]{./PerformanceEvaluation/overhead.pdf}
    \caption{Induzierter Overhead der neuen Server-Architektur durch das Serialisieren auf den Socket.}
    \label{fig:overhead}
\end{figure}
% \FloatBarrier
% \clearpage

Der lineare Fit zeigt augenscheinlich eine gute Übereinstimmung mit den Daten, welche in dem dritten Subgrafen aus \cref{fig:overhead} verifiziert wird.
Hierbei wird die Differenz des induzierten Overheads und des linearen Fits gegen die Anzahl der Anfragen dargestellt.
Es ergibt sich ein Residuenplot.
In diesem Plot ist die Systematik des obersten Subgrafen erneut zu erkennen.
Sie bildet ein kleines Cluster oberhalb der $0$-Linie, bei sehr kleinen Anfragenzahlen.
Es stellt sich heraus, dass innerhalb dieses Datenclusters alle Datenpunkte die QueryID $1$ aufweisen.
Somit lässt sich die Abweichung durch das Initialisieren des Sockets, der Verbindung des Z3-Servers und dessen initialen Overheads zum Aufsetzten der Z3-Konfigurationen erklären.
Innerhalb einer TestID wird der Z3-Server nur einmalig initialisiert und folgende Prädikate desselben Tests nutzen denselben Z3-Server, indem sie die Konfiguration zurücksetzten und nicht grundlegend neu initialisieren.

Zur Überprüfung stellt die letzte Subgrafik erneut den Residuenplot dar, jedoch ohne diejenigen Datenpunkte mit QueryID $1$.
Hierbei ist zu erkennen, dass die Systematik des vorherigen Grafen verschwindet und der lineare Fit eine gute Übereinstimmung mit den Daten aufweist.
Somit ist gezeigt, dass der Overhead der neuen Architektur durch das Serialisieren auf den Socket annähernd linear mit der Anzahl der Anfragen zunimmt
und keine weiteren relevanten Systematiken besitzt. Eine Voraussage des Overheads ist mithilfe des linearen Fits möglich.
Es ist zu beachten, dass die absoluten Werte des Overheads nur in der gegebenen Testumgebung gültig sind und nicht auf andere Umgebungen übertragen werden können.

Ein einzelner Datenpunkt scheint eine unwahrscheinlich hohe Laufzeitverbesserung von durchschnittlich $2$ Sekunden aufzuweisen, weist jedoch hohe Messunsicherheiten auf.
In \cref{tab:broken-datapoint} ist dieser Datenpunkt dargestellt.

\begin{table}[!htp]
    \centering
    \caption{Ausschnitt der gesammelten Messwerte eines Ausreißers.}
    \label{tab:broken-datapoint}
    \begin{tabular}{ cccccc }
        \toprule
        \textbf{TestID-QueryID} & \textbf{$Old_1(ms)$} & \textbf{$Old_2(ms)$} & \textbf{$Old_3(ms)$} & \textbf{$Old_4(ms)$} & \textbf{Req. Count} \\
        \midrule
        2122-90                 & 36                   & 4033                 & 3993                 & 43                   & 119                 \\
        \bottomrule
    \end{tabular}
\end{table}
% \FloatBarrier

Die verschiedenen Messwerte der alten Architektur in Millisekunden zeigen zwei starke Ausreißer bei der zweiten und dritten Messung.
Diese Varianz ist womöglich auf die in \cref{subsec:softlock} beschriebene Problematik der Endlosschleife zurückzuführen.
In jedem Fall ist dieser Datenpunkt als individueller Ausreißer zu betrachten und weist keine besondere statistische Relevanz auf.

Auch wenn die ermittelte Performance-Verbesserung eine interessante Erkenntnis darstellt,
wird sie an dieser Stelle nicht weiter analysiert, da sie nicht Gegenstand dieser Arbeit ist.
Ein möglicher Grund für die Verbesserung könnte die Kompilierung sein, die durch die neue Architektur ermöglicht wird,
da der Z3-Solver als eigenständiger Prozess womöglich besser vom Compiler optimiert werden kann.
Als eigenständiger Prozess ist es ebenfalls möglich, dass der runtime linker die Z3-Bibliothek optimierter laden und ausführen kann.
Zusätzlich könnten die in \cref{subsec:optimizations} beschriebenen Optimierungen eine Rolle spielen.




\section{Zukünftige Arbeiten}


\subsection{Weitere Analyse und Optimierungen}

    analyse performance improvement

\subsection{Deinit Hook}

 deinit für dangling zocket und prozess (cleaner)

\subsection{Parallelisierung}

 threading in ProB (eigentlicher Sinn der Arbeit um zu parallelisieren)




\section{Konklusion}

Zusammenfassend lässt sich festhalten, dass die Restrukturierung des bestehenden Systems erfolgreich umgesetzt werden konnte.
Die Z3-Komponente wurde gänzlich von der Hauptanwendung ProB entkoppelt und in einen seperaten Prozess ausgelagert.
Durch die Einführung einer Kommunikationsstruktur ist die Interaktion zwischen den Komponenten gewährleistet.
Die durchgeführten Tests bestätigen die vollständige Funktionalität des Systems nach der durchgeführten Architekturänderung,
da alle Tests der relevanten Testkategorien fehlerfrei bestanden worden sind.

Durch die Entkopplung der Komponenten konnte die Wartbarkeit und Erweiterbarkeit des Systems verbessert werden.
Insbesondere behebt die neue Architektur bereits eine Vielzahl von potenziellen Problemen,
wobei speziell die Überarbeitung des Exception-Handlings bei der Verwendung des Z3-Solvers hervorzuheben ist.

Während der gesamten Implementierung wurde Rücksicht genommen auf zukünftige Erweiterungen und Verbesserungen des Gesamtsystems,
sodass zukünftig die Option besteht, das Z3-Interface dahingehend zu erweitern, dass gegebenenfalls mehrere Instanzen des Z3-Prozesses gestartet werden und das Lösen der Prädikate parallelisiert wird.

Eine überraschende Erkenntnis war die erkennbare Laufzeitverbesserung durch die neue Architektur.
Die Entkopplung und Optimierung der Komponenten führte
zu einer messbaren Effizienzsteigerung bei dem Lösen von Prädikaten.

Aufgrund der gelungenen Umsetzung der Architekturänderung und der positiven Ergebnisse aller durchgeführten Tests
sowie der erzielten Erweiterbarkeit hat die Arbeit ihre angestrebten Ziele erreicht und alle gesetzten Anforderungen erfüllt.


%% Dieser Part kann auskommentiert werden, sollte kein Anhang nötig sein.
%% Der \appendix-Befehl leitet hierbei den Anhang ein.
%  \appendix
%  \input{appendix.tex}

%%%%%%%%%%%%%%%%%%%%%%%%%%%%%%%%%%%%%%%%%%%%%%%%%%%%%%%%%%%%%%%%%%%%%%%%%%%%%%%%
%% (Ende) Der Inhalt der Arbeit                                               %%
%%%%%%%%%%%%%%%%%%%%%%%%%%%%%%%%%%%%%%%%%%%%%%%%%%%%%%%%%%%%%%%%%%%%%%%%%%%%%%%%


\backmatter

\clearpage
\bibliography{references}
%% Depending on Language, use german alphadin or original alpha
\iflanguage{ngerman}{
  \bibliographystyle{alphadin}
}{
  \bibliographystyle{alpha}
}

\end{document}
