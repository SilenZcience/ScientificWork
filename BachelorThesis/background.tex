
\section{Grundlagen}

Zur Förderung eines einheitlichen Verständnisses werden in diesem Abschnitt zunächst die erforderlichen Hintergrundinformationen illustriert.
Im Folgenden werden die drei zentralen Konzepte behandelt, die für das Verständnis dieser Arbeit von Bedeutung sind: ProB, Z3 und ZeroMQ.

\subsection{ProB}
\todo[]{Relevanz für die Arbeit suchen? Wo ist sie? Man weiß es nicht. Aber irgendwas muss man ja schreiben...}
\todo[]{Inhaltliche Korrektheit prüfen. Nach A kommt B und was ist das?}
Die B-Methode, entwickelt von J.-R. Abrial \cite{abrial1996b}, ist eine formale Methode zur Entwicklung von Softwaresystemen,
die auf der Idee der abstrakten Maschinen basiert.
Mit abstrakten Maschinen lassen sich Zustände und deren Veränderungen mithilfe mathematischer Konzepte wie Mengen,
Relationen und Funktionen modellieren \cite{leuschel2003prob}.
Durch sogenannte Verfeinerungen wird schrittweise von einer abstrakten Beschreibung zu einer konkreten Implementierung übergegangen.
Dabei stellt die Methode sicher, dass Invarianten stets eingehalten werden,
um die Korrektheit des Systems zu garantieren.

Die an der HHU am Lehrstuhl der Softwaretechnik und Programmiersprachen entwickelte Software ProB \cite{leuschel2003prob} ist ein Validierungstoolset für Modelle der B-Methode.
Als solcher unterstützt ProB mitunter die Modellierung, Animation und Verifikation von B-Modellen,
indem Funktionalitäten wie Consistency Checking und Constraint Solving bereitgestellt werden.
ProB findet bereits in vielen Systemen Verwendung zur Datenvalidierung und Validierung komplexer Eigenschaften für sicherheitskritische Systeme.
Es wird bereits von mehreren Unternehmen verwendet und ist November 2022 mit dem \enquote{AlainColmerauer Prize} ausgezeichnet worden.

Der Animator in ProB ermöglicht es, formale Spezifikationen zu visualisieren und zu animieren.
Nutzer können durch die Simulation in Echtzeit einen Einblick in die Zustandsübergänge einer Maschine erhalten und schrittweise die Veränderungen nachvollziehen.
Der aktuelle Zustand der Maschine wird dabei in einer grafischen Benutzeroberfläche dargestellt.

Ein weiterer Kernbestandteil von ProB ist das Consistency Checking, welches in zwei Ansätzen realisiert wird: Temporal Model Checking und Constraint-Based Checking.

Beim Temporal Model Checking wird versucht, eine Sequenz von Operationen zu finden, die ausgehend von einem Anfangszustand
zu einer Verletzung der Invariante oder einem anderen Fehler führt.
Im Gegensatz dazu fokussiert sich das Constraint-based Checking auf die Suche nach einem Zustand des Systems,
der die Invariante noch erfüllt. Von dort aus wird geprüft, ob es eine einzelne Operation gibt,
welche die Invariante verletzt oder einen anderen Fehler erzeugt.

Während das Model Checking eine umfassende Exploration aller Zustände ermöglicht, ist das Constraint-based Checking spezifischer,
da es nur auf Fehler bei einzelnen Operationen fokussiert ist.
Zusammen lassen sich so vollständige Fehler und problematische Operationen identifizieren.

Beide Ansätze bieten wertvolle Instrumente für die Konsistenzprüfung von B-Modellen, und sind in der Lage, die Verletzung von Invarianten und daraus folgenden Bedingungen sowie
die Abwesenheit von Deadlocks und das Erreichen von spezifizierten Zielprädikaten zu überprüfen \cite{leuschel2008prob}.

Zuletzt bietet ProB auch eine Constraint-Solving-Funktionalität, die es ermöglicht, unter Berücksichtigung von gegebenen Constraints (Einschränkungen) Lösungen für spezifische Prädikate zu finden.
Derartige Einschränkungen oder Bedingungen können in Form von logischen Ausdrücken oder Gleichungen gegeben sein, die es zu erfüllen gilt.
Ein Constraint-Solver ist ein Algorithmus oder System, welches darauf abzielt, unter Berücksichtigung eben jener Bedingungen eine Belegung aller Variablen zu finden, die die gegebenen Prädikate erfüllt
und somit ein Problem auf dessen Erfüllbarkeit zu prüfen.
ProB implementiert hierfür verschiedene Constraint-Solving-Strategien, die auf unterschiedlichen Algorithmen basieren und es ermöglichen, Prädikate effizient zu lösen.
Einerseits wird CLP(FD)\footnote{Constraint Logic Programming over Finite Domains} verwendet, um auf endlichen Domänen beispielsweise Gleichheits- und Ungleichheitsbedingungen sowie arithmetische Relationen zu lösen.
Ein weiterer Ansatz ist die Integration des SAT-basierten Kodkod,
einem effizienten Constraint-Solver für die Prädikatenlogik erster Ordnung mit Relationen, transitiven Hüllen, Bit-Vektor-Arithmetik und partiellen Modellen \cite{torlak2007kodkod}.
Zuletzt wird auch der SMT-Solver Z3 in ProB integriert, um komplexere Prädikate zu lösen, die über simple boolsche und arithmetische Ausdrücke hinausgehen.

ProB ist im Kern in SICStus Prolog \cite{carlsson1988sicstus} implementiert, bietet jedoch verschiedene Programmerweiterungen, welche zumeist in C oder C\texttt{++} geschrieben sind.
Einer dieser Erweiterungen ist das Z3-Interface, welches die Integration des Z3-Solvers in ProB ermöglicht.

\subsection{Z3 Solver}

Das Satisfiability Modulo Theories (SMT)-Problem lässt sich als eine natürliche Erweiterung des klassischen Boolean Satisfiability (SAT)-Problems verstehen.
Während SAT sich ausschließlich mit der Erfüllbarkeit von booleschen Formeln beschäftigt,
umfasst SMT zusätzliche Hintergrundtheorien wie Arithmetik, Bit-Vektoren, Arrays und uninterpretierten Funktionen.
Diese Erweiterung ermöglicht eine umfassendere Form der logischen Schlussfolgerung.

Der Z3-Solver \cite{10.1007/978-3-540-78800-3_24} wurde von Microsoft Research entwickelt.
Seit seiner ersten Veröffentlichung hat sich Z3 zu einem der leistungsfähigsten und am weitesten verbreiteten SMT-Solver entwickelt.
Microsoft hat Z3 als Open-Source-Software frei zugänglich unter der MIT-Lizenz veröffentlicht,
was seine Verbreitung und Nutzung in verschiedenen Bereichen weiter gefördert hat.

Er wurde in C\texttt{++} implementiert, bietet jedoch eine Vielzahl von externen API Anbindungen, die es ermöglichen, den Solver in verschiedenen Programmiersprachen zu verwenden,
wie beispielsweise OCaml, Python, Ruby und Rust.
Er ist ein leistungsfähiger SMT-Solver, der sich speziell auf die Lösung von Problemen im Bereich der Softwareverifikation und -analyse spezialisiert.
Zum Beispiel wird der Z3-Solver in der Softwareverifikation eingesetzt, um die Korrektheit von Programmen zu überprüfen, indem er formale Spezifikationen testet und beweist.
Andere Einsatzbereiche sind die automatische Generierung von Testfällen basierend auf formalen Modellen sowie dem Modellieren von Entscheidungsproblemen und dem Abstrahieren von Prädikaten.

Der Aufbau von Z3 ist modular und basiert auf dem DPLL\footnote{Davis-Putnam-Logemann-Loveland}(T)-Framework,
das die grundlegenden Prinzipien des DPLL-Algorithmus zur Lösung des CNF-SAT-Problems um weitere Theorien (wie SMT) ergänzt.
Der DPLL ist ein Algorithmus zur Entscheidung der Erfüllbarkeit einer booleschen Aussagenlogikformel in konjunktiver Normalform (CNF).
Er arbeitet rekursiv und kombiniert folgende Schritte:

\begin{enumerate}
    \item Unit Propagation: Ableitung neuer Wahrheitswerte aus Klauseln mit nur einem Literal (Einheitsklausel).
    \item Decision Making: Entscheidung für ein Literal, um die Suchrichtung voranzutreiben.
    \item Conflict Detection/Backtracking: Zurücksetzen von Entscheidungen, wenn ein Konflikt entdeckt wird.
\end{enumerate}    

In Z3 dient der DPLL-Algorithmus als Kern des SAT-Solvers.
Weitere Theorien wie lineare und nichtlineare Arithmetik, Bit-Vektoren und Arrays werden durch spezielle Theorie-Solver behandelt,
welche als spezialisierte Module eng mit dem SAT-Solver integriert sind.

Ein zentrales Merkmal von Z3 ist seine hohe Effizienz und seine Fähigkeit,
große und komplexe Probleme in akzeptabler Zeit zu lösen.
Diese Leistungsfähigkeit wird durch verschiedene Optimierungen und Heuristiken erreicht,
die den Suchraum effizient einschränken und die Lösung von Constraints beschleunigen.
Zudem ist ein Simplifer in die System Architektur integriert, der gegebene Constraints vereinfacht.
Z3 hat nur wenige Abhängigkeiten an externe Bibliotheken und ist daher leicht zu integrieren und zu verwenden.
Es wird die C\texttt{++} Bibliothek für das Threaden der Anwendung verwendet, um die parallele Verarbeitung zu ermöglichen und damit die Leistungsfähigkeit weiter zu steigern. 


Innerhalb von ProB wird Z3 vor allem dann eingesetzt, wenn ProBs interner Constraint-Solver bei der Lösung bestimmter Probleme an seine Grenzen stößt.
Dies betrifft insbesondere große oder komplexe Constraints mit nicht-linearen Bedingungen.
Beispielsweise lässt sich die folgende Formel, welche in \cref{lst:z3-example} gezeigt wird, in Z3 aber nicht in ProB lösen:

\begin{lstlisting}[
    float, caption={Ein Beispiel zur Verwendung des Z3-Solvers innerhalb der ProB REPL.}, label={lst:z3-example}, language=bash
  ]
  >>> :z3 X<Y & Y<X & X:INTEGER
  PREDICATE is FALSE
\end{lstlisting}
\FloatBarrier


Insgesamt macht die Leistungsfähigkeit und Flexibilität des Z3-Solvers ihn zu einem wichtigen Bestandteil von ProB,
um Verifikationsaufgaben zu bewältigen, die über die Fähigkeiten des internen Constraint-Solvers hinausgehen.


\subsection{ZeroMQ}
\label{sec:zeromq}
\cite{hintjens2013zeromq} \cite{sustrik2015zeromq}
