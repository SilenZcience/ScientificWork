

\section{Zukünftige Arbeiten}

Es existieren verschiedene Möglichkeiten, um auf die in dieser Arbeit vorgestellte Implementierung aufzubauen und sie weiter zu verbessern.
Folgende Abschnitte stellen einige dieser Möglichkeiten vor.

\subsection{Weitere Analyse und Optimierungen}

Die in \cref{sec:performance-evaluation} vorgestellte Performance-Evaluation ist eine erste Analyse der Implementierung und der Auswirkungen durch die Entkopplung von ProB und Z3.
Sie stellt überraschenderweise eine Verbesserung der Performance fest, obwohl die Systemänderung an sich einen Overhead darstellt.
Um ein tieferes Verständnis über die Laufzeit zu erlangen, ist es sinnvoll, die Analyse zu erweitern und zu vertiefen.
Eine ausreichend tiefgründige Analyse könnte insofern von Vorteil sein, um ein besseres Verständnis über die Laufzeitkomponenten zu erlangen und somit gegebenenfalls gezielt weitere Optimierungen vorzunehmen.

\subsection{Deinit Hook}

Nach der erfolgreich abgeschlossenen Systemarchitekturänderung, die in dieser Arbeit vorgestellt wurde,
bleibt ein letzter Schritt offen, um die Implementierung vollständig auszubauen.
Derzeitig wird der Z3-Solver-Prozess nicht sauber beendet, sondern existiert als sogenannter \textit{child process} von ProB und ist somit an die Existenz des ProB-Prozesses gekoppelt.
Dies stellt insofern grundsätzlich kein großes Problem dar, da der Z3-Solver-Prozess nach Beendigung des ProB-Prozesses automatisch beendet wird.
Dennoch besteht in der Theorie die Möglichkeit, dass der Z3-Server als sogenannter \textit{dangling process} zurückbleibt.
Grundsätzlich ist es zudem eine sauberere Lösung, den Z3-Prozess manuell zu beenden.

Des Weiteren bleibt bei Beendigung des Solvers der von ZeroMQ instanziierte Socket geöffnet und als \textit{dangling socket} im Arbeitsspeicher zurück.
Dieser Umstand kann zu Speicherlecks führen und sollte daher vermieden werden.

Um diese Probleme zu beheben, ist es sinnvoll, die Implementierung um einen Deinit-Hook zu erweitern.
Dieser Hook wird aufgerufen, wenn der Z3-Solver beendet beziehungsweise deinitialisiert wird, und dient dazu, den Z3-Solver-Prozess und den Z3-Socket sauber zu beenden und zu schließen.

\subsection{Parallelisierung}

Die Entkopplung von ProB und Z3 dient elementar als Grundlage für die Parallelisierung des Z3-Solvers
beziehungsweise die Parallelisierung der Lösung mehrerer Prädikate.
Wie in \cref{sec:goals} beschrieben, ist die Entkopplung mit eben jenem Interesse der Erweiterbarkeit verrichtet worden.

Es bestehen verschiedene Ansätze, um die Parallelisierung zu realisieren.
Eine Möglichkeit wäre, mehrere Instanzen des Z3-Solvers zu starten und die Prädikate auf diese Instanzen zu verteilen.
In diesem Fall müsste die Kommunikationsschnittstelle auf der Seite von ProB so erweitert werden, dass sie mehrere Z3-Solver-Prozesse unterstützt.
Der Z3-Server selbst müsste in diesem Szenario nicht verändert werden.

Eine andere Möglichkeit besteht darin, auch den Z3-Server zu erweitern, sodass dieser mehrere Prädikate gleichzeitig lösen kann.
Hierbei müsste die Kommunikationsschnittstelle auf der Seite des Z3-Servers so erweitert werden, dass sie mehrere Prädikate gleichzeitig empfangen und lösen kann.
Es würde sich anbieten, in diesem Fall das Kommunikationsmustern zu wechseln und auf ein asynchrones Modell umzusteigen, wie beispielsweise das Dealer-Router Modell,
welches in \cref{sec:zeromq} bereits erwähnt wurde.
