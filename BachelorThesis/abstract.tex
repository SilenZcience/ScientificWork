
In der Welt der formalen Methoden stellen Modellierungs- und Verifikationswerkzeuge wie die Software ProB eine essenzielle Rolle dar,
um beispielsweise die Korrektheit von sicherheitskritischen Systemen zu gewährleisten.
Zur Lösung von logischen Problemen verwenden diese Werkzeuge oftmals SMT-Solver, die eine umfassende Form der logischen Schlussfolgerung ermöglichen.
So auch der Constraint-Solver und Model-Checker ProB, welcher mitunter den SMT-Solver Z3 verwendet.
Dieser wies jedoch in seiner ursprünglichen Implementierung innerhalb von ProB Instabilitäten auf, wie beispielsweise sporadische Speicherlecks und Segmentation Faults.
Um diese und weitere Problematiken zu beheben, wurde in dieser Arbeit eine neue Implementierung des Interfaces zum Z3-Solver in ProB entwickelt, welche
die vollständige Entkopplung der Z3-Komponente in einen separaten Prozess vorsah.
Die Architekturänderung konnte erfolgreich in ProB integriert werden, was somit bereits viele der bestehenden Probleme löste.
Ein wichtiges Kriterium für die neue Implementierung war es, einen Grundstein für zukünftige Erweiterungen zu legen, insbesondere der Parallelisierung des Lösens mehrerer Prädikate.
Anhand einer folgenden Leistungsanalyse konnte zudem gezeigt werden, dass die neue Implementierung des Z3-Interfaces
in ProB bereits eine messbar verbesserte Performance aufweist und somit nicht nur die Stabilität, sondern auch Effizienz des Systems erhöht hat.
