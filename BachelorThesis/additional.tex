
\section{Zusätzliche Komplikationen}

Im Verlauf der Arbeit traten einige Probleme auf, die nicht direkt mit der Entkopplung der Z3-Komponente zusammenhängen. Diese werden im Folgenden kurz erläutert.

\subsection{Softlock}
\label{subsec:softlock}

Bei der exzessiven Ausführung der relevanten Testkategorie $cat(smt\_solver\_integration).$ fiel während der Entwicklung auf,
dass die Tests sporadisch nicht terminieren.
Mithilfe eines simplen Skripts wurde die Testkategorie endlos ausgeführt, um das Problem zu reproduzieren und einen Einblick in dessen Häufigkeit zu erhalten.
Am meisten war die Testnummer $2395$ betroffen, welche sich auf der Entwicklungsumgebung etwa alle $18$ Durchläufe nicht beendete. 
Nach einer Analyse des Problems stellte sich heraus, dass die Endlosschleife in \cref{lst:softlock} die Ursache war, welche in der Methode $reset$ ausgeführt wird.
Die Schleife wartet auf das Beenden aller Threads, die zur Lösung eines Prädikates gestartet wurden.

\begin{lstlisting}[
    float, caption={Problematische Endlosschleife.}, label={lst:softlock}, language=C++
  ]
send_solver_interrupts("all"); // implemented workaround

while (true) {
 { // wait for threads to finish
  std::lock_guard<std::mutex> threads_guard(running_threads_mutex);
  if (running_threads.size() == 0)
   break;
 }
 std::this_thread::sleep_for(std::chrono::milliseconds(5));
}
\end{lstlisting}

Mithilfe des GNU Debuggers (GDB) \cite{stallman1988debugging} konnte das Problem dahingehend eingegrenzt werden,
dass die laufenden Threads sich während der problematischen Endlosschleife innerhalb der Z3-Bibliothek befanden
und dort mutmaßlich feststeckten.
Genauer befanden sich die Threads in einer der beiden Methoden $sat::local\_search::flip\_walksat(unsigned int) ()$ oder $sat::local\_search::pick\_flip\_walksat() ()$.
Die Ursache für das Problem konnte nicht abschließend geklärt werden, jedoch wurde ein (unschöner) Workaround implementiert, der das Problem behebt.
Durch den in Zeile 1 als Kommentar dargestellten Aufruf zur Methode $send\_solver\_interrupts$ werden alle laufenden Threads unterbrochen und somit die Endlosschleife verlassen.
Da zum Zeitpunkt der Ausführung der $reset$ Funktion mindestens ein Thread bereits ein Ergebnis berechnet hat, stellt das Unterbrechen der anderen Threads kein Problem dar.

\subsection{Versionsinkompatibilität}
\todo[]{finde xyz version, evtl finde fehlermeldung}
Ein weiteres Problem, das während der Entwicklung auftrat, war die Inkompatibilität der Z3-Bibliothek ($z3lib.so$) mit dem Betriebssystem.
Diese Problematik trat auf der automatischen Testumgebung auf, welche die Darwin Plattform xyz innerhalb von Docker nutzt.
Anhand der Fehlermeldungen konnte festgestellt werden, dass sich die Inkompatibilität auf die glibc-Version zurückführen lässt.
Die Darwin-Plattform verwendete eine ältere Version von glibc, sodass die Z3-Bibiliothek auf inkorrekte oder nicht vorhandene Funktionen zugriff.
Das Problem wurde gelöst, indem die Testumgebung auf die neuere Systemversion Darwin $12$ aktualisiert wurde, welche mit der Z3-Bibliothek kompatibel ist.
