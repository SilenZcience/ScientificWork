


\section{Einführung}



\subsection{Motivation}


Jedoch birgt der Einsatz des Z3 Solvers auch Herausforderungen, die die Effizienz und Zuverlässigkeit der Anwendung beeinträchtigen können.
Ein bekanntes Problem besteht in dem sporadischen Auftreten von Speicherlecks und Segmentation Faults,
die sowohl die Stabilität als auch die Nutzbarkeit von ProB's Z3 Interface negativ beeinflussen.
Diese technischen Mängel erschweren nicht nur die Durchführung formaler Verifikationen,
sondern können auch zu einer zeitraubenden Verwendung der Z3 Solver Komponente sowie Unterbrechung von Arbeitsprozessen führen.

Ein weiterer Mangel liegt in der aktuellen sequentiellen Lösung mehrerer Prädikate.
Dieser Ansatz, bei dem die Prädikate nacheinander gelöst werden
ist in seiner Natur ressourcenintensiv und zeitaufwändig.
Angesichts der steigenden Komplexität formaler Modelle und der wachsenden Nachfrage nach schnellerer Verifikation wird die Limitierung durch die sequentielle Verarbeitung immer offensichtlicher.
Eine Parallelisierung der Lösung von Prädikaten könnte hier erhebliche Leistungsverbesserungen bringen,
indem moderne Mehrkernarchitekturen effizienter ausgenutzt werden.

Die Kombination dieser Herausforderungen (sporadische technische Instabilitäten und begrenzte Effizienz durch sequentielle Verarbeitung) macht es notwendig,
alternative Ansätze oder Verbesserungen für die Integration des Z3 Solvers in ProB zu erforschen.
Ziel ist es, sowohl die Zuverlässigkeit als auch die Leistung zu steigern,
um den Anforderungen der Nutzer und der immer komplexer werdenden Modelle gerecht zu werden.
Diese Problematik bildet die Grundlage und Motivation für die vorliegende Arbeit.
Sie zielt darauf ab, die Integration des Z3 Solvers in ProB zu verbessern,
indem die bestehende Vorgehensweise verworfen und durch eine neue Architektur ersetzt wird.


\subsection{Architekturänderung}

